% Created 2025-02-09 dom 14:20
% Intended LaTeX compiler: lualatex
\documentclass[11pt]{scrartcl}


\KOMAoptions{
%headings=chapterprefix,
twocolumn=true,
%toc=indenttextentries,
%toc=flat,
twoside=true,
headinclude=true,
footinclude=true
%  captions=topbeside
}
%\usepackage[fontsize=12.3]{scrextend}
\usepackage{fontspec}
\usepackage[T1]{fontenc}
\usepackage{hyperref}
\usepackage[x11names,svgnames,table]{xcolor}
\defaultfontfeatures{Ligatures=TeX}
%%\setmainfont{Lato}
%%\setmainfont{Charis SIL}
\setmainfont{IBM Plex Serif}
\usepackage{typearea}
\usepackage{lscape}
\usepackage[a4paper]{geometry}
\geometry{a4paper,total={170mm,257mm},left=10mm,right=10mm, top=15mm, bottom=20mm}
\usepackage[english, portuguese, american]{babel}
\usepackage{amsmath,amsfonts,amsthm,bm}
\usepackage{graphicx}
\usepackage{float,wrapfig}
\usepackage{colortbl}
\usepackage{tabularx}
\usepackage{pst-labo}
\usepackage{setspace}
\usepackage{xfrac}
\usepackage{tikz}
\usepackage{pgfplots}
\pgfplotsset{compat=1.3}
%% Diagraman latex
\usepackage{endiagram}
\usepackage{smartdiagram}
\usepackage[tikz]{bclogo}
\usetikzlibrary{fit,patterns,shadows.blur,shapes,decorations.pathreplacing,decorations.markings,arrows.meta,arrows,positioning,shadows,trees}
\usetikzlibrary{decorations.pathmorphing} %% to chemfig config bond
\usepackage{upgreek}
\usepackage[modules={all}]{chemmacros}
%%\chemsetup{modules={reactions,spectroscopy,thermodynamics,redox,isotopes}}
%%\chemsetup{modules={all}}
\NewChemState\EPot{ symbol=E , subscript-pos=right , superscript=o, pre= , unit=\volt }
%\usepackage[version=4,arrows=pgf-filled]{mhchem}
\usepackage{chemfig,elements,cancel,siunitx}
\NewChemPhase\lqdd{\(\ell\)}
\NewChemPhase\gr{grafite}
\NewChemPhase\reac{reação}
%\setchemfig{fixed length=false, atom sep=2.0em, arrow offset=6pt, scheme debug=false,angle increment=30}
\setchemfig{angle increment=30, atom sep=1.67em, double bond sep=0.67ex, bond style={line width=0.1em}, cram width=0.8ex, cram dash width=0.1em, cram dash sep=0.2em, arrow style={line width=0.067em},  arrow head=-{Triangle}, arrow label sep=1ex, cycle radius coeff=0.75, chemfig style={line width=0.1em}, }
\renewcommand{\CancelColor}{\color{red}}
\usepackage{circuitikz}
\usepackage{mol2chemfig}
\usepackage{subfig,caption}
\captionsetup{font=small, labelfont={bf,sf}}
\usepackage{wrapfig,qrcode}
\usepackage{array,longtable} % ajust colunm table
\newcolumntype{J}{>{\centering\arraybackslash}m{7.5cm}}
\newcolumntype{K}{>{\centering\arraybackslash}m{6.5cm}}
\newcolumntype{L}{>{\centering\arraybackslash}m{5cm}}
\newcolumntype{B}{>{\centering\arraybackslash}m{2.5cm}}
\newcolumntype{N}{>{\centering\arraybackslash}m{1.4cm}}
\usepackage[most]{tcolorbox}
\newcounter{mycounter}
%%% Colobor
%%% Example colorbox
\newtcolorbox{Box2}[2][]{
lower separated=false,
colback=white,
colframe=black,fonttitle=\bfseries,
colbacktitle=black,
coltitle=white,
enhanced, attach boxed title to top left={yshift=-0.1in,xshift=0.15in}, boxed title style={boxrule=0pt,colframe=white,}, title=#2,#1}
%%%%%%%% Cabecalho
\usepackage{framed,amsmath}
\newtcolorbox{mybox}[2][]{
enhanced,title=#2, fonttitle=\sffamily\small,
top=2pt,
bottom=1mm,
boxrule=0.4pt,
coltitle=black,
colback=white,
attach boxed title to top center={yshift=-\tcboxedtitleheight/2,
yshifttext=-\tcboxedtitleheight/2},
boxed title style={
colframe=white,
colback=white,
left=0.2pt,
right=0.2pt},
#1}
\usepackage{tabularray}
%%%%%%
\newtcolorbox{exercisebox}%
{enhanced,breakable,colback=white, colframe=green!15!white,colbacktitle=white!15!pink, coltitle=pink!50!black,left=0pt,right=0mm,top=3mm,bottom=3mm,pad at break=0pt,bottomrule at break=0pt,toprule at break=0pt,borderline={0mm}{0mm}{green!50!white,dashed}, attach boxed title to top center={yshift=-2mm},boxed title style={boxrule=0.4pt},title=Exercícios,}
\usepackage{eso-pic}
\usepackage{etoolbox}
\usepackage{enumitem}
\newcommand\circitem[1]{%
\tikz[baseline=(char.base)]{%https://tex.stackexchange.com/questions/204116/uniform-size-of-circles-around-enumitems
\node[circle,draw=gray, fill=gray!30,
minimum size=1.2em,inner sep=0] (char) {#1};}}
\newcommand\boxitem[1]{%
\tikz[baseline=(char.base)]{%https://tex.stackexchange.com/questions/204116/uniform-size-of-circles-around-enumitems
\node[fill=orange!30,
minimum size=1.2em,inner sep=0] (char) {#1};}}
%\usepackage{widetext}% needs packages "flushend" & "cuted" of "sttools" % bundle, which perhaps must separately be installed
\newcommand{\dd}[1]{\hspace{2pt}d#1}
\definecolor{color1}{RGB}{0,0,90} % Color of the article title and sections
\definecolor{color2}{RGB}{0,20,20} % Color of the boxes behind the abstract and
\definecolor{cinza}{HTML}{C0C0C0}
%%% Custom Exercios
\usepackage{bohr}
\usepackage{multicol}
\setlength{\columnsep}{1.5cm}
\setlength{\columnseprule}{0.2pt}
\usepackage[no-files]{xsim}
\usepackage{tasks}
\xsimsetup{
goal-print={\pgfmathprintnumber[fixed zerofill,precision=1]{#1}}
}
\newcommand*\circled[2]{\tikz[baseline=(char.base)]{
\node[shape=circle,fill,inner sep=2pt, text=white] (char) {#1};}}
%%%%%-Custom Xsim exercises %%%%%
\DeclareExerciseEnvironmentTemplate{custom}
{%\item[\GetExerciseProperty{counter}]
\Needspace*{0\baselineskip}
\noindent
\circled{\XSIMmixedcase{\GetExerciseProperty{counter}}}~~~%
\noindent
\IfInsideSolutionF{%
\GetExercisePropertyT{points}{ % notice the space
(%
\printgoal{\PropertyValue}
\IfExerciseGoalSingularTF{points}
{%\XSIMtranslate{point}
}
{% \XSIMtranslate{points}
}%
)%
}
}}
{\vspace{\baselineskip}}
%%%%%------- Custom  resposta -------%%%%%%%
\DeclareExerciseEnvironmentTemplate{space}
%{\textbf{\GetExerciseProperty{counter}} }
{\noindent\circled{\XSIMmixedcase{\GetExerciseProperty{counter}}}~~~}
% {\circled{\XSIMmixedcase{\GetExerciseProperty{counter}}}}~~~%
{\qquad}
\newcommand*\answer[1]{%
\XSIMexpandcode{%
\SetExerciseProperty{solution-body}
{\noexpand{\Alph{task}}}}%
#1%
}
%\sisetup{locale=DE}
\xsimsetup{
collect = true,
exercise/within = section,
exercise/template = custom,
exercise/the-counter =  \arabic{exercise},
solution/template= custom ,
%%solution-name = solution,  % used with headings=true
%solution/print=true,
%print-collection/print=both,
%print-solutions/collection=true
%goal-print= {\pgfmathprintnumber[fixed zerofill,precision=1]\num{#1}}
}
\RenewDocumentCommand\printpoints{}{%
\TotalExerciseTypeGoal{exercise}{points}{}{}%
}
\NewTasksEnvironment[label = (\emph{\alph*}), label-width = 12pt]{choice}[\choice]
\newenvironment{questions}{\itemize}{\enditemize}
\everymath{\displaystyle}
\DeclareExerciseHeadingTemplate{solution}{%
\section*{Gabarito}%
}
%\usepackage{filecontents}
\newcommand{\lh}{\underline{\hspace{1cm}}}
%%\onehalfspacing
\def\professor{Fábio Lima}
\def\aluno{ }
\def\numerochamada{}
\def\disciplina{Química}
%%\def\disciplina{UC III}
%%\def\disciplina{R.A.}
\def\turma{}
\def\tipo{{\bfseries Avaliação Bimestral}}
%%\def\tipo{\bfseries Avaliação Mensal}
\def\tipo{\bfseries Exame Final}
\def\bimestre{4 Bimestre}
\def\escola{E.E. 26 de Agosto}
%\def\escola{E.E. José Mamede de Aquino}
%\def\escola{E.E. Amélio de Carvalho Baís}
\def\dataprova{}
\DeclareExerciseCollection{kps}
\author{fabio}
\date{\today}
\title{}
\hypersetup{
 pdfauthor={fabio},
 pdftitle={},
 pdfkeywords={},
 pdfsubject={},
 pdfcreator={Emacs 29.4 (Org mode 9.6.15)}, 
 pdflang={English}}
\begin{document}

\twocolumn[
\input{../Modelos/CabeOficial}
%\input{../Modelos/cabenovo}
%\input{../Modelos/mamede}
%\input{../Modelos/26agosto}
%% \input{../Modelos/geral}
%Cada questão vale {\textbf 2,0}

%%\section*{Regime de Progressão Parcial}
%\section*{Atividade}
%\section*{Trabalho}
%%\section*{\disciplina}

%{\bfseries Obrigatório a resolução das questões }

%\input{../Modelos/gabarito}

%%Total Prova: \printpoints
\smallbreak
\medbreak
\par\vspace{2ex}]%%%%\input{../Modelos/mamede}




\collectexercises{kps}

\begin{exercise}
Escreva a expressão da constante de solubilidade dos seguintes compostos

\begin{choice}
\choice \ch{BaCO3} 
\choice \ch{CaSO4} 
\choice \ch{Al$\ell$(OH)3} 
\choice \ch{Ag2CrO4} 
\choice CoS 
\choice \ch{Mg(OH)2} 
\choice \ch{Fe(OH)3} 
\choice \ch{Zn(OH)2} 
\choice \ch{PbC$\ell$2} 
\choice \ch{Ag2S} 
\choice \ch{Pb3(PO4)2} 
\choice \ch{CaF2} 
\choice \ch{PbBr4} 
\choice \ch{Fe4(P2O7)3} 
\choice FeS
\end{choice}
\end{exercise}

\begin{exercise}
Sabendo que a solubilidade do cromato de prata – \ch{Ag2CrO4} – é de \num{2.5e-2} g/L, a determinada temperatura, calcular o seu produto de solubilidade nessa temperatura.
\end{exercise}
\begin{solution}
\num{1.7e-12} \unit{\mol\per\litre}
\end{solution}

\begin{exercise}
Calcule o K\(_{ps}\) do fosfato de prata – \ch{Ag3PO4}, sabendo que a sua solubilidade é de \num{6.5e-3} g/L.
\end{exercise}
\begin{solution}
\num{1.59e-18}
\end{solution}

\begin{exercise}
A solubilidade do fosfato de chumbo II – \ch{Pb3(PO4)2} – é de \num{1.4e-4} g/L. Determine o k\textsubscript{s} desse sal.
\end{exercise}
\begin{solution}
\num{1.62e-32}
\end{solution}

\begin{exercise}
Determine o produto de solubilidade do cloreto de chumbo II – \ch{PbC$\ell$2}  – cuja solubilidade é de 11 \unit{\gram\per\litre}
\end{exercise}
\begin{solution}
\num{2.37e-4}
\end{solution}

\begin{exercise}
A solubilidade do hidróxido férrico – \ch{Fe(OH)3} – em uma determinada temperatura é de \num{4.82e-8} \unit{\gram\per\litre}. Determine o seu produto de solubilidade nessa temperatura. 
\end{exercise}
\begin{solution}
\num{1.1e-36}
\end{solution}

\begin{exercise}
A solubilidade do carbonato de bário – \ch{BaCO3} – é de \num{1.3e-2} \unit{\mol\per\litre}. Descubra o produto de solubilidade desse sal.
\end{exercise}
\begin{solution}
\num{1.69e-4}
\end{solution}

\begin{exercise}
Prepara-se um litro de uma solução saturada de cloreto de prata – \ch{AgC$\ell$} . Calcule a constante de solubilidade, sabendo que foi dissolvido \num{1.7e-3} g desse sal.
\end{exercise}
\begin{solution}
\num{1.4e-10}
\end{solution}

\begin{exercise}
Determinar a solubilidade (em \unit{\mol\per\litre}) do sulfeto de prata (\ch{Ag2S}) a certa temperatura, sabendo que o produto de solubilidade nessa temperatura é \num{1.6e-48}.
\end{exercise}
\begin{solution}
\num{7.7e-17} mol/L
\end{solution}


\begin{exercise}
Sabe-se que o K\textsubscript{s} do fosfato de prata – \ch{Ag3PO4}, a 25O ºC, é \num{1.56e-48}. Determine a solubilidade do sal nessa temperatura.
\end{exercise}
\begin{solution}
\num{4.8e-13}
\end{solution}


\begin{exercise}
Calcule a solubilidade em água, a 18O ºC, do sulfeto férrico – \ch{Fe2S3} – sabendo que nessa temperatura o seu produto de solubilidade é \num{3.456e-52}.
\end{exercise}
\begin{solution}
\num{2e-11}
\end{solution}


\begin{exercise}
O produto de solubilidade, a 20O ºC, do hidróxido férrico – \ch{Fe(OH)3} – é \num{1.16e-16}. Descubra a solubilidade, a 20O ºC, em g/L, dessa base.
\end{exercise}
\begin{solution}
\num{4.8e-3} g/L
\end{solution}

\begin{exercise}
Calcule a solubilidade do sulfeto de manganês II – MnS – em g/L, sabendo que o seu produto de solubilidade é \num{2e-15} .
\end{exercise}
\begin{solution}
\num{3.88e-6}
\end{solution}


\begin{exercise}
Descubra a massa de sulfeto de ferro II – FeS – necessária para que sejam preparados 500 mL de solução saturada desse sal, sabendo que o seu produto de solubilidade é \num{4e-19}.
\end{exercise}
\begin{solution}
\num{2.78e-8}
\end{solution}

\begin{exercise}
Quantos gramas de sulfato de bário – \ch{BaSO4} – são necessários para preparar um litro de solução saturada, sabendo que o produto de solubilidade desse sal é \num{1.1e-10}?
\end{exercise}
\begin{solution}
\num{2.44e-3}
\end{solution}

\begin{exercise}
A dose letal de íons  \ch{Ba^{+2}} para o ser humano é de \num{2e-3} mol de íons  \ch{Ba^{+2}} por litro de sangue. Para se submeter a um exame de raios-X, um paciente ingeriu 200 mL de uma suspensão de sulfato de bário – \ch{BaSO4} . Supondo-se que os íons  \ch{Ba^{+2}} solubilizados na suspensão foram integralmente absorvidos pelo organismo e dissolvidos em 5 litros de sangue, discuta se essa dose coloca em risco a vida do paciente, considerando que o produto de solubilidade do sulfato de bário é igual a \num{1e-10} \unit{\mol\per\litre}.
\end{exercise}
\begin{solution}
Não coloca em risco, pois [ \ch{Ba^{+2}}]=\num{4e-7} \unit{\mol\per\litre}
\end{solution}


\begin{exercise}
Em um béquer foram misturadas soluções aquosas de cloreto de potássio (\ch{KC$\ell$}), sulfato de sódio (\ch{Na2SO4}) e nitrato de prata (\ch{AgNO3}), ocorrendo, então, a formação de um precipitado branco, que se depositou no fundo de um béquer. A análise da solução sobrenadante revelou as seguintes concentrações: [\ch{Ag^+}]=\num{1.0e-3} \unit{\mol\per\litre}; [ \ch{SO4^{-2}}]=\num{1.0e-1} \unit{\mol\per\litre} e [\ch{C$\ell$^-}]=\num{1.6e-7} \unit{\mol\per\litre}. De que é constituído o sólido formado? Justifique com cálculos.

\begin{tblr}[
]{
colspec = {ccc}, colsep = 2mm, hlines = {1pt, white},
row{1} = {2em,azure2,fg=white,font=\bfseries},
}
%\hline
Composto & {Produto \\ de Solubilidade} & Cor \\ %x\hline
{Cloreto de Prata \\ AgC$\ell$} & \num{1.6e-10} & Branca \\ \hline
{Sulfato de Prata \\ \ch{Ag2SO4}} & \num{1.4e-5} & Branca\\ \hline
\end{tblr}
\end{exercise}
\begin{solution}
Note que para o sulfato de prata , o k\textsubscript{s} não é alcançado (\num{1.4e-5}), indicando que a presença de íons \ch{Ag^+} e \ch{C$\ell$^-} é a máxima possível na solução. Então, se houver formação de precipitado, este será de cloreto de prata.
\end{solution}

\begin{exercise}
Uma solução saturada de base, representada por \ch{X(OH)2}, tem pH=10 a 25O ºC. Qual o produto de solubilidade do \ch{X(OH)2}?
\end{exercise}
\begin{solution}
\num{5e-13}
\end{solution}






\collectexercisesstop{kps}

\printcollection{kps}
\end{document}
