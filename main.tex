% Created 2024-08-11 Sun 17:13
% Intended LaTeX compiler: lualatex
\documentclass[10pt]{scrartcl}


\KOMAoptions{
%headings=chapterprefix,
twocolumn=true,
%toc=indenttextentries,
%toc=flat,
twoside=true,
headinclude=true,
footinclude=true
%  captions=topbeside
}
%\usepackage[fontsize=12.3]{scrextend}
\usepackage{fontspec}
\usepackage[T1]{fontenc}
\usepackage{hyperref}
\usepackage[x11names,svgnames,table]{xcolor}
\defaultfontfeatures{Ligatures=TeX}
%%\setmainfont{Lato}
%%\setmainfont{Charis SIL}
\setmainfont{IBM Plex Serif}
\usepackage{typearea}
\usepackage{lscape}
\usepackage[a4paper]{geometry}
\geometry{a4paper,total={170mm,257mm},left=10mm,right=10mm, top=15mm, bottom=20mm}
\usepackage[english, portuguese, american]{babel}
\usepackage{amsmath,amsfonts,amsthm,bm}
\usepackage{graphicx}
\usepackage{float,wrapfig}
\usepackage{colortbl}
\usepackage{tabularx}
\usepackage{pst-labo}
\usepackage{setspace}
\usepackage{xfrac}
\usepackage{tikz}
\usepackage{pgfplots}
\pgfplotsset{compat=1.3}
%% Diagraman latex
\usepackage{endiagram}
\usepackage{smartdiagram}
\usepackage[tikz]{bclogo}
\usetikzlibrary{fit,patterns,shadows.blur,shapes,decorations.pathreplacing,decorations.markings,arrows.meta,arrows,positioning,shadows,trees}
\usetikzlibrary{decorations.pathmorphing} %% to chemfig config bond
\usepackage{upgreek}
\usepackage[modules={all}]{chemmacros}
%%\chemsetup{modules={reactions,spectroscopy,thermodynamics,redox,isotopes}}
%%\chemsetup{modules={all}}
\NewChemState\EPot{ symbol=E , subscript-pos=right , superscript=o, pre= , unit=\volt }
%\usepackage[version=4,arrows=pgf-filled]{mhchem}
\usepackage{chemfig,elements,cancel,siunitx}
\NewChemPhase\lqdd{\(\ell\)}
\NewChemPhase\gr{grafite}
\NewChemPhase\reac{reação}
\setchemfig{fixed length=false, atom sep=2.0em, arrow offset=6pt, scheme debug=false,angle increment=30}
\renewcommand{\CancelColor}{\color{red}}
\usepackage{circuitikz}
\usepackage{mol2chemfig}
\usepackage{subfig,caption}
\captionsetup{font=small, labelfont={bf,sf}}
\usepackage{wrapfig,qrcode}
\usepackage{array,longtable} % ajust colunm table
\newcolumntype{J}{>{\centering\arraybackslash}m{7.5cm}}
\newcolumntype{K}{>{\centering\arraybackslash}m{6.5cm}}
\newcolumntype{L}{>{\centering\arraybackslash}m{5cm}}
\newcolumntype{B}{>{\centering\arraybackslash}m{2.5cm}}
\newcolumntype{N}{>{\centering\arraybackslash}m{1.4cm}}
\usepackage[most]{tcolorbox}
\newcounter{mycounter}
%%% Colobor
%%% Example colorbox
\newtcolorbox{Box2}[2][]{
lower separated=false,
colback=white,
colframe=black,fonttitle=\bfseries,
colbacktitle=black,
coltitle=white,
enhanced, attach boxed title to top left={yshift=-0.1in,xshift=0.15in}, boxed title style={boxrule=0pt,colframe=white,}, title=#2,#1}
%%%%%%%% Cabecalho
\usepackage{framed,amsmath}
\newtcolorbox{mybox}[2][]{
enhanced,title=#2, fonttitle=\sffamily\small,
top=2pt,
bottom=1mm,
boxrule=0.4pt,
coltitle=black,
colback=white,
attach boxed title to top center={yshift=-\tcboxedtitleheight/2,
yshifttext=-\tcboxedtitleheight/2},
boxed title style={
colframe=white,
colback=white,
left=0.2pt,
right=0.2pt},
#1}
\usepackage{tabularray}
%%%%%%
\newtcolorbox{exercisebox}%
{enhanced,breakable,colback=white, colframe=green!15!white,colbacktitle=white!15!pink, coltitle=pink!50!black,left=0pt,right=0mm,top=3mm,bottom=3mm,pad at break=0pt,bottomrule at break=0pt,toprule at break=0pt,borderline={0mm}{0mm}{green!50!white,dashed}, attach boxed title to top center={yshift=-2mm},boxed title style={boxrule=0.4pt},title=Exercícios,}
\usepackage{eso-pic}
\usepackage{etoolbox}
\usepackage{enumitem}
\newcommand\circitem[1]{%
\tikz[baseline=(char.base)]{%https://tex.stackexchange.com/questions/204116/uniform-size-of-circles-around-enumitems
\node[circle,draw=gray, fill=gray!30,
minimum size=1.2em,inner sep=0] (char) {#1};}}
\newcommand\boxitem[1]{%
\tikz[baseline=(char.base)]{%https://tex.stackexchange.com/questions/204116/uniform-size-of-circles-around-enumitems
\node[fill=orange!30,
minimum size=1.2em,inner sep=0] (char) {#1};}}
%\usepackage{widetext}% needs packages "flushend" & "cuted" of "sttools" % bundle, which perhaps must separately be installed
\newcommand{\dd}[1]{\hspace{2pt}d#1}
\definecolor{color1}{RGB}{0,0,90} % Color of the article title and sections
\definecolor{color2}{RGB}{0,20,20} % Color of the boxes behind the abstract and
\definecolor{cinza}{HTML}{C0C0C0}
%%% Custom Exercios
\usepackage{bohr}
\usepackage{multicol}
\setlength{\columnsep}{1.5cm}
\setlength{\columnseprule}{0.2pt}
\usepackage[no-files]{xsim}
\usepackage{tasks}
\xsimsetup{
goal-print={\pgfmathprintnumber[fixed zerofill,precision=1]{#1}}
}
\newcommand*\circled[2]{\tikz[baseline=(char.base)]{
\node[shape=circle,fill,inner sep=2pt, text=white] (char) {#1};}}
%%%%%-Custom Xsim exercises %%%%%
\DeclareExerciseEnvironmentTemplate{custom}
{%\item[\GetExerciseProperty{counter}]
\Needspace*{0\baselineskip}
\noindent
\circled{\XSIMmixedcase{\GetExerciseProperty{counter}}}~~~%
\noindent
\IfInsideSolutionF{%
\GetExercisePropertyT{points}{ % notice the space
(%
\printgoal{\PropertyValue}
\IfExerciseGoalSingularTF{points}
{%\XSIMtranslate{point}
}
{% \XSIMtranslate{points}
}%
)%
}
}}
{\vspace{\baselineskip}}
%%%%%------- Custom  resposta -------%%%%%%%
\DeclareExerciseEnvironmentTemplate{space}
%{\textbf{\GetExerciseProperty{counter}} }
{\noindent\circled{\XSIMmixedcase{\GetExerciseProperty{counter}}}~~~}
% {\circled{\XSIMmixedcase{\GetExerciseProperty{counter}}}}~~~%
{\qquad}
\newcommand*\answer[1]{%
\XSIMexpandcode{%
\SetExerciseProperty{solution-body}
{\noexpand{\Alph{task}}}}%
#1%
}
%\sisetup{locale=DE}
\xsimsetup{
collect = true,
exercise/within = section,
exercise/template = custom,
exercise/the-counter =  \arabic{exercise},
solution/template= custom ,
%%solution-name = solution,  % used with headings=true
solution/print=false,
%print-collection/print=both,
%goal-print= {\pgfmathprintnumber[fixed zerofill,precision=1]\num{#1}}
}
\RenewDocumentCommand\printpoints{}{%
\TotalExerciseTypeGoal{exercise}{points}{}{}%
}
\NewTasksEnvironment[label = (\emph{\alph*}), label-width = 12pt]{choice}[\choice]
\newenvironment{questions}{\itemize}{\enditemize}
\everymath{\displaystyle}
\DeclareExerciseHeadingTemplate{solution}{%
\section*{Gabarito}%
}
%\usepackage{filecontents}
\newcommand{\lh}{\underline{\hspace{1cm}}}
%%\onehalfspacing
\def\professor{Fábio Lima}
\def\aluno{ ARIEL DA MATA FRANCO }
\def\numerochamada{05}
\def\disciplina{Química}
%%\def\disciplina{UC3}
%%\def\disciplina{R.A.}
\def\turma{3 Ano }
\def\tipo{{\bfseries Avaliação Bimestral}}
%\def\tipo{\bfseries Atividade}
%%\def\tipo{\bfseries Exame Final}
\def\bimestre{2 Bimestre}
%\def\escola{E.E. 26 de Agosto}
%\def\escola{E.E. José Mamede de Aquino}
\def\escola{E.E. Amélio de Carvalho Baís}
\def\dataprova{}
\DeclareExerciseCollection{LeiLavosier}
\date{\today}
\title{}
\hypersetup{
 pdfauthor={},
 pdftitle={},
 pdfkeywords={},
 pdfsubject={},
 pdfcreator={Emacs 29.4 (Org mode 9.6.15)}, 
 pdflang={English}}
\begin{document}

\twocolumn[
\input{../Modelos/CabeOficial}
%%\input{../Modelos/cabenovo}
%%%\input{../Modelos/mamede}
%\input{../Modelos/26agosto}
%% \input{../Modelos/geral}
%Cada questão vale {\textbf 2,0}

%%\section*{Regime de Progressão Parcial}
%\section*{Atividade}
%%%\section*{Trabalho}
%%\section*{\disciplina}



%\input{../Modelos/gabarito}

%Total Prova: \printpoints
\smallbreak
\medbreak
\par\vspace{2ex}]%%%%\input{../Modelos/mamede}





\collectexercises{LeiLavosier}


\begin{exercise}
Quando 32g de enxofre reagem 32g de oxigênio apresentando como único
produto o dióxido de enxofre, podemos afirmar, obedecendo a Lei de Lavoisier,
que a massa de dióxido de enxofre produzida é:
\end{exercise}




\begin{exercise}
Para obtermos 16 g de metano temos que fazer a reação entre 4 g de
hidrogênio e 12 g de carbono. Quais as massas de hidrogênio e carbono são
necessárias para a obtenção de 80 g de metano?
\end{exercise}




\begin{exercise}
Se 71 g de cloro reagem com 200 g de brometo de cálcio, originando 111 g
de cloreto de cálcio e 160 g de bromo, pergunta-se:

\begin{choice}
\choice Qual a massa de cloro necessária para se obter 480 g de bromo?
\choice  Qual a massa de cloreto de cálcio obtida nesse caso?
\end{choice}
\end{exercise}



\begin{exercise}
Verifica-se que 56 g de óxido de cálcio reagem completamente com 44 g de
gás carbônico. Pede-se:

\begin{choice}
\choice Qual a massa do produto formado?
\choice Qual a massa de óxido de cálcio necessária para se obter 25 g do
produto da reação?
\end{choice}
\end{exercise}





\begin{exercise}
Provoca-se reação da mistura formada por 10,0g de hidrogênio e 500g de cloro. Após a reação constata-se a presença de 145g de cloro sem reagir, junto com o produto obtido. Qual é a massa, em gramas, da substância formada?
\end{exercise}



\begin{exercise}
Nos carros movidos a álcool, o etanol reage com o oxigênio do ar, produzindo gás carbônico e vapor de água. Sabendo que 46g de álcool reagem com 96g de oxigênio produzindo 88g de gás carbônico, que massa de vapor d'água será produzida nesta reação?
\end{exercise}



\begin{exercise}
Analise os dados abaixo:


\begin{tblr}{
		vlines, hlines,
		colspec = {X[c]X[c]X[c]},
		%vline{2} = {1}{text=\clap{:}},
		vline{2} = {1}{text=\clap{\ch{+}}},
		vline{3} = {1}{text=\clap{\ch{->}}},
		%vline{5} = {1}{text=\clap{\ch{+}}},
	}
	  \ch{SO3} & \ch{H2O} & \ch{H2SO4}  \\
	  $x$ & 18 & 98  \\
	  120 & 27 & y \\
\end{tblr}

Quais são os valores de "x" e de "y" ?
\end{exercise}




\begin{exercise}
Em uma oficina, 22,4g de pregos são deixados expostos ao ar. Supondo que, os pregos sejam constituídos unicamente por átomos de ferro e, que após algumas semanas a massa dos mesmos pregos tenha aumentado para 32g, pergunta-se:

\begin{choice}
\choice Que massa de oxigênio foi envolvida no processo?
\choice Em que Lei das Combinações Químicas você se baseou para responder o item anterior?
\end{choice}
\end{exercise}



\begin{exercise}
Analise o quadro a seguir:

\begin{tblr}{
		vlines, hlines,
		colspec = {X[c]X[c]X[c]X[c]},
		%vline{2} = {1}{text=\clap{:}},
		vline{2} = {1}{text=\clap{\ch{+}}},
		vline{3} = {1}{text=\clap{\ch{->}}},
		vline{4} = {1}{text=\clap{\ch{+}}},
	}
	  \ch{NaOH} & \ch{HC$\ell$} & \ch{NaC$\ell$} & \ch{H2O}  \\
	  40g & 36,5 g & $x$ & 18 g    \\
	  120 g & $y$ & $z$ & $t$\\
\end{tblr}
\end{exercise}



\begin{exercise}
Por aquecimento, 50g de \ch{CaCO3} decompõe-se em 28g de CaO e 22g de \ch{CO2}. Que massas de CaO e de \ch{CO2} serão obtidos na decomposição de 200g de \ch{CaCO3}? Qual a Lei das Combinações permite tais conclusões?
\end{exercise}


\collectexercisesstop{LeiLavosier}

\printcollection{LeiLavosier}
\end{document}
