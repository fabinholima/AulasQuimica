% Created 2024-10-21 Mon 15:14
% Intended LaTeX compiler: lualatex
\documentclass[10pt]{scrartcl}


\KOMAoptions{
%headings=chapterprefix,
twocolumn=true,
%toc=indenttextentries,
%toc=flat,
twoside=true,
headinclude=true,
footinclude=true
%  captions=topbeside
}
%\usepackage[fontsize=12.3]{scrextend}
\usepackage{fontspec}
\usepackage[T1]{fontenc}
\usepackage{hyperref}
\usepackage[x11names,svgnames,table]{xcolor}
\defaultfontfeatures{Ligatures=TeX}
%%\setmainfont{Lato}
%%\setmainfont{Charis SIL}
\setmainfont{IBM Plex Serif}
\usepackage{typearea}
\usepackage{lscape}
\usepackage[a4paper]{geometry}
\geometry{a4paper,total={170mm,257mm},left=10mm,right=10mm, top=15mm, bottom=20mm}
\usepackage[english, portuguese, american]{babel}
\usepackage{amsmath,amsfonts,amsthm,bm}
\usepackage{graphicx}
\usepackage{float,wrapfig}
\usepackage{colortbl}
\usepackage{tabularx}
\usepackage{pst-labo}
\usepackage{setspace}
\usepackage{xfrac}
\usepackage{tikz}
\usepackage{pgfplots}
\pgfplotsset{compat=1.3}
%% Diagraman latex
\usepackage{endiagram}
\usepackage{smartdiagram}
\usepackage[tikz]{bclogo}
\usetikzlibrary{fit,patterns,shadows.blur,shapes,decorations.pathreplacing,decorations.markings,arrows.meta,arrows,positioning,shadows,trees}
\usetikzlibrary{decorations.pathmorphing} %% to chemfig config bond
\usepackage{upgreek}
\usepackage[modules={all}]{chemmacros}
%%\chemsetup{modules={reactions,spectroscopy,thermodynamics,redox,isotopes}}
%%\chemsetup{modules={all}}
\NewChemState\EPot{ symbol=E , subscript-pos=right , superscript=o, pre= , unit=\volt }
%\usepackage[version=4,arrows=pgf-filled]{mhchem}
\usepackage{chemfig,elements,cancel,siunitx}
\NewChemPhase\lqdd{\(\ell\)}
\NewChemPhase\gr{grafite}
\NewChemPhase\reac{reação}
\setchemfig{fixed length=false, atom sep=2.0em, arrow offset=6pt, scheme debug=false,angle increment=30}
\renewcommand{\CancelColor}{\color{red}}
\usepackage{circuitikz}
\usepackage{mol2chemfig}
\usepackage{subfig,caption}
\captionsetup{font=small, labelfont={bf,sf}}
\usepackage{wrapfig,qrcode}
\usepackage{array,longtable} % ajust colunm table
\newcolumntype{J}{>{\centering\arraybackslash}m{7.5cm}}
\newcolumntype{K}{>{\centering\arraybackslash}m{6.5cm}}
\newcolumntype{L}{>{\centering\arraybackslash}m{5cm}}
\newcolumntype{B}{>{\centering\arraybackslash}m{2.5cm}}
\newcolumntype{N}{>{\centering\arraybackslash}m{1.4cm}}
\usepackage[most]{tcolorbox}
\newcounter{mycounter}
%%% Colobor
%%% Example colorbox
\newtcolorbox{Box2}[2][]{
lower separated=false,
colback=white,
colframe=black,fonttitle=\bfseries,
colbacktitle=black,
coltitle=white,
enhanced, attach boxed title to top left={yshift=-0.1in,xshift=0.15in}, boxed title style={boxrule=0pt,colframe=white,}, title=#2,#1}
%%%%%%%% Cabecalho
\usepackage{framed,amsmath}
\newtcolorbox{mybox}[2][]{
enhanced,title=#2, fonttitle=\sffamily\small,
top=2pt,
bottom=1mm,
boxrule=0.4pt,
coltitle=black,
colback=white,
attach boxed title to top center={yshift=-\tcboxedtitleheight/2,
yshifttext=-\tcboxedtitleheight/2},
boxed title style={
colframe=white,
colback=white,
left=0.2pt,
right=0.2pt},
#1}
\usepackage{tabularray}
%%%%%%
\newtcolorbox{exercisebox}%
{enhanced,breakable,colback=white, colframe=green!15!white,colbacktitle=white!15!pink, coltitle=pink!50!black,left=0pt,right=0mm,top=3mm,bottom=3mm,pad at break=0pt,bottomrule at break=0pt,toprule at break=0pt,borderline={0mm}{0mm}{green!50!white,dashed}, attach boxed title to top center={yshift=-2mm},boxed title style={boxrule=0.4pt},title=Exercícios,}
\usepackage{eso-pic}
\usepackage{etoolbox}
\usepackage{enumitem}
\newcommand\circitem[1]{%
\tikz[baseline=(char.base)]{%https://tex.stackexchange.com/questions/204116/uniform-size-of-circles-around-enumitems
\node[circle,draw=gray, fill=gray!30,
minimum size=1.2em,inner sep=0] (char) {#1};}}
\newcommand\boxitem[1]{%
\tikz[baseline=(char.base)]{%https://tex.stackexchange.com/questions/204116/uniform-size-of-circles-around-enumitems
\node[fill=orange!30,
minimum size=1.2em,inner sep=0] (char) {#1};}}
%\usepackage{widetext}% needs packages "flushend" & "cuted" of "sttools" % bundle, which perhaps must separately be installed
\newcommand{\dd}[1]{\hspace{2pt}d#1}
\definecolor{color1}{RGB}{0,0,90} % Color of the article title and sections
\definecolor{color2}{RGB}{0,20,20} % Color of the boxes behind the abstract and
\definecolor{cinza}{HTML}{C0C0C0}
%%% Custom Exercios
\usepackage{bohr}
\usepackage{multicol}
\setlength{\columnsep}{1.5cm}
\setlength{\columnseprule}{0.2pt}
\usepackage[no-files]{xsim}
\usepackage{tasks}
\xsimsetup{
goal-print={\pgfmathprintnumber[fixed zerofill,precision=1]{#1}}
}
\newcommand*\circled[2]{\tikz[baseline=(char.base)]{
\node[shape=circle,fill,inner sep=2pt, text=white] (char) {#1};}}
%%%%%-Custom Xsim exercises %%%%%
\DeclareExerciseEnvironmentTemplate{custom}
{%\item[\GetExerciseProperty{counter}]
\Needspace*{0\baselineskip}
\noindent
\circled{\XSIMmixedcase{\GetExerciseProperty{counter}}}~~~%
\noindent
\IfInsideSolutionF{%
\GetExercisePropertyT{points}{ % notice the space
(%
\printgoal{\PropertyValue}
\IfExerciseGoalSingularTF{points}
{%\XSIMtranslate{point}
}
{% \XSIMtranslate{points}
}%
)%
}
}}
{\vspace{\baselineskip}}
%%%%%------- Custom  resposta -------%%%%%%%
\DeclareExerciseEnvironmentTemplate{space}
%{\textbf{\GetExerciseProperty{counter}} }
{\noindent\circled{\XSIMmixedcase{\GetExerciseProperty{counter}}}~~~}
% {\circled{\XSIMmixedcase{\GetExerciseProperty{counter}}}}~~~%
{\qquad}
\newcommand*\answer[1]{%
\XSIMexpandcode{%
\SetExerciseProperty{solution-body}
{\noexpand{\Alph{task}}}}%
#1%
}
%\sisetup{locale=DE}
\xsimsetup{
collect = true,
exercise/within = section,
exercise/template = custom,
exercise/the-counter =  \arabic{exercise},
solution/template= custom ,
%%solution-name = solution,  % used with headings=true
solution/print=false,
%print-collection/print=both,
%goal-print= {\pgfmathprintnumber[fixed zerofill,precision=1]\num{#1}}
}
\RenewDocumentCommand\printpoints{}{%
\TotalExerciseTypeGoal{exercise}{points}{}{}%
}
\NewTasksEnvironment[label = (\emph{\alph*}), label-width = 12pt]{choice}[\choice]
\newenvironment{questions}{\itemize}{\enditemize}
\everymath{\displaystyle}
\DeclareExerciseHeadingTemplate{solution}{%
\section*{Gabarito}%
}
%\usepackage{filecontents}
\newcommand{\lh}{\underline{\hspace{1cm}}}
%%\onehalfspacing
\def\professor{Fábio Lima}
\def\aluno{ }
\def\numerochamada{}
\def\disciplina{Química}
%%\def\disciplina{UC3}
%%\def\disciplina{R.A.}
\def\turma{3 Ano }
%\def\tipo{{\bfseries Avaliação Bimestral}}
\def\tipo{\bfseries Avaliação Mensal}
%%\def\tipo{\bfseries Exame Final}
\def\bimestre{4 Bimestre}
%%\def\escola{E.E. 26 de Agosto}
\def\escola{E.E. José Mamede de Aquino}
%%\def\escola{E.E. Amélio de Carvalho Baís}
\def\dataprova{}
\DeclareExerciseCollection{FuncoesOxigenadas}
\author{fabio}
\date{\today}
\title{}
\hypersetup{
 pdfauthor={fabio},
 pdftitle={},
 pdfkeywords={},
 pdfsubject={},
 pdfcreator={Emacs 29.4 (Org mode 9.6.15)}, 
 pdflang={English}}
\begin{document}

\twocolumn[
\input{../Modelos/CabeOficial}
%\input{../Modelos/cabenovo}
%\input{../Modelos/mamede}
%\input{../Modelos/26agosto}
%% \input{../Modelos/geral}
%Cada questão vale {\textbf 2,0}

%%\section*{Regime de Progressão Parcial}
%\section*{Atividade}
%\section*{Trabalho}
%%\section*{\disciplina}



%\input{../Modelos/gabarito}

%Total Prova: \printpoints
\smallbreak
\medbreak
\par\vspace{2ex}]%%%%\input{../Modelos/mamede}



\collectexercises{FuncoesOxigenadas}

\begin{exercise}
Identifique as funções oxigenadas nas estruturas.


\begin{choice}{2}
\choice \chemfig{-[:30,,,1]NH-[:90,,1]-[:30]-[:90](-[:150]=^[:90]-[:150]=^[:210]%
-[:270]=^[:330]-[:30])-[:30]O-[:330]=^[:270]-[:330]=^[:30](-[:330](-[:270]F%
)(-[:330]F)-[:30]F)-[:90]=^[:150](-[:210])}
\choice \chemfig{Cl-[:330]=^[:270]-[:330]=^[:30](-[:90]=^[:150]-[:210])-[:338.6]O%
-[:287.1]=_[:355.7]-[:295.7]=_[:235.7]-[:175.7]=_[:115.7](-[:55.7])%
-[:184.3]N=_[:132.9](-[:81.4])-[:197.1]N-[:137.1]-[:197.1]-[:257.1,,,2]HN%
-[:317.1,,2]-[:17.1](-[:77.1]\phantom{N})}
\choice \chemfig{-N-[:300]--[:60]N>:[:21.5]-[:330]-[:30]-[:90]N(-[:150]>:[:210](%
-[:158.5]=_[:240](-[:180](=_[:120]-[:60]=_-[:300])-[:240]\phantom{N})%
-[:300]\phantom{N})-[:270])-[:30]-[:330]-[:30]-[:330](=[:270]O)-[:30]%
=^[:330]-[:30]=^[:90](-[:30]F)-[:150]=^[:210](-[:270])}
\choice \chemfig{HO-[:300,,2]=_-[:300]=^[:240](-[:180]=_[:120]-[:60])%
-[:312]\mcfbelow{N}{H}-[:24]=^[:96](-[:168])-[:42]-[:342](<[:282,,,1]NH_2)%
-[:42](-[:342,,,1]OH)=[:102]O}
\choice \chemfig{>[:54]-[:300]->[:60]-[:120](-[:180](<[:132]-[:204]-[:276](<[:222]O%
-[:162]=[:222]O)-[:348])-[:240])<:[:60]--[:300]=_-[:300](=O)-[:240]-[:180]%
-[:120](-[:180])(-[:60])<[:240]}
\choice \chemfig{-[:29.2]N>[:50.3]-[:96.1,1.069]-[:298.7,0.88]>:[:264.3,0.943](%
-[:188.1,0.864]\phantom{N})-[:329,1.126](<:[:279.7](-[:339.7]O-[:279.7])%
=[:219.7]O)-[:50.4,1.005](-[:189.7,0.813]-[:150,1.178])<:[:30.1]O-[:330.1](%
=[:270.1]O)-[:30.1]=^[:330.1]-[:30.1]=^[:90.1]-[:150.1]=^[:210.1](-[:270.1]%
)}
\choice \chemfig{-[:300]-(=[:60]O)-[:300]O--[:300](=[:240]O)>(-[:264]O-[:224](%
-[:284]-[:224])=[:164]O)-[:96](<[:150])-[:24]>[:312]-[:240](-[:168])(%
<[:244])-[:300]-(<[:300,,,1]OH)-[:60](<:[:280]Cl)-[:120](-[:180])<:[:60]-%
-[:300]=_-[:300](=O)-[:240]=_[:180]-[:120](-[:60])(-[:180])<[:260]}
\choice \chemfig{CH_3-CH_2-CH_2-C([:60]=O)([:-30]-OH)}
\choice \chemfig{CH_3-CH([:90]-CH_3)-CH_2-CH_2-C([:60]=O)([:-30]-H)}
\choice \chemfig{-[:30](<[:90])-[:330](<:[:270])-[:30](=[:90]O)-[:330]O-[:30]-[:330]}
\choice \chemfig{-[:30]-[:330](<[:270])-[:30](-[:330]-[:30](=[:90]O)-[:330]H)-[:90]-[:30]=^[:90]-[:150]=^[:210]-[:270](=^[:330])}
\choice \chemfig{H-[:30](=[:90]O)-[:330](<[:270])-[:30]-[:330]-[:30]-[:330]-[:30]}
\choice \chemfig{-[:330](-[:270])-[:30](-[:90](-[:30])-[:150])-[:330](-[:30](-[:90])-[:330])-[:270]-[:330,,,1]OH}
\choice \chemfig{H_3C-[:330,,2]-[:30]-[:330](-[:270,,,1]CH_3)-[:30](-[:90](-[:30,,,1]CH_3)-[:150,,,2]H_3C)-[:330](=[:270]O)-[:30,,,1]OH}
\choice \chemfig{-[:30]-[:330](-[:270,,,1]OH)-[:30](-[:90,,,1]OH)-[:330]-[:30]-[:330]}
\choice \chemfig{-[:330]-[:30](-[:90,,,1]OH)-[:330]-[:30]}
\choice \chemfig{O=[:270]-[:216]-[:288]--[:72](-[:144])}
\choice \chemfig{-[:330]-[:30](=[:90]O)-[:330]O-[:30]-[:330]}
\choice \chemfig{-[:30]-[:330]O-[:30]-[:330](-[:270])-[:30]-[:330]}
\choice \chemfig{OH-[:270,,1]=_[:330]-[:270](-[:330])-[:210]-[:150](-[:210])-[:90](-[:30])}
\choice \chemfig{OH-[:270,,1]-[:330]=_[:270](-[:330](-[:270])-[:30])-[:210]=_[:150]-[:90](=_[:30])}
\choice \chemfig{OH-[:150,,1](=[:90]O)-[:210]-[:150]-[:210](-[:150]-[:210]-[:150]-[:210]-[:150])-[:270]-[:330]-[:270]=^[:330]-[:30]=^[:90]-[:150](=^[:210])}
\choice \chemfig{H-[:150](=[:90]O)-[:210]=_[:270]-[:210]=_[:150]-[:90]=_[:30](-[:330])}

\end{choice}
\
\end{exercise}





\collectexercisesstop{FuncoesOxigenadas}

\printcollection{FuncoesOxigenadas}
\end{document}
