% Created 2024-08-29 Thu 18:32
% Intended LaTeX compiler: lualatex
\documentclass[10pt]{scrartcl}


\KOMAoptions{
%headings=chapterprefix,
twocolumn=true,
%toc=indenttextentries,
%toc=flat,
twoside=true,
headinclude=true,
footinclude=true
%  captions=topbeside
}
%\usepackage[fontsize=12.3]{scrextend}
\usepackage{fontspec}
\usepackage[T1]{fontenc}
\usepackage{hyperref}
\usepackage[x11names,svgnames,table]{xcolor}
\defaultfontfeatures{Ligatures=TeX}
%%\setmainfont{Lato}
%%\setmainfont{Charis SIL}
\setmainfont{IBM Plex Serif}
\usepackage{typearea}
\usepackage{lscape}
\usepackage[a4paper]{geometry}
\geometry{a4paper,total={170mm,257mm},left=10mm,right=10mm, top=15mm, bottom=20mm}
\usepackage[english, portuguese, american]{babel}
\usepackage{amsmath,amsfonts,amsthm,bm}
\usepackage{graphicx}
\usepackage{float,wrapfig}
\usepackage{colortbl}
\usepackage{tabularx}
\usepackage{pst-labo}
\usepackage{setspace}
\usepackage{xfrac}
\usepackage{tikz}
\usepackage{pgfplots}
\pgfplotsset{compat=1.3}
%% Diagraman latex
\usepackage{endiagram}
\usepackage{smartdiagram}
\usepackage[tikz]{bclogo}
\usetikzlibrary{fit,patterns,shadows.blur,shapes,decorations.pathreplacing,decorations.markings,arrows.meta,arrows,positioning,shadows,trees}
\usetikzlibrary{decorations.pathmorphing} %% to chemfig config bond
\usepackage{upgreek}
\usepackage[modules={all}]{chemmacros}
%%\chemsetup{modules={reactions,spectroscopy,thermodynamics,redox,isotopes}}
%%\chemsetup{modules={all}}
\NewChemState\EPot{ symbol=E , subscript-pos=right , superscript=o, pre= , unit=\volt }
%\usepackage[version=4,arrows=pgf-filled]{mhchem}
\usepackage{chemfig,elements,cancel,siunitx}
\NewChemPhase\lqdd{\(\ell\)}
\NewChemPhase\gr{grafite}
\NewChemPhase\reac{reação}
\setchemfig{fixed length=false, atom sep=2.0em, arrow offset=6pt, scheme debug=false,angle increment=30}
\renewcommand{\CancelColor}{\color{red}}
\usepackage{circuitikz}
\usepackage{mol2chemfig}
\usepackage{subfig,caption}
\captionsetup{font=small, labelfont={bf,sf}}
\usepackage{wrapfig,qrcode}
\usepackage{array,longtable} % ajust colunm table
\newcolumntype{J}{>{\centering\arraybackslash}m{7.5cm}}
\newcolumntype{K}{>{\centering\arraybackslash}m{6.5cm}}
\newcolumntype{L}{>{\centering\arraybackslash}m{5cm}}
\newcolumntype{B}{>{\centering\arraybackslash}m{2.5cm}}
\newcolumntype{N}{>{\centering\arraybackslash}m{1.4cm}}
\usepackage[most]{tcolorbox}
\newcounter{mycounter}
%%% Colobor
%%% Example colorbox
\newtcolorbox{Box2}[2][]{
lower separated=false,
colback=white,
colframe=black,fonttitle=\bfseries,
colbacktitle=black,
coltitle=white,
enhanced, attach boxed title to top left={yshift=-0.1in,xshift=0.15in}, boxed title style={boxrule=0pt,colframe=white,}, title=#2,#1}
%%%%%%%% Cabecalho
\usepackage{framed,amsmath}
\newtcolorbox{mybox}[2][]{
enhanced,title=#2, fonttitle=\sffamily\small,
top=2pt,
bottom=1mm,
boxrule=0.4pt,
coltitle=black,
colback=white,
attach boxed title to top center={yshift=-\tcboxedtitleheight/2,
yshifttext=-\tcboxedtitleheight/2},
boxed title style={
colframe=white,
colback=white,
left=0.2pt,
right=0.2pt},
#1}
\usepackage{tabularray}
%%%%%%
\newtcolorbox{exercisebox}%
{enhanced,breakable,colback=white, colframe=green!15!white,colbacktitle=white!15!pink, coltitle=pink!50!black,left=0pt,right=0mm,top=3mm,bottom=3mm,pad at break=0pt,bottomrule at break=0pt,toprule at break=0pt,borderline={0mm}{0mm}{green!50!white,dashed}, attach boxed title to top center={yshift=-2mm},boxed title style={boxrule=0.4pt},title=Exercícios,}
\usepackage{eso-pic}
\usepackage{etoolbox}
\usepackage{enumitem}
\newcommand\circitem[1]{%
\tikz[baseline=(char.base)]{%https://tex.stackexchange.com/questions/204116/uniform-size-of-circles-around-enumitems
\node[circle,draw=gray, fill=gray!30,
minimum size=1.2em,inner sep=0] (char) {#1};}}
\newcommand\boxitem[1]{%
\tikz[baseline=(char.base)]{%https://tex.stackexchange.com/questions/204116/uniform-size-of-circles-around-enumitems
\node[fill=orange!30,
minimum size=1.2em,inner sep=0] (char) {#1};}}
%\usepackage{widetext}% needs packages "flushend" & "cuted" of "sttools" % bundle, which perhaps must separately be installed
\newcommand{\dd}[1]{\hspace{2pt}d#1}
\definecolor{color1}{RGB}{0,0,90} % Color of the article title and sections
\definecolor{color2}{RGB}{0,20,20} % Color of the boxes behind the abstract and
\definecolor{cinza}{HTML}{C0C0C0}
%%% Custom Exercios
\usepackage{bohr}
\usepackage{multicol}
\setlength{\columnsep}{1.5cm}
\setlength{\columnseprule}{0.2pt}
\usepackage[no-files]{xsim}
\usepackage{tasks}
\xsimsetup{
goal-print={\pgfmathprintnumber[fixed zerofill,precision=1]{#1}}
}
\newcommand*\circled[2]{\tikz[baseline=(char.base)]{
\node[shape=circle,fill,inner sep=2pt, text=white] (char) {#1};}}
%%%%%-Custom Xsim exercises %%%%%
\DeclareExerciseEnvironmentTemplate{custom}
{%\item[\GetExerciseProperty{counter}]
\Needspace*{0\baselineskip}
\noindent
\circled{\XSIMmixedcase{\GetExerciseProperty{counter}}}~~~%
\noindent
\IfInsideSolutionF{%
\GetExercisePropertyT{points}{ % notice the space
(%
\printgoal{\PropertyValue}
\IfExerciseGoalSingularTF{points}
{%\XSIMtranslate{point}
}
{% \XSIMtranslate{points}
}%
)%
}
}}
{\vspace{\baselineskip}}
%%%%%------- Custom  resposta -------%%%%%%%
\DeclareExerciseEnvironmentTemplate{space}
%{\textbf{\GetExerciseProperty{counter}} }
{\noindent\circled{\XSIMmixedcase{\GetExerciseProperty{counter}}}~~~}
% {\circled{\XSIMmixedcase{\GetExerciseProperty{counter}}}}~~~%
{\qquad}
\newcommand*\answer[1]{%
\XSIMexpandcode{%
\SetExerciseProperty{solution-body}
{\noexpand{\Alph{task}}}}%
#1%
}
%\sisetup{locale=DE}
\xsimsetup{
collect = true,
exercise/within = section,
exercise/template = custom,
exercise/the-counter =  \arabic{exercise},
solution/template= custom ,
%%solution-name = solution,  % used with headings=true
solution/print=false,
%print-collection/print=both,
%goal-print= {\pgfmathprintnumber[fixed zerofill,precision=1]\num{#1}}
}
\RenewDocumentCommand\printpoints{}{%
\TotalExerciseTypeGoal{exercise}{points}{}{}%
}
\NewTasksEnvironment[label = (\emph{\alph*}), label-width = 12pt]{choice}[\choice]
\newenvironment{questions}{\itemize}{\enditemize}
\everymath{\displaystyle}
\DeclareExerciseHeadingTemplate{solution}{%
\section*{Gabarito}%
}
%\usepackage{filecontents}
\newcommand{\lh}{\underline{\hspace{1cm}}}
%%\onehalfspacing
\def\professor{Fábio Lima}
\def\aluno{ ARIEL DA MATA FRANCO }
\def\numerochamada{05}
\def\disciplina{Química}
%%\def\disciplina{UC3}
%%\def\disciplina{R.A.}
\def\turma{3 Ano }
\def\tipo{{\bfseries Avaliação Bimestral}}
%\def\tipo{\bfseries Atividade}
%%\def\tipo{\bfseries Exame Final}
\def\bimestre{2 Bimestre}
%\def\escola{E.E. 26 de Agosto}
%\def\escola{E.E. José Mamede de Aquino}
\def\escola{E.E. Amélio de Carvalho Baís}
\def\dataprova{}
\DeclareExerciseCollection{CadeiasCarbonicasII}
\date{\today}
\title{}
\hypersetup{
 pdfauthor={},
 pdftitle={},
 pdfkeywords={},
 pdfsubject={},
 pdfcreator={Emacs 29.4 (Org mode 9.6.15)}, 
 pdflang={English}}
\begin{document}

\twocolumn[
\input{../Modelos/CabeOficial}
%%\input{../Modelos/cabenovo}
%%%\input{../Modelos/mamede}
%\input{../Modelos/26agosto}
%% \input{../Modelos/geral}
%Cada questão vale {\textbf 2,0}

%%\section*{Regime de Progressão Parcial}
%\section*{Atividade}
%%%\section*{Trabalho}
%%\section*{\disciplina}



%\input{../Modelos/gabarito}

%Total Prova: \printpoints
\smallbreak
\medbreak
\par\vspace{2ex}]%%%%\input{../Modelos/mamede}




\collectexercises{CadeiasCarbonicasII}


\begin{exercise}[points=1]
O gosto amargo da cerveja se deve ao mirceno, substância proveniente das folhas de lúpulo que é adicionada à bebida durante sua fabricação. Em relação à estrutura desse composto, é correto afirmar que possui.

\chemfig{-[:30](-[:90])=[:330]-[:30]-[:330](=[:270])-[:30]=[:330]}

\begin{choice}
\choice  fórmula molecular \ch{C9H12}.
\choice  dois carbonos assimétricos.
\choice três ligações pi e vinte sigma.
\choice cadeia carbônica linear e saturada.
\choice três carbonos híbridos \(sp^3\) e seis \(sp^2\)
\end{choice}
\end{exercise}
\begin{solution}

\end{solution}





\begin{exercise}[points=1]
Isoflavonas são compostos encontrados em sementes de soja associados a proteínas. Seu consumo
frequente pode auxiliar as mulheres a minimizar os efeitos negativos da menopausa. A estrutura química de uma isoflavona está representada abaixo: Sua estrutura química relativamente plana, fundamental em suas propriedades, é uma conseqüência das características dos carbonos envolvidos.
Considerando-se a estrutura da isoflavona, quantos carbonos terciários podem ser evidenciados?

\chemfig{OH-[:150,,1]=_[:210]-[:150]=_[:90](-[:30]=_[:330]-[:270])-[:150]%
-[:210](=[:270]O)-[:150]=_[:210](-[:270,,,1]OH)-[:150]=_[:90](-[:150,,,2]HO%
)-[:30]=_[:330](-[:270])-[:30]O-[:330](=_[:270])}

\begin{choice}(2)
\choice 0
\choice 1
\choice 2
\choice 3
\choice 4
\end{choice}
\end{exercise}

\begin{solution}

\end{solution}






\begin{exercise}[points=1]
Muitos inseticidas utilizados na agricultura e no ambiente doméstico pertencem à classe de compostos denominados piretróides. Dentre os muitos piretróides disponíveis comercialmente, encontra-se a deltametrina, cujo isômero mais potente tem sua fórmula estrutural representada a seguir:

\chemfig{-[:230](-[:310])-[:210](<[:240]-[:300,,,,dlh](-[:240]Br)-Br)-[:90]%
(-[:330])<[:120](=[:60]O)-[:180]O>[:120](-[:60]-[:60,,,,trpl={0}{0}]N)%
-[:180]-[:240,,,,drh]-[:180](-[:120,,,,drh]-[:60]-[,,,,drh]-[:300])-[:240]O%
-[:180]-[:240,,,,drh]-[:180]-[:120,,,,drh]-[:60]-[,,,,drh](-[:300])}

\begin{choice}
\choice Existe um carbono quaternário.
\choice O composto apresenta dez ligações pi. 
\choice O composto possui três carbonos assimétricos.
\choice O composto possui sete carbonos quaternários.
\choice O composto possui quinze carbonos com hibridação sp\textsuperscript{2} e um carbono sp
\end{choice}
\end{exercise}
\begin{solution}

\end{solution}





\begin{exercise}[points=1]
O chá da planta Bidens pilosa, conhecida vulgarmente pelo nome de picão, é usado para combater icterícia de recém-nascidos. Das folhas dessa planta, é extraída uma substância química, cujo nome oficial é 1 - fenilepta - 1, 3, 5 - triino e cuja estrutura é apresentada abaixo. Essa substância possui propriedades antimicrobianas e, quando irradiada com luz ultravioleta, apresenta atividade contra larvas de mosquitos e nematóides.
Sobre a estrutura dessa substância, pode-se afirmar que:

%\chemfig{*6(-=-([:0]-~-~-~-)=-=)

\begin{choice}
\choice possui 12 átomos de carbono com hibridização sp\textsuperscript{2}.
\choice possui 12 ligações  carbono-carbono.
\choice não possui carbonos com hibridização sp\textsuperscript{3}.
\choice possui 3 átomos de carbono com hibridização sp.
\choice possui 9 ligações \(\pi\) carbono-carbono
\end{choice}
\end{exercise}
\begin{solution}

\end{solution}





\begin{exercise}[points=1]
A borracha natural é um líquido branco e leitoso, extraído da seringueira, conhecido como látex. O monômero que origina a borracha natural é o 2 -metil- 1, 3-butadieno. Sua fórmula estrutural está representada abaixo.

\chemfig{H_2C=C([:-90]-CH_3)-CH=CH_2}

Sobre a estrutura do monômero, é correto afirmar que:
\begin{choice}
\choice é um hidrocarboneto insaturado de fórmula molecular \ch{C5H8}.
\choice é um hidrocarboneto de cadeia saturada e ramificada.
\choice tem fórmula molecular \ch{C4H5}.
\choice apresenta dois carbonos terciários, um carbono secundário e dois carbonos primários.
\choice apresenta dois carbonos carbonos quartenários.
\end{choice}
\end{exercise}
\begin{solution}

\end{solution}





\begin{exercise}[points=1]
As fenil-ureias substituídas pertencem ao primeiro grupo de herbicidas de alta eficiência
introduzido em 1956, do qual o [3–(3,4–diclorofenil)–1,1–dimetiluréia] (DCMU) faz parte.



Em relação à molécula do DCMU, é correto afirmar, exceto:


A) Apresenta o fenômeno de ressonância.
B) Apresenta carbonos trigonais e tetraédricos.
C) Sua fórmula molecular é C8H10N2Cl2O.
D) Possui três hidrogênios ligados a carbonos aromáticos.
\end{exercise}
\begin{solution}

\end{solution}







\begin{exercise}[points=1]
A sibutramina é um fármaco controlado pela Agência Nacional de Vigilância Sanitária que tem por finalidade agir como moderador de apetite.
Sobre a sibutramina, é incorreto afirmar que:

\chemfig{-[:210](-[:150])-[:270]-[:330](-[:30]N(-[:90])-[:330])-[:270](-%
-[:270]-[:180]-[:90])-[:180]=_[:240]-[:180]=_[:120](-[:180]C{\ell})-[:60]=_(%
-[:300])}

\begin{choice}
\choice trata-se de uma substância aromática.
\choice sua fórmula molecular é \ch{C17H25NC$\ell$}
\choice  identifica-se um elemento da família dos halogênios em sua
estrutura.
\choice identifica-se a presença de ligações \(\pi\) (pi) em sua estrutura.
\choice O composto é um álcool
\end{choice}
\end{exercise}
\begin{solution}

\end{solution}





\begin{exercise}[points=1]
Considerando a metionina e a cisteína, assinale a afirmativa correta sobre suas estruturas.

\chemname{\chemfig{H_3C-S-CH_2-CH_2-CH([:-90]-NH_2)-COOH}}{Metionina} \par

\chemname{\chemfig{HS-CH_2-CH([:-90]-NH_2)-COOH}}{Cisteína}

A) Ambos os aminoácidos apresentam um átomo de carbono
cuja hibridização é sp2
e cadeia carbônica homogênea.
B) Ambos os aminoácidos apresentam um átomo de carbono
cuja hibridização é sp2
, mas a metionina tem cadeia
carbônica heterogênea e a cisteína, homogênea.
C) Ambos os aminoácidos apresentam um átomo de carbono
cuja hibridização é sp2
e cadeia carbônica heterogênea.
D) Ambos os aminoácidos apresentam os átomos de carbono
com hibridização sp e cadeia carbônica homogênea.
\end{exercise}
\begin{solution}

\end{solution}








\collectexercisesstop{CadeiasCarbonicasII}
\printcollection{CadeiasCarbonicasII}
\end{document}
