% Created 2024-10-22 Tue 06:36
% Intended LaTeX compiler: lualatex
\documentclass[9pt]{scrartcl}


\KOMAoptions{
%headings=chapterprefix,
twocolumn=true,
%toc=indenttextentries,
%toc=flat,
twoside=true,
headinclude=true,
footinclude=true
%  captions=topbeside
}
%\usepackage[fontsize=12.3]{scrextend}
\usepackage{fontspec}
\usepackage[T1]{fontenc}
\usepackage{hyperref}
\usepackage[x11names,svgnames,table]{xcolor}
\defaultfontfeatures{Ligatures=TeX}
%%\setmainfont{Lato}
%%\setmainfont{Charis SIL}
\setmainfont{IBM Plex Serif}
\usepackage{typearea}
\usepackage{lscape}
\usepackage[a4paper]{geometry}
\geometry{a4paper,total={170mm,257mm},left=10mm,right=10mm, top=15mm, bottom=20mm}
\usepackage[english, portuguese, american]{babel}
\usepackage{amsmath,amsfonts,amsthm,bm}
\usepackage{graphicx}
\usepackage{float,wrapfig}
\usepackage{colortbl}
\usepackage{tabularx}
\usepackage{pst-labo}
\usepackage{setspace}
\usepackage{xfrac}
\usepackage{tikz}
\usepackage{pgfplots}
\pgfplotsset{compat=1.3}
%% Diagraman latex
\usepackage{endiagram}
\usepackage{smartdiagram}
\usepackage[tikz]{bclogo}
\usetikzlibrary{fit,patterns,shadows.blur,shapes,decorations.pathreplacing,decorations.markings,arrows.meta,arrows,positioning,shadows,trees}
\usetikzlibrary{decorations.pathmorphing} %% to chemfig config bond
\usepackage{upgreek}
\usepackage[modules={all}]{chemmacros}
%%\chemsetup{modules={reactions,spectroscopy,thermodynamics,redox,isotopes}}
%%\chemsetup{modules={all}}
\NewChemState\EPot{ symbol=E , subscript-pos=right , superscript=o, pre= , unit=\volt }
%\usepackage[version=4,arrows=pgf-filled]{mhchem}
\usepackage{chemfig,elements,cancel,siunitx}
\NewChemPhase\lqdd{\(\ell\)}
\NewChemPhase\gr{grafite}
\NewChemPhase\reac{reação}
\setchemfig{fixed length=false, atom sep=2.0em, arrow offset=6pt, scheme debug=false,angle increment=30}
\renewcommand{\CancelColor}{\color{red}}
\usepackage{circuitikz}
\usepackage{mol2chemfig}
\usepackage{subfig,caption}
\captionsetup{font=small, labelfont={bf,sf}}
\usepackage{wrapfig,qrcode}
\usepackage{array,longtable} % ajust colunm table
\newcolumntype{J}{>{\centering\arraybackslash}m{7.5cm}}
\newcolumntype{K}{>{\centering\arraybackslash}m{6.5cm}}
\newcolumntype{L}{>{\centering\arraybackslash}m{5cm}}
\newcolumntype{B}{>{\centering\arraybackslash}m{2.5cm}}
\newcolumntype{N}{>{\centering\arraybackslash}m{1.4cm}}
\usepackage[most]{tcolorbox}
\newcounter{mycounter}
%%% Colobor
%%% Example colorbox
\newtcolorbox{Box2}[2][]{
lower separated=false,
colback=white,
colframe=black,fonttitle=\bfseries,
colbacktitle=black,
coltitle=white,
enhanced, attach boxed title to top left={yshift=-0.1in,xshift=0.15in}, boxed title style={boxrule=0pt,colframe=white,}, title=#2,#1}
%%%%%%%% Cabecalho
\usepackage{framed,amsmath}
\newtcolorbox{mybox}[2][]{
enhanced,title=#2, fonttitle=\sffamily\small,
top=2pt,
bottom=1mm,
boxrule=0.4pt,
coltitle=black,
colback=white,
attach boxed title to top center={yshift=-\tcboxedtitleheight/2,
yshifttext=-\tcboxedtitleheight/2},
boxed title style={
colframe=white,
colback=white,
left=0.2pt,
right=0.2pt},
#1}
\usepackage{tabularray}
%%%%%%
\newtcolorbox{exercisebox}%
{enhanced,breakable,colback=white, colframe=green!15!white,colbacktitle=white!15!pink, coltitle=pink!50!black,left=0pt,right=0mm,top=3mm,bottom=3mm,pad at break=0pt,bottomrule at break=0pt,toprule at break=0pt,borderline={0mm}{0mm}{green!50!white,dashed}, attach boxed title to top center={yshift=-2mm},boxed title style={boxrule=0.4pt},title=Exercícios,}
\usepackage{eso-pic}
\usepackage{etoolbox}
\usepackage{enumitem}
\newcommand\circitem[1]{%
\tikz[baseline=(char.base)]{%https://tex.stackexchange.com/questions/204116/uniform-size-of-circles-around-enumitems
\node[circle,draw=gray, fill=gray!30,
minimum size=1.2em,inner sep=0] (char) {#1};}}
\newcommand\boxitem[1]{%
\tikz[baseline=(char.base)]{%https://tex.stackexchange.com/questions/204116/uniform-size-of-circles-around-enumitems
\node[fill=orange!30,
minimum size=1.2em,inner sep=0] (char) {#1};}}
%\usepackage{widetext}% needs packages "flushend" & "cuted" of "sttools" % bundle, which perhaps must separately be installed
\newcommand{\dd}[1]{\hspace{2pt}d#1}
\definecolor{color1}{RGB}{0,0,90} % Color of the article title and sections
\definecolor{color2}{RGB}{0,20,20} % Color of the boxes behind the abstract and
\definecolor{cinza}{HTML}{C0C0C0}
%%% Custom Exercios
\usepackage{bohr}
\usepackage{multicol}
\setlength{\columnsep}{1.5cm}
\setlength{\columnseprule}{0.2pt}
\usepackage[no-files]{xsim}
\usepackage{tasks}
\xsimsetup{
goal-print={\pgfmathprintnumber[fixed zerofill,precision=1]{#1}}
}
\newcommand*\circled[2]{\tikz[baseline=(char.base)]{
\node[shape=circle,fill,inner sep=2pt, text=white] (char) {#1};}}
%%%%%-Custom Xsim exercises %%%%%
\DeclareExerciseEnvironmentTemplate{custom}
{%\item[\GetExerciseProperty{counter}]
\Needspace*{0\baselineskip}
\noindent
\circled{\XSIMmixedcase{\GetExerciseProperty{counter}}}~~~%
\noindent
\IfInsideSolutionF{%
\GetExercisePropertyT{points}{ % notice the space
(%
\printgoal{\PropertyValue}
\IfExerciseGoalSingularTF{points}
{%\XSIMtranslate{point}
}
{% \XSIMtranslate{points}
}%
)%
}
}}
{\vspace{\baselineskip}}
%%%%%------- Custom  resposta -------%%%%%%%
\DeclareExerciseEnvironmentTemplate{space}
%{\textbf{\GetExerciseProperty{counter}} }
{\noindent\circled{\XSIMmixedcase{\GetExerciseProperty{counter}}}~~~}
% {\circled{\XSIMmixedcase{\GetExerciseProperty{counter}}}}~~~%
{\qquad}
\newcommand*\answer[1]{%
\XSIMexpandcode{%
\SetExerciseProperty{solution-body}
{\noexpand{\Alph{task}}}}%
#1%
}
%\sisetup{locale=DE}
\xsimsetup{
collect = true,
exercise/within = section,
exercise/template = custom,
exercise/the-counter =  \arabic{exercise},
solution/template= custom ,
%%solution-name = solution,  % used with headings=true
solution/print=false,
%print-collection/print=both,
%goal-print= {\pgfmathprintnumber[fixed zerofill,precision=1]\num{#1}}
}
\RenewDocumentCommand\printpoints{}{%
\TotalExerciseTypeGoal{exercise}{points}{}{}%
}
\NewTasksEnvironment[label = (\emph{\alph*}), label-width = 12pt]{choice}[\choice]
\newenvironment{questions}{\itemize}{\enditemize}
\everymath{\displaystyle}
\DeclareExerciseHeadingTemplate{solution}{%
\section*{Gabarito}%
}
%\usepackage{filecontents}
\newcommand{\lh}{\underline{\hspace{1cm}}}
%%\onehalfspacing
\def\professor{Fábio Lima}
\def\aluno{ }
\def\numerochamada{}
\def\disciplina{Química}
%%\def\disciplina{UC3}
%%\def\disciplina{R.A.}
\def\turma{3 Ano }
%\def\tipo{{\bfseries Avaliação Bimestral}}
\def\tipo{\bfseries Avaliação Mensal}
%%\def\tipo{\bfseries Exame Final}
\def\bimestre{4 Bimestre}
%%\def\escola{E.E. 26 de Agosto}
%\def\escola{E.E. José Mamede de Aquino}
\def\escola{E.E. Amélio de Carvalho Baís}
\def\dataprova{}
\DeclareExerciseCollection{FuncoesOxigenadasIII}
\DeclareExerciseCollection{FuncoesOxigenadasIV}
\date{\today}
\title{}
\hypersetup{
 pdfauthor={},
 pdftitle={},
 pdfkeywords={},
 pdfsubject={},
 pdfcreator={Emacs 29.4 (Org mode 9.6.15)}, 
 pdflang={English}}
\begin{document}

\twocolumn[
%%\input{../Modelos/CabeOficial}
\input{../Modelos/cabenovo}
%\input{../Modelos/mamede}
%\input{../Modelos/26agosto}
%% \input{../Modelos/geral}
%Cada questão vale {\textbf 2,0}

%%\section*{Regime de Progressão Parcial}
%\section*{Atividade}
%\section*{Trabalho}
%%\section*{\disciplina}



%\input{../Modelos/gabarito}

%Total Prova: \printpoints
\smallbreak
\medbreak
\par\vspace{2ex}]%%%%\input{../Modelos/mamede}




\collectexercises{FuncoesOxigenadasIII}



\begin{exercise}[points=1.0]
A baunilha é uma espécie de orquídea. A partir de sua flor, é produzida a vanilina (conforme representação química), que dá origem ao aroma de baunilha.

\begin{center}
\chemfig{OH-[:270,,1]-[:330](-[:30,,,1]OCH_3)=_[:270]-[:210](-[:270]COH)=_[:150]-[:90](=_[:30])}
\end{center}


Na vanilina estão presentes as funções orgânicas.

\begin{choice}
\choice aldeído, éter e fenol.
\choice álcool, aldeído e éter.
\choice álcool, cetona e fenol.
\choice aldeído, cetona e fenol.
\choice ácido carboxílico, aldeído e éter.
\end{choice}
\end{exercise}
\begin{solution}
LETRA A
\end{solution}




\begin{exercise}[points=1.0]
Uma forma de organização de um sistema biológico é a presença de sinais diversos utilizados pelos indivíduos para se comunicarem. No caso das abelhas da espécie \emph{Apis mellifera}, os sinais utilizados podem ser feromônios. Para saírem e voltarem de suas colmeias, usam um feromônio que indica a trilha percorrida por elas (Composto A). Quando pressentem o perigo, expelem um feromônio de alarme (Composto B), que serve de sinal para um combate coletivo. O que diferencia cada um desses sinais utilizados pelas abelhas são as estruturas e funções orgânicas dos feromônios.

\begin{tabular}{cc}
\chemfig{-[:270](=[:330]-[:30]CH_2OH)-[:210]-[:270]-[:330]=[:270](-[:210])-[:330]} & \chemfig{CH_3COO|{(CH_2)}CH(-[:-30]CH_3)-[:30]CH_3}\\
{\bfseries Composto A} & {\bfseries Composto B}
\end{tabular}

As funções orgânicas que caracterizam os feromônios de trilha e de alarme são, respectivamente,

\begin{choice}(2)
\choice álcool e éster.
\choice aldeído e cetona.
\choice éter e hidrocarboneto.
\choice enol e ácido carboxílico.
\choice ácido carboxílico e amida.
\end{choice}
\end{exercise}
\begin{solution}
Letra A álcool e éster
\end{solution}





\begin{exercise}[points=1.0]
A aspirina é um composto que possui propriedades antitérmica e analgésica, e tem como princípio ativo a estrutura representada na figura a seguir. Quais grupos funcionais orgânicos encontram-se neste composto.
\begin{center}
\chemfig{CH_3-[:210,,1](=[:270]O)-[:150]O-[:210]=_[:270]-[:210]=_[:150]-[:90]=_[:30](-[:330])-[:90](=[:150]O)-[:30,,,1]OH}
\end{center}
\begin{choice}
\choice cetona, aldeído e aromático.
\choice ácido carboxílico, éter e alcano.
\choice cetona, amida e alceno.
\choice ácido carboxílico, éster e aromático.
\choice fenol, éster e aromático.
\end{choice}
\end{exercise}
\begin{solution}
LETRA D
\end{solution}





\begin{exercise}[points=1.0]
A testosterona (I) é um hormônio sexual masculino responsável, entre outras coisas, pelas alterações sofridas pelos rapazes na puberdade. Já a progesterona (II) é um hormônio sexual feminino, indispensável à gravidez e estão representadas pelas respectivas estruturas abaixo. Assinale a alternativa que indica corretamente às funções orgânicas presentes nas duas substâncias
\begin{center}
{\bfseries I Testosterona}
\chemfig[cram width=4pt]{OH>[:252,,1]-[:306]-[:234]>[:162]-[:210]-[:270]-[:210]-[:150]=_[:210]-[:150](=[:210]O)-[:90]-[:30]-[:330](-[:270])(<[:90,,,1]CH_3)-[:30](-[:330])-[:90]-[:30]-[:330](-[:270])(-[:18])<[:84,,,1]CH_3}

{\bfseries II Progesterona}

 \chemfig[cram width=4pt]{H_3C-[:282,,2](=[:342]O)>[:222]-[:276]-[:204]>[:132]-[:60](-[:348])(<[:54,,,1]CH_3)-[:120]-[:180]>[:240]-[:300](-)<:[:240]-[:180]-[:120]=_[:180]-[:120](=[:180]O)-[:60]--[:300](-)(-[:240])<[:60,,,1]CH_3}
 \end{center}
\begin{choice}
\choice I – fenol e cetona; II - cetona.
\choice I – ácido e cetona; II - aldeído.
\choice I – álcool e cetona; II - aldeído.
\choice I – fenol e cetona; II - ácido.
\choice I – álcool e cetona; II - cetona
\end{choice}
\end{exercise}
\begin{solution}
LETRA E 
\end{solution}





\begin{exercise}[points=1.0]
A curcumina, substância encontrada no pó amarelo-alaranjado extraído da raiz da curcuma ou açafrão-daíndia (\emph{Curcuma longa}), aparentemente, pode ajudar a combater vários tipos de câncer, o mal de Parkinson e o de Alzheimer e até mesmo retardar o envelhecimento. Usada há quatro milênios por algumas culturas orientais, apenas nos últimos anos passou a ser investigada pela ciência ocidental.

\begin{center}
\begin{center}
\setchemfig{atom style={scale=0.8}}
\chemfig{
          HO% 21
     -[:330]% 18
    =^[:270]% 17
               (
         -[:210]O% 22
         -[:270]% 23
               )
     -[:330]% 16
     =^[:30]% 15
               (
          -[:90]% 20
        =^[:150]% 19
         -[:210]% -> 18
               )
     -[:330]% 14
      =[:30]% 13
     -[:330]% 12
      -[:30]% 11
               (
          -[:90]OH% 24
               )
     =[:330]% 10
      -[:30]% 9
               (
          =[:90]O% 25
               )
     -[:330]% 8
      =[:30]% 7
     -[:330]% 6
    =^[:270]% 5
     -[:330]% 4
     =^[:30]% 3
               (
         -[:330]OH% 26
               )
      -[:90]% 2
               (
        =^[:150]% 1
         -[:210]% -> 6
               )
      -[:30]O% 27
      -[:90]% 28
}
\end{center}
\end{center}

Na estrutura da curcumina, identificam-se grupos característicos das funções

\begin{choice}(2)
\choice éter e álcool.
\choice éter e fenol.
\choice éster e fenol.
\choice aldeído e enol.
\choice aldeído e éster.
\end{choice}
\end{exercise}
\begin{solution}
B
\end{solution}



\begin{exercise}[points=1.0]
Um trabalho publicado na Nature Medicine, em 2016, mostrou que Withaferin A, um componente do extrato da planta \emph{Withania somnifera} (cereja de inverno), reduziu o peso, entre 20 a 25\%, em ratos obesos alimentados em dieta de alto teor de gorduras



\begin{center}
\chemfig[cram width=3.5pt]{
    HO% 7
     >[:60]% 4
          -% 3
     -[:60]% 2
              (
        -[:120]% 1
                  (
             =[:60]O% 27
                  )
        -[:180]% 6
        -[:240]% 5
        -[:300]% -> 4
              )
              (
         <[:80]% 26 metil 
              )
          -% 11
              (
        <:[:100]H% 22
              )
    -[:300]% 10
              (
        -[:240]% 9
        -[:180]% 8
                  (
            -[:180]O% 24
             >[:60]% -> 3
                  )
                  (
            <:[:270]H% 25
                  )
        -[:120]% -> 3
              )
              (
        <[:280]H% 23
              )
          -% 15
              (
        <:[:266]OH% 20
              )
     -[:60]% 14
              (
        -[:120]% 13
        -[:180]% 12
        -[:240]% -> 11
              )
              (
         <[:54]% 21
              )
    -[:348]% 18
              (
         -[:42]% 19
              )
    <[:276]% 17
    -[:204]% 16
              (
        -[:132]% -> 15
              )
}
\end{center}

Entre as funções orgânicas presentes na Withaferin A, estão

\begin{choice}
\choice ácido carboxílico e cetona.
\choice aldeído e éter.
\choice cetona e hidroxila alcoólica.
\choice cetona e éster.
\choice éster e hidroxila fenólica. 
\end{choice}
\end{exercise}







\begin{exercise}[points=1.0]
A questão refere-se ao geraniol, um óleo essencial de aroma floral, como o de rosas.

\begin{center}
\chemfig{
HO% 4
    -[:330,,2]% 3
        -[:30]% 2
       =[:330]% 1
                 (
           -[:270]% 11
                 )
        -[:30]% 5
       -[:330]% 6
        -[:30]% 7
       =[:330]% 8
                 (
           -[:270]% 9
                 )
        -[:30]% 10
}
\end{center}

O geraniol é um


\begin{choice}(2)
\choice álcool.
\choice enol.
\choice fenol.
\choice alcino.
\choice aldeído. 
\end{choice}
\end{exercise}






\begin{exercise}[points=1.0]
A bile é produzida pelo fígado, armazenada na vesícula biliar e tem papel fundamental na digestão de lipídeos. Os sais biliares são esteroides sintetizados no fígado a partir do colesterol, e sua rota de síntese envolve várias etapas. Partindo do ácido cólico representado na figura, ocorre a formação dos ácidos glicocólico e taurocólico; o prefixo glico- significa a presença de um resíduo do aminoácido glicina e o prefixo tauro-, do aminoácido taurina.

\begin{center}
\setchemfig{atom style={rotate=30}}
\chemfig[cram width=3.7pt]{
      H% 1
    >:[:300]% 2
     -[:180]% 3
     -[:240]% 4
               (
     <[:180,,,2]HO% 5
               )
     -[:300]% 6
           -% 7
      -[:60]% 8
               (
         -[:120]% -> 2
               )
               (
        <:[:280]% 9
               )
           -% 10
               (
         <[:260]H% 11
               )
     -[:300]% 12
           -% 13
               (
     <[:300,,,1]OH% 14
               )
      -[:60]% 15
               (
        <:[:306]% 16
               )
      -[:12]% 17
               (
         <[:286]H% 18
               )
               (
               -% 28
                   (
              -[:60]% 30
                   -% 31
              -[:60]% 32
                       (
                       =O% 34
                       )
         -[:120,,,2]HO% 33
                   )
         <[:300]% 29
               )
      -[:84]% 19
     -[:156]% 20
     -[:228]% 21
               (
         -[:300]% -> 15
               )
               (
          <[:84]H% 22
               )
     -[:180]% 23
               (
         -[:240]% -> 10
               )
               (
         <:[:70]H% 24
               )
     -[:120]% 25
               (
      <[:60,,,1]OH% 26
               )
     -[:180]% 27
               (
         -[:240]% -> 2
               )
}
\end{center}

as funções orgânicas presentes na estrutura são:


\begin{choice}(2)
\choice Enol e álcool.
\choice Fenol e Enol.
\choice Fenol e Éter.
\choice Ácido e álcool.
\choice Aldeído e éter. 
\end{choice}
\end{exercise}
\begin{solution}
D
\end{solution}



\begin{exercise}[points=1.0]
Uma das formas de se obter tinta para pintura corporal utilizada por indígenas brasileiros é por
meio do fruto verde do jenipapo. A substância responsável pela cor azul intensa dessa tinta é a
genipina, cuja estrutura está representada a seguir.

\begin{center}
\begin{tikzpicture}
\node at (0,0) {\chemfig[cram width=4pt]{
          OCH_3% 8
      -[:90]% 7
               (
         =[:150]O% 9
               )
      -[:30]% 4
    >:[:330]% 3
      -[:30]% 2
               (
         <:[:90]% 1
         -[:150]O% 6
         -[:210]% 5
        =^[:270]% -> 4
               )
     -[:318]% 12
               (
          -[:12]% 13
     -[:312,,,1]OH% 14
               )
    =_[:246]% 11
     -[:174]% 10
               (
         -[:102]% -> 3
               )
}};
\node at (2.4, -0.8) [draw,dashed,inner sep=0pt,circle,yscale=1.8cm,xscale=2.0cm]{};
\end{tikzpicture}
\end{center}

A estrutura assinalada mostra que a genipina possui, entre outras, a função orgânica


\begin{choice}(2)
\choice aldeído.
\choice álcool.
\choice cetona.
\choice ácido carboxílico.
\choice éter.
\end{choice}
\end{exercise}



\begin{exercise}[points=1.0]
O sesterpenóide manoalido, isolado de uma esponja do Pacífico (\emph{Luffariella variablis}), é um inibidor irreversível de fosfolipase A2 (PLA2). Dessa forma, é um alvo terapêutico para ser usado no tratamento de doenças inflamatórias. Na representação de uma de suas formas tautoméricas, a seguir, podemos encontrar respectivamente as seguintes funções orgânicas


\begin{center}
\small
\chemfig{-[:290](-[:70])-[:330](=_[:270](-[:330])-[:210]-[:150]-[:90]-[:30])-[:30]-[:330]-[:30](-[:90])=[:330]-[:30]-[:330]-[:30]-[:330]-[:30]-[:90](-[:150]O-[:210](<[:150,,,2]HO)-[:270])<[:30](-[:336]=[:270]O)-[:84]-[:12](=[:66]O)-[:300,,,1]OH}
\end{center}

\begin{choice}
\choice ácido carboxílico, fenol, éster, álcool.
\choice ácido carboxílico, éster, amina, álcool.
\choice álcool, ácido carboxílico, éter, aldeído.
\choice ácido carboxílico, éter, fenol, álcool.
\choice álcool, fenol, éster, éter.
\end{choice}
\end{exercise}
\begin{solution}
LETRA C
\end{solution}






\begin{exercise}[points=1.0]
A estrutura da fenolftaleína contém os grupos 
\begin{center}
\chemfig{
              OH% 14
    -[:240,,1]% 11
      =_[:300]% 10
       -[:240]% 9
      =_[:180]% 8
                 (
           -[:120]% 13
           =_[:60]% 12
                 -% -> 11
                 )
       -[:240]C% 7
                 (
           -[:228]% 6
          =_[:300]% 5
                     (
                -[:12]% 23
                         (
                   =[:318]O% 24
                         )
                -[:84]O% 22
               -[:156]\phantom{C}% -> 7
                     )
           -[:240]% 4
          =_[:180]% 3
           -[:120]% 2
           =_[:60]% 1
                 -% -> 6
                 )
       -[:144]% 15
       =^[:84]% 16
       -[:144]% 17
      =^[:204]% 18
                 (
       -[:144,,,2]HO% 21
                 )
       -[:264]% 19
      =^[:324]% 20
                 (
            -[:24]% -> 15
                 )
}
\end{center}

os seguintes grupos funcionais


\begin{choice}(2)
\choice ácido carboxílico.
\choice aldeído.
\choice álcool.
\choice éster.
\choice éter.
\end{choice}
\end{exercise}





\begin{exercise}[points=1.0]
O aroma natural da baunilha, encontrado em doces e sorvetes, deve-se ao composto chamado vanilina, cuja fórmula estrutural está reproduzida ao lado. Em relação à molécula da vanilina, é correto afirmar que as funções químicas encontradas são:


\chemfig{
   O% 8
     =[:90]% 7
              (
    -[:150]H% 9
              )
     -[:30]% 6
   =^[:330]% 5
     -[:30]% 4
              (
        -[:330]O% 10
        -[:270]CH_3% 11
              )
    =^[:90]% 3
              (
     -[:30,,,1]OH% 12
              )
    -[:150]% 2
   =^[:210]% 1
              (
        -[:270]% -> 6
              )
}

\begin{choice}
\choice álcool, éter e éster.
\choice álcool, ácido e fenol.
\choice aldeído, álcool e éter.
\choice aldeído, éster e fenol.
\choice aldeído, éter e fenol.
\end{choice}
\end{exercise}
\begin{solution}
E
\end{solution}


\begin{exercise}[points=1]
A estrutura acima representa a alizarina, um corante amarelo conhecido desde a antiguidade.

\begin{center}
\chemfig{
   O =[:300]% 10
           -% 9
    =^[:300]% 8
               (
               -% 15
                   (
         -[:300,,,1]OH% 16
                   )
         =^[:60]% 14
                   (
             -[,,,1]OH% 17
                   )
         -[:120]% 13
        =^[:180]% 12
         -[:240]% -> 9
               )
     -[:240]% 7
               (
         =[:300]O% 18
               )
     -[:180]% 6
    =^[:120]% 5
               (
          -[:60]% -> 10
               )
     -[:180]% 4
    =^[:240]% 3
     -[:300]% 2
          =^% 1
               (
          -[:60]% -> 6
               )
}
\end{center}

Com base nessa informação e nos conhecimentos sobre as cadeias e funções orgânicas, pode-se afirmar que esse corante:
\begin{choice}
\choice possui grupos funcionais cetona e fenol.
\choice é um álcool secundário.
\choice tem cadeia alicíclica insaturada.
\choice apresenta heteroátomo na cadeia.
\choice possui núcleos isolados.
\end{choice}
\end{exercise}
\begin{solution}
A
\end{solution}


\begin{exercise}[points=1]
O bactericida FOMECIN A, cuja fórmula estrutural é:

\begin{center}
\chemfig{
       HO% 8
    -[:300,,2]% 7
             -% 6
      =^[:300]% 5
                 (
           -[:240]% 9
           =[:180]O% 10
                 )
             -% 4
                 (
       -[:300,,,1]OH% 11
                 )
       =^[:60]% 3
                 (
           -[,,,1]OH% 12
                 )
       -[:120]% 2
                 (
        -[:60,,,1]OH% 13
                 )
      =^[:180]% 1
                 (
           -[:240]% -> 6
                 )
}
\end{center}


O mesmo apresenta as funções de:
\begin{choice}
\choice ácido carboxílico e fenol.
\choice álcool, fenol e éter.
\choice álcool, fenol e aldeído.
\choice éter, álcool e aldeído.
\choice cetona, fenol e hidrocarboneto.
\end{choice}
\begin{solution}
C
\end{solution}
\end{exercise}




\begin{exercise}[points=1.0]
Compostos mais complexos que contêm grupos funcionais fenólicos são comumente encontrados na natureza, especialmente como produtos naturais vegetais. Por exemplo, alguns dos principais metabólitos encontrados no chá verde são os compostos polifenólicos de catequina.

\chemfig{
           HO% 19
    -[:300,,2]% 17
       -[:240]% 16
                 (
       -[:180,,,2]HO% 20
                 )
      =^[:300]% 15
                 (
       -[:240,,,2]HO% 21
                 )
             -% 14
       =^[:60]% 13
                 (
           -[:120]% 18
          =^[:180]% -> 17
                 )
             -% 12
                 (
            =[:60]O% 22
                 )
       -[:300]O% 11
             -% 9
       -[:300]% 8
                 (
          <:[:240]% 23
          =_[:300]% 24
           -[:240]% 25
                     (
           -[:300,,,1]OH% 31
                     )
          =_[:180]% 26
                     (
           -[:240,,,2]HO% 30
                     )
           -[:120]% 27
                     (
           -[:180,,,2]HO% 29
                     )
           =_[:60]% 28
                 -% -> 23
                 )
             -O% 7
        -[:60]% 6
      =_[:120]% 5
                 (
           -[:180]% 10
           -[:240]% -> 9
                 )
        -[:60]% 4
                 (
       -[:120,,,2]HO% 32
                 )
            =_% 3
       -[:300]% 2
                 (
           -[,,,1]OH% 33
                 )
      =_[:240]% 1
                 (
           -[:180]% -> 6
                 )
}

\begin{choice}(2)
\choice Álcool, Fenol e  Cetona
\choice Fenol, Epoxi e Ester
\choice Aldeído, Éter e Enol
\choice Cetona, Enol e Aldeído
\choice Eter, Álcool e Enol 
\end{choice}
\end{exercise}
\begin{solution}
B
\end{solution}



\begin{exercise}[points=1.0]
Qual é a fórmula geral de um álcool?
\begin{choice}
\choice R-COOH
\choice R-OH
\choice R-CO-R'
\choice R-CHO
\choice R-O-R
\end{choice}
\end{exercise}
\begin{solution}
B
\end{solution}







\begin{exercise}[points=1.0]
O tetraidrocanabinol (THC), um dos principais componentes da
\emph{Cannabis}, é o responsável pelas propriedades medicinais.


\begin{tikzpicture}
\node[draw=none] at (0,0) { 
 \chemfig[cram width=4pt]{
         % 8
     -[:140]% 7
               (
         -[:260]% 9
               )
      -[:60]% 4
               (
        <:[:300]H% 23
               )
     -[:120]% 3
               (
          -[:60]% 2
              =_% 1
                   (
              -[:60]% 25
                   )
         -[:300]% 6
         -[:240]% 5
         -[:180]% -> 4
               )
               (
         <[:120]H% 24
               )
     -[:180]% 12
    =_[:240]% 11
               (
         -[:300]O% 10
               -% -> 7
               )
     -[:180]% 16
    =_[:120]% 15
               (
          -[:60]% 14
              =_% 13
                   (
          -[:60,,,1]OH% 22
                   )
         -[:300]% -> 12
               )
     -[:180]% 17
     -[:120]% 18
     -[:180]% 19
     -[:120]% 20
     -[:180]% 21
}
};
\node[draw=none] at (1,-2) {\bfseries THC};
\end{tikzpicture}  

Quais as funções orgânicas presentes na estrutura.

\begin{choice}(2)
\choice éster e fenol.
\choice éter e fenol.
\choice éster e álcool.
\choice fenol e álcool.
\choice éter e álcool.
\end{choice}
\end{exercise}
\begin{solution}
B
\end{solution}



\begin{exercise}[points=1.0]
A fórmula representa a estrutura do geranial, também conhecido como citral A, um dos compostos responsáveis pelo aroma do limão.

\begin{center}
\chemfig{
        O% 4
              =[:330]% 3
               -[:30]% 2
              =[:330]% 1
                        (
                  -[:270]CH_3% 11
                        )
               -[:30]% 5
              -[:330]% 6
               -[:30]% 7
              =[:330]% 8
                        (
                  -[:270]CH_3% 9
                        )
               -[:30]CH_3% 10
}
\end{center}

O geranial é um composto pertencente à função orgânica

\begin{choice}(2)
\choice cetona.
\choice éter.
\choice éster.
\choice ácido carboxílico
\choice aldeído
\end{choice}
\end{exercise}
\begin{solution}
E
\end{solution}



\begin{exercise}[points=1.0]
A cerveja de raiz não tem o mesmo sabor desde que o uso do óleo de sassafrás como aditivo alimentar foi proibido porque o óleo de sassafrás contém 80\% de safrol, que comprovadamente causa câncer em ratos e camundongos. Identifique os grupos funcionais na estrutura do safrol.



\chemfig{=[:330]-[:30]-[:330]-[:30]-[:90](-[:150]-[:210]-[:270])-[:18]O%
-[:306]-[:234]O(-[:162])}


\begin{choice}(2)
\choice cetona.
\choice éter.
\choice éster.
\choice ácido carboxílico
\choice aldeído
\end{choice}
\end{exercise}
\begin{solution}
B
\end{solution}




\begin{exercise}[points=1.0]
A descoberta da penicilina em 1928 marcou o início do que foi chamado de “era de ouro da quimioterapia”, na qual infecções bacterianas que antes ameaçavam a vida foram transformadas em pouco mais do que uma fonte de desconforto. Para aqueles que são alérgicos à penicilina, estão disponíveis uma variedade de antibióticos, incluindo a tetraciclina.

\chemfig[atom style={scale=0.7},cram width=4pt]{O=[:270,1.613]-[:210,1.613](<:[:90,1.613]O-[:130]H)-[:270,1.613](%
-[:210.9,1.68](-[:150.4,1.68](-[:90,1.68](-[:149.8,1.613](-[:209.8,1.613]N(%
-[:269.8]H)-[:149.8]H)=[:89.8,1.613]O)=_[:29.6,1.68](-[:329.1,1.68])%
-[:109.3,1.613]O-[:169.3]H)=[:210.2,1.613]O)(-[:330.8]H)<:[:270.7,1.613]N(%
-[:330.7,1.613](-[:330.7]H)(-[:60.7]H)-[:240.7]H)-[:210.7,1.613](-[:210.7]H%
)(-[:300.7]H)-[:120.7]H)(<:[:270.4,1.371]H)-[:330,1.613](-[:230]H)(-[:310]H%
)-[:30,1.613](<:[:269.6,1.371]H)-[:329.1,1.68](<[:299.3,1.613]O-[:239.3]H)(%
-[:239.3,1.613](-[:239.3]H)(-[:329.3]H)-[:149.3]H)-[:29.6,1.68]=_[:90,1.68]%
(-[:150.4,1.68](=^[:210.9,1.68](-[:150,1.613])-[:270,1.613])-[:90.7,1.613]O%
-[:30.7]H)-[:30.9,1.75](-[:90.7,1.613]O-[:30.7]H)=_[:330.4,1.75](-[:30.2]H)%
-[:270,1.75](-[:329.8]H)=_[:209.6,1.75](-[:149.1,1.75])-[:269.3]H}

Identifique os numerosos grupos funcionais oxigenados na molécula de tetraciclina.




\begin{choice}(1)
\choice cetona, enol, éter.
\choice éter, éster , cetona
\choice cetona, álcool e enol.
\choice aldeído, cetona, fenol
\choice aldeído, éter, éster
\end{choice}
\end{exercise}
\begin{solution}
A
\end{solution}



\begin{exercise}[points=1.0]
A descoberta da penicilina em 1928 marcou o início do que foi chamado de “era de ouro da quimioterapia”, na qual infecções bacterianas que antes ameaçavam a vida foram transformadas em pouco mais do que uma fonte de desconforto. Para aqueles que são alérgicos à penicilina, estão disponíveis uma variedade de antibióticos, incluindo a tetraciclina.

\chemfig[atom style={scale=0.7},cram width=4pt]{O=[:270,1.613]-[:210,1.613](<:[:90,1.613]O-[:130]H)-[:270,1.613](%
-[:210.9,1.68](-[:150.4,1.68](-[:90,1.68](-[:149.8,1.613](-[:209.8,1.613]N(%
-[:269.8]H)-[:149.8]H)=[:89.8,1.613]O)=_[:29.6,1.68](-[:329.1,1.68])%
-[:109.3,1.613]O-[:169.3]H)=[:210.2,1.613]O)(-[:330.8]H)<:[:270.7,1.613]N(%
-[:330.7,1.613](-[:330.7]H)(-[:60.7]H)-[:240.7]H)-[:210.7,1.613](-[:210.7]H%
)(-[:300.7]H)-[:120.7]H)(<:[:270.4,1.371]H)-[:330,1.613](-[:230]H)(-[:310]H%
)-[:30,1.613](<:[:269.6,1.371]H)-[:329.1,1.68](<[:299.3,1.613]O-[:239.3]H)(%
-[:239.3,1.613](-[:239.3]H)(-[:329.3]H)-[:149.3]H)-[:29.6,1.68]=_[:90,1.68]%
(-[:150.4,1.68](=^[:210.9,1.68](-[:150,1.613])-[:270,1.613])-[:90.7,1.613]O%
-[:30.7]H)-[:30.9,1.75](-[:90.7,1.613]O-[:30.7]H)=_[:330.4,1.75](-[:30.2]H)%
-[:270,1.75](-[:329.8]H)=_[:209.6,1.75](-[:149.1,1.75])-[:269.3]H}

Identifique os numerosos grupos funcionais oxigenados na molécula de tetraciclina.




\begin{choice}(2)
\choice cetona, enol, éter.
\choice éter, éster , cetona
\choice cetona, álcool e enol.
\choice aldeído, cetona, fenol
\choice aldeído, éter, éster
\end{choice}
\end{exercise}
\begin{solution}
A
\end{solution}


\collectexercisesstop{FuncoesOxigenadasIII}
\collectexercises{FuncoesOxigenadasIV}

\begin{exercise}[points=1.0]
A estrutura da aspirina abaixo contém os seguintes grupos funcionais

\begin{center}
\chemfig{-[:60](=[:120]O)-O-[:300]=^[:240]-[:300]=^-[:60]=^[:120](-[:180])%
-[:60](=[:120]O)-[,,,1]OH}
\end{center}

Quais os grupos funcionais oxigenados presente.


\begin{choice}(2)
\choice cetona  e etér.
\choice éter e enol.
\choice éster e álcool.
\choice ácido carboxílico e éster.
\choice aldeído e éster
\end{choice}
\end{exercise}
\begin{solution}
D
\end{solution}



\begin{exercise}[points=1.0]
O cetoprofeno é usado para aliviar a dor, sensibilidade, inchaço e rigidez causada pela osteoartrite (artrite causada por uma ruptura do revestimento das articulações) e artrite reumatóide (artrite causada pelo inchaço do revestimento das articulações).

\begin{center}
\chemfig{-[:270](-[:330](-[:30,,,1]OH)=[:270]O)-[:210]=^[:150]-[:210](%
=^[:270]-[:330]=^[:30]-[:90])-[:150](=[:90]O)-[:210]=^[:150]-[:210]=^[:270]%
-[:330]=^[:30](-[:90])}
\end{center}

Quais os grupos funcionais oxigenados presente.


\begin{choice}(2)
\choice cetona  e enol.
\choice éter e ácido carboxílico 
\choice cetona e ácido carboxílico.
\choice ácido carboxílico e éster.
\choice aldeído e álcool.
\end{choice}
\end{exercise}
\begin{solution}
A
\end{solution}


\begin{exercise}[points=1.0]
Khellin tem sido usado como uma medicina popular herbal, com uso no Mediterrâneo que remonta ao Egito Antigo, para tratar uma variedade de doenças, incluindo: cólicas renais, pedras nos rins, doença coronária, asma brônquica, vitiligo e psoríase.

\begin{center}
\chemfig{-[:330]=_[:30]-[:330](=[:30]O)-[:270]=_[:330](-[:30]O-[:330])%
-[:270]=^[:210](-[:150](-[:210]O-[:270])=_[:90](-[:150]O-[:90])-[:30])%
-[:282]O-[:354]=^[:66](-[:138])}
\end{center}

a estrutura acima tem quais grupos funcionais

\begin{choice}(2)
\choice cetona  e enol.
\choice éter e álcool.
\choice cetona e éter.
\choice ácido carboxílico e éster.
\choice éster e álcool.
\end{choice}
\end{exercise}
\begin{solution}
C
\end{solution}

\begin{exercise}[points=1.0]
\emph{Streptomycetaceae} é uma família de Actinomycetota, que compõe a ordem monotípica \emph{Streptomycetales}. Inclui o importante gênero \emph{Streptomyces}. Esta foi a fonte original de muitos antibióticos, ou seja, a estreptomicina, o primeiro antibiótico contra a tuberculose.
Veja a estrutura abaixo
\begin{center}
\chemfig[atom style={scale=.9}]{-[:204]-[:264](-[:204]-[:150](-[:96])-[:222]-[:294](-[:282]-[:342]%
)(-[:198]-[:138]-[:198]-[:258](-[:158]-[:218])(-[:238,,,2]HO)-[:318](%
-[:258])-[:18]O-[:78])-[:6]O-[:78])-[:324](=[:264]O)-[:24](-[:84])-[:324](%
-[:264,,,1]OH)-[:24](-[:84])-[:324]-[:24]-[:324]=^[:264](-[:204](%
-[:264,,,1]OH)=[:144]O)-[:324](-[:264,,,1]OH)=^[:24](-[:324])-[:84]=^[:144]%
(-[:204])}
\end{center}

a estrutura acima tem quais grupos funcionais

\begin{choice}(1)
\choice Aldeído, cetona, fenol, éter  e enol.
\choice Aldeído, éster, fenol, éter e álcool.
\choice Cetona, álcool, fenol, epóxi  e éter.
\choice ácido carboxílico, enol, álcool, cetona  e éster.
\choice Ácido carboxílico, fenol, cetona, éter e álcool.
\end{choice}
\end{exercise}
\begin{solution}
E
\end{solution}



\begin{exercise}[points=1.0]
A GESTRINONA é um 19-nor-esteróide, anti-estrogênio e antiprogesterona empregado no tratamento da endometriose e da miomatose. Apresenta características anovulatórias, efeito anabolizante e hemostático. Possui ainda indicação para tratamento da Tensão Pré-Menstrual (TPM), hipertrofia uterina, baixa de libido, perda de massa muscular e massa óssea, revertendo, quando associado a um estrogênio, a osteopenia após alguns meses de tratamento

\begin{center}
\begin{tikzpicture}
\node[draw=none] at (0,0) {
\chemfig[cram width=4pt]{-[:114]>[:54]-[:300]->[:60]-[:120](-[:180](<[:132]=^[:204]-[:276](%
-[:180]~[:180])(<[:254,,,2]HO)-[:348])-[:240])<:[:60]--[:300]=_-[:300](=O)%
-[:240]-[:180]>:[:120](-[:180])(-[:60])}
};
\node at (3.4, -0.2) [draw,dashed,inner sep=0pt,circle,yscale=1.8cm,xscale=2.0cm]{};
\node at (-3.2, 0) [draw,dashed,inner sep=0pt,circle,yscale=1.3cm,xscale=1.7cm]{};
\node at (-2.2, -0.9) [draw,dashed,inner sep=0pt,circle,yscale=1.3cm,xscale=1.7cm]{};
\end{tikzpicture}
\end{center}

a estrutura acima tem quais grupos funcionais destacados

\begin{choice}(2)
\choice Vinil, cetona  e enol.
\choice Etinil, cetona e álcool
\choice Fenol, cetona e éter.
\choice Alcino, cetona e éster.
\choice Alcinio, éster e álcool.
\end{choice}
\end{exercise}
\begin{solution}
B
\end{solution}



\begin{exercise}[points=1.0]
A eritromicina foi isolada pela primeira vez em 1952 a partir da bactéria \emph{Saccharopolyspora erythraea}. A eritromicina é um antibiótico utilizado no tratamento de diversas infecções bacterianas. Isso inclui infecções do trato respiratório , infecções de pele, infecções por clamídia , doença inflamatória pélvica e sífilis. 

\begin{center}
\chemfig[cram width=4pt]{HO>:[:345,,2](-[:225])-[:330](<[:270]O>[:210]-[:150]O-[:210](%
<[:150])-[:270]-[:330](-[:30](-[:90])<:[:330,,,1]OH)<[:270]N(-[:330])%
-[:210])-[:30](<[:90])-[:330](<[:270]O>:[:330]-[:30]O-[:330](<[:30])-[:270]%
(-[:210](-[:150]-[:90])(-[:240])<:[:300]O-[:240])<:[:330,,,1]OH)-[:30](%
<:[:330])-[:90](=[:150]O)-[:30]O-[:90](<[:30]-[:330])-[:150](<:[:75,,,1]OH)%
(-[:125])-[:210](<:[:310,,,1]OH)-[:150](<:[:90])-[:210](=[:150]O)-[:270](%
<:[:270])-[:210](-[:270])}

\end{center}

Na estrutura da eritromicina contém os seguintes grupos funcionais oxigenados.

\begin{choice}
\choice Vinil, cetona  e enol.
\choice Cetona, álcool e éster
\choice Cetona, éter e ácido carboxílico.
\choice Enol, éster e aldeído.
\choice Éster, álcool e anidrido.
\end{choice}
\end{exercise}




\begin{exercise}[points=1.0]
As catequinas são compostos incolores, hidrossolúveis, que contribuem para o amargor e a adstringência do chá verde. As teaflavinas são compostos responsáveis por parte da cor (alaranjada) e sabor (adstringência) da infusão de chá preto.


\chemfig[cram width=4pt]{HO>[:300,,2]-[:240](<:[:180]=_[:240]-[:180]=_[:120](-[:180,,,2]HO)%
-[:60](-[:120,,,2]HO)=_-[:300])-[:300]O-=_[:60](-[:120]-[:180])-(%
-[:60,,,1]OH)=_[:300]-[:240](-[:300,,,1]OH)=_[:180](-[:120])}

\begin{choice}
\choice Cetona, álcool  e enol.
\choice Cetona, fenol e éster
\choice Fenol, éter e álcool
\choice Enol, éster e aldeído.
\choice Éster, álcool e enol.
\end{choice}
\end{exercise}


\collectexercisesstop{FuncoesOxigenadasIV}



\printrandomexercises[collection=FuncoesOxigenadasIII,exclude=one]{5}
\printrandomexercises[collection=FuncoesOxigenadasIV,exclude=two]{5}




\printcollection{FuncoesOxigenadas}
\end{document}
