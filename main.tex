% Created 2024-11-13 Wed 12:42
% Intended LaTeX compiler: lualatex
\documentclass[11pt]{scrartcl}


\KOMAoptions{
%headings=chapterprefix,
twocolumn=true,
%toc=indenttextentries,
%toc=flat,
twoside=true,
headinclude=true,
footinclude=true
%  captions=topbeside
}
%\usepackage[fontsize=12.3]{scrextend}
\usepackage{fontspec}
\usepackage[T1]{fontenc}
\usepackage{hyperref}
\usepackage[x11names,svgnames,table]{xcolor}
\defaultfontfeatures{Ligatures=TeX}
%%\setmainfont{Lato}
%%\setmainfont{Charis SIL}
\setmainfont{IBM Plex Serif}
\usepackage{typearea}
\usepackage{lscape}
\usepackage[a4paper]{geometry}
\geometry{a4paper,total={170mm,257mm},left=10mm,right=10mm, top=15mm, bottom=20mm}
\usepackage[english, portuguese, american]{babel}
\usepackage{amsmath,amsfonts,amsthm,bm}
\usepackage{graphicx}
\usepackage{float,wrapfig}
\usepackage{colortbl}
\usepackage{tabularx}
\usepackage{pst-labo}
\usepackage{setspace}
\usepackage{xfrac}
\usepackage{tikz}
\usepackage{pgfplots}
\pgfplotsset{compat=1.3}
%% Diagraman latex
\usepackage{endiagram}
\usepackage{smartdiagram}
\usepackage[tikz]{bclogo}
\usetikzlibrary{fit,patterns,shadows.blur,shapes,decorations.pathreplacing,decorations.markings,arrows.meta,arrows,positioning,shadows,trees}
\usetikzlibrary{decorations.pathmorphing} %% to chemfig config bond
\usepackage{upgreek}
\usepackage[modules={all}]{chemmacros}
%%\chemsetup{modules={reactions,spectroscopy,thermodynamics,redox,isotopes}}
%%\chemsetup{modules={all}}
\NewChemState\EPot{ symbol=E , subscript-pos=right , superscript=o, pre= , unit=\volt }
%\usepackage[version=4,arrows=pgf-filled]{mhchem}
\usepackage{chemfig,elements,cancel,siunitx}
\NewChemPhase\lqdd{\(\ell\)}
\NewChemPhase\gr{grafite}
\NewChemPhase\reac{reação}
\setchemfig{fixed length=false, atom sep=2.0em, arrow offset=6pt, scheme debug=false,angle increment=30}
\renewcommand{\CancelColor}{\color{red}}
\usepackage{circuitikz}
\usepackage{mol2chemfig}
\usepackage{subfig,caption}
\captionsetup{font=small, labelfont={bf,sf}}
\usepackage{wrapfig,qrcode}
\usepackage{array,longtable} % ajust colunm table
\newcolumntype{J}{>{\centering\arraybackslash}m{7.5cm}}
\newcolumntype{K}{>{\centering\arraybackslash}m{6.5cm}}
\newcolumntype{L}{>{\centering\arraybackslash}m{5cm}}
\newcolumntype{B}{>{\centering\arraybackslash}m{2.5cm}}
\newcolumntype{N}{>{\centering\arraybackslash}m{1.4cm}}
\usepackage[most]{tcolorbox}
\newcounter{mycounter}
%%% Colobor
%%% Example colorbox
\newtcolorbox{Box2}[2][]{
lower separated=false,
colback=white,
colframe=black,fonttitle=\bfseries,
colbacktitle=black,
coltitle=white,
enhanced, attach boxed title to top left={yshift=-0.1in,xshift=0.15in}, boxed title style={boxrule=0pt,colframe=white,}, title=#2,#1}
%%%%%%%% Cabecalho
\usepackage{framed,amsmath}
\newtcolorbox{mybox}[2][]{
enhanced,title=#2, fonttitle=\sffamily\small,
top=2pt,
bottom=1mm,
boxrule=0.4pt,
coltitle=black,
colback=white,
attach boxed title to top center={yshift=-\tcboxedtitleheight/2,
yshifttext=-\tcboxedtitleheight/2},
boxed title style={
colframe=white,
colback=white,
left=0.2pt,
right=0.2pt},
#1}
\usepackage{tabularray}
%%%%%%
\newtcolorbox{exercisebox}%
{enhanced,breakable,colback=white, colframe=green!15!white,colbacktitle=white!15!pink, coltitle=pink!50!black,left=0pt,right=0mm,top=3mm,bottom=3mm,pad at break=0pt,bottomrule at break=0pt,toprule at break=0pt,borderline={0mm}{0mm}{green!50!white,dashed}, attach boxed title to top center={yshift=-2mm},boxed title style={boxrule=0.4pt},title=Exercícios,}
\usepackage{eso-pic}
\usepackage{etoolbox}
\usepackage{enumitem}
\newcommand\circitem[1]{%
\tikz[baseline=(char.base)]{%https://tex.stackexchange.com/questions/204116/uniform-size-of-circles-around-enumitems
\node[circle,draw=gray, fill=gray!30,
minimum size=1.2em,inner sep=0] (char) {#1};}}
\newcommand\boxitem[1]{%
\tikz[baseline=(char.base)]{%https://tex.stackexchange.com/questions/204116/uniform-size-of-circles-around-enumitems
\node[fill=orange!30,
minimum size=1.2em,inner sep=0] (char) {#1};}}
%\usepackage{widetext}% needs packages "flushend" & "cuted" of "sttools" % bundle, which perhaps must separately be installed
\newcommand{\dd}[1]{\hspace{2pt}d#1}
\definecolor{color1}{RGB}{0,0,90} % Color of the article title and sections
\definecolor{color2}{RGB}{0,20,20} % Color of the boxes behind the abstract and
\definecolor{cinza}{HTML}{C0C0C0}
%%% Custom Exercios
\usepackage{bohr}
\usepackage{multicol}
\setlength{\columnsep}{1.5cm}
\setlength{\columnseprule}{0.2pt}
\usepackage[no-files]{xsim}
\usepackage{tasks}
\xsimsetup{
goal-print={\pgfmathprintnumber[fixed zerofill,precision=1]{#1}}
}
\newcommand*\circled[2]{\tikz[baseline=(char.base)]{
\node[shape=circle,fill,inner sep=2pt, text=white] (char) {#1};}}
%%%%%-Custom Xsim exercises %%%%%
\DeclareExerciseEnvironmentTemplate{custom}
{%\item[\GetExerciseProperty{counter}]
\Needspace*{0\baselineskip}
\noindent
\circled{\XSIMmixedcase{\GetExerciseProperty{counter}}}~~~%
\noindent
\IfInsideSolutionF{%
\GetExercisePropertyT{points}{ % notice the space
(%
\printgoal{\PropertyValue}
\IfExerciseGoalSingularTF{points}
{%\XSIMtranslate{point}
}
{% \XSIMtranslate{points}
}%
)%
}
}}
{\vspace{\baselineskip}}
%%%%%------- Custom  resposta -------%%%%%%%
\DeclareExerciseEnvironmentTemplate{space}
%{\textbf{\GetExerciseProperty{counter}} }
{\noindent\circled{\XSIMmixedcase{\GetExerciseProperty{counter}}}~~~}
% {\circled{\XSIMmixedcase{\GetExerciseProperty{counter}}}}~~~%
{\qquad}
\newcommand*\answer[1]{%
\XSIMexpandcode{%
\SetExerciseProperty{solution-body}
{\noexpand{\Alph{task}}}}%
#1%
}
%\sisetup{locale=DE}
\xsimsetup{
collect = true,
exercise/within = section,
exercise/template = custom,
exercise/the-counter =  \arabic{exercise},
solution/template= custom ,
%%solution-name = solution,  % used with headings=true
solution/print=false,
%print-collection/print=both,
%goal-print= {\pgfmathprintnumber[fixed zerofill,precision=1]\num{#1}}
}
\RenewDocumentCommand\printpoints{}{%
\TotalExerciseTypeGoal{exercise}{points}{}{}%
}
\NewTasksEnvironment[label = (\emph{\alph*}), label-width = 12pt]{choice}[\choice]
\newenvironment{questions}{\itemize}{\enditemize}
\everymath{\displaystyle}
\DeclareExerciseHeadingTemplate{solution}{%
\section*{Gabarito}%
}
%\usepackage{filecontents}
\newcommand{\lh}{\underline{\hspace{1cm}}}
%%\onehalfspacing
\def\professor{Fábio Lima}
\def\aluno{ }
\def\numerochamada{}
%%\def\disciplina{Química}
\def\disciplina{UC3}
%%\def\disciplina{R.A.}
\def\turma{1 Ano B}
\def\tipo{{\bfseries Avaliação Bimestral}}
%%\def\tipo{\bfseries Avaliação Mensal}
%%\def\tipo{\bfseries Exame Final}
\def\bimestre{4 Bimestre}
%%\def\escola{E.E. 26 de Agosto}
%\def\escola{E.E. José Mamede de Aquino}
\def\escola{E.E. Amélio de Carvalho Baís}
\def\dataprova{}
\DeclareExerciseCollection{MoleMassaMolar}
\DeclareExerciseCollection{Solucoes}
\author{fabio}
\date{\today}
\title{}
\hypersetup{
 pdfauthor={fabio},
 pdftitle={},
 pdfkeywords={},
 pdfsubject={},
 pdfcreator={Emacs 29.4 (Org mode 9.6.15)}, 
 pdflang={English}}
\begin{document}

\twocolumn[
%%\input{../Modelos/CabeOficial}
\input{../Modelos/cabenovo}
%\input{../Modelos/mamede}
%\input{../Modelos/26agosto}
%% \input{../Modelos/geral}
%Cada questão vale {\textbf 2,0}

%%\section*{Regime de Progressão Parcial}
%\section*{Atividade}
%\section*{Trabalho}
%%\section*{\disciplina}



%\input{../Modelos/gabarito}

%Total Prova: \printpoints
\smallbreak
\medbreak
\par\vspace{2ex}]%%%%\input{../Modelos/mamede}



\collectexercises{MoleMassaMolar}

\begin{exercise}[points=1]
(\textbf{UFG}) O corpo humano necessita diariamente de 12 mg de ferro. Uma colher de feijão contém cerca de \num{4.28e-5} mol de ferro. Quantas colheres de feijão, no mínimo, serão necessárias para que se atinja a dose diária de ferro no organismo? Dado Fe 56.

\begin{choice}(2)
\choice 1
\choice 3
\choice 5
\choice 7
\choice 9
\end{choice}
\end{exercise}


\begin{exercise}[points=1]
(\textbf{UNESP}) As hemácias apresentam grande quantidade de hemoglobina, pigmento vermelho que transporta oxigênio dos pulmões para os tecidos. A hemoglobina é constituída por uma parte não proteica, conhecida como grupo heme. Num laboratório de análises foi feita a separação de 22,0 mg de grupo heme de uma certa amostra de sangue, onde constatou-se a presença de 2,0 mg de ferro.
Se a molécula do grupo heme contiver apenas um átomo de ferro [Fe = 56 \unit{\mol\per\litre}], qual a sua massa molar em gramas por mol?


\begin{choice}(2)
\choice 154.
\choice 205.
\choice 308.
\choice 616.
\choice 1 232
\end{choice}
\end{exercise}
\begin{solution}
D
\end{solution}



\begin{exercise}[points=1]
(ITA) Considere as afirmações de I a V feitas em relação a um mol de \ch{H2O}

\begin{enumerate}[label=\Roman*]
\item Contém 2 átomos de hidrogênio.
\item Contém 1 átomo de oxigênio.
\item Contém 16 g de oxigênio.
\item Contém um total de 10 mols de prótons nos núcleos.
\item Pode ser obtido a partir de 0,5 mol de oxigênio molecular.
\end{enumerate}

Destas afirmações estão \textbf{CORRETAS:} Dados: H=1 e O=16:

\begin{choice}
\choice Apenas I e II.
\choice Apenas I, II e III.
\choice Apenas III e V.
\choice Apenas III, IV e V.
\choice Todas
\end{choice}
\end{exercise}
\begin{solution}
D
\end{solution}


\begin{exercise}[points=1]
(PUC-MG) O ácido tereftálico (\ch{C8H6O4}) é utilizado na fabricação de fibras sintéticas, do tipo poliéster. A massa de oxigênio existente em 0,5 mol de moléculas desse ácido é, em gramas, igual a: Dados: \ch{C8H6O4} 166 \unit{\gram\per\mol}.

\begin{choice}(2)
\choice 8,0
\choice 16,0
\choice 32,0
\choice 48,0
\choice 64,0
\end{choice}
\end{exercise}


\begin{exercise}[points=1]
O número de mols contido em 90 g de água é: (Dados: \ch{H2O}=18 \unit{\gram\per\mol})

\begin{choice}(2)
\choice 10 mols
\choice  5 mols
\choice 16 mols
\choice 7 mols
\choice 1 mol
\end{choice}
\end{exercise}

\collectexercisesstop{MoleMassaMolar}



\collectexercises{Solucoes}


\begin{exercise}[points=1.0]
Foram preparadas três soluções de sulfato de cobre, \ch{CuSO4}, um soluto de coloração azul, em frascos iguais de mesmo diâmetro interno. As quantidades de soluto e solução são mostradas na tabela a seguir. Dados: massa molar \ch{CuSO4} =\(1,6 \cdot 10^2\) g/mol

\begin{center}
\begin{tabular}{lll}
\hline
Soluções & Quantidade de CuSO\textsubscript{4} & Volume de CuSO\textsubscript{4}\\[0pt]
\hline
X & 4g & 500 mL\\[0pt]
Y & \(1\cdot 10^{-2}\) mol & 100 mL\\[0pt]
Z & \(3\cdot 10^{-2}\) mol & 300 mL\\[0pt]
\hline
\end{tabular}
\end{center}


Relacionando a cor da solução com suas concentrações e comparando-as entre si, observou-se que a
intensidade da cor azul da solução:

\begin{choice}
\choice X era maior do que a de Y e Z.
\choice Y era maior do que a de X e Z.
\choice Z era maior do que a de X e Y.
\choice X da solução Z era igual à de Y.
\choice Y era igual à de Z.
\end{choice}
\end{exercise}

\begin{exercise}[points=1.0]
Soro fisiológico contém 0,900 gramas de \ch{NaC$\ell$}, massa molar=58,5g/mol, em 100mL de solução
aquosa. A concentração do soro fisiológico, expressa em mol/L, é igual a


\begin{choice}(2)
\choice 0,009.
\choice 0,015.
\choice 0,100.
\choice 0,154.
\choice 0,900
\end{choice}
\end{exercise}


\begin{exercise}[points=1.0]
Dissolveram-se 2,48 g de tiossulfato de sódio pentaidratado (\ch{Na2S2O3.5 H2O}) em água para se
obter 100 cm\textsuperscript{3} de solução. A concentração molar dessa solução é. Dados: \ch{Na2S2O3. 5 H2O}=248 g/mol

\begin{choice}
\choice 0,157
\choice 0,100
\choice 0,000100
\choice 1,00
\choice 0,000157
\end{choice}
\end{exercise}


\begin{exercise}[points=1.0]
Uma solução 0,8 mol/L de NaOH possui 32g desta base dissolvida em água. O volume da solução assim
preparada é igual a: Dados:  NaOH = 40 u

\begin{choice}(2)
\choice 100 mL.
\choice 10 L.
\choice 10 mL.
\choice 1,0 L.
\choice 250 mL.
\end{choice}
\end{exercise}

\begin{exercise}[points=1.0]
No preparo de solução alvejante de tinturaria, 521,5g de hipoclorito de sódio são dissolvidos em água
suficiente para 10,0 litros de solução. A concentração, em mols/litro, da solução é: Dado: massa molar do \ch{NaC$\ell$O} = 74,5 g/mol

\begin{choice}
\choice 7,0 mol/L.
\choice 3,5 mol/L.
\choice 0,70 mol/L.
\choice 0,35 mol/L.
\choice  0,22 mol/L.
\end{choice}
\end{exercise}

\begin{exercise}[points=1.0]
A cafeína é a alcaloíde, identificado como 1,3,7-trimetilxantina (massa molar igual a 194 g/mol), cuja estrutura química contém uma unidade de purina, conforme representado. Esse alcaloide é encontrado em grande quantidade nas sementes de café e nas folhas de chá-verde. Uma xícara de café contém, em média, 80 mg de cafeína.

\begin{center}
\chemfig{CH_3-[:222,,1]N-[:276]=_[:204]N-[:132]=_[:60](-[:348]\phantom{N})-[:120](=[:60]O)-[:180]N(-[:120,,,2]H_3C)-[:240](=[:180]O)-[:300]N(-)-[:240,,,2]H_3C}

\scriptsize 
MARIA, C. A. B., MOREIRA, R. F. A Cafeína: revisão sobre métodos de análise, \textbf{Química Nova} 30 (1) Fev 2007
\end{center}


Considerando que a xícara descrita contém um volume de 200 mL de café, a concentração, em mol/L, de cafeína nessa xícara é mais próxima de:

\begin{choice}(2)
\choice 0,0004.
\choice 0,002.
\choice 0,4.
\choice 2.
\choice 4.
\end{choice}
\end{exercise}


\begin{exercise}[points=1.0]
Uma solução de hidróxido de magnésio, utilizada no combate à acidez estomacal, apresenta uma concentração igual a 2,9 g/L. A concentração, em \unit{\mole\per\liter}, dessa solução é igual a: Dado: \ch{Mg(OH)2} = 58.

\begin{choice}(2)
\choice 0,01 \unit{\mole\per\liter}
\choice 0,05 \unit{\mole\per\liter}
\choice 0,10 \unit{\mole\per\liter}
\choice 0,50 \unit{\mole\per\liter}
\choice 20 \unit{\mole\per\liter}
\end{choice}
\end{exercise}


\collectexercisesstop{Solucoes}



\printrandomexercises[collection=MoleMassaMolar]{5}
\printrandomexercises[collection=Solucoes,exclude=four]{5}



\printcollection{FuncoesOxigenadas}
\end{document}
