% Created 2025-03-19 qua 15:18
% Intended LaTeX compiler: lualatex
\documentclass[10pt]{scrartcl}


\KOMAoptions{
%headings=chapterprefix,
twocolumn=true,
%toc=indenttextentries,
%toc=flat,
twoside=true,
headinclude=true,
footinclude=true
%  captions=topbeside
}
%\usepackage[fontsize=12.3]{scrextend}
\usepackage{fontspec}
\usepackage[T1]{fontenc}
\usepackage[english, portuguese]{babel}
\usepackage{hyperref}
%\usepackage[x11names,svgnames,table]{xcolor}
\usepackage[usenames,dvipsnames]{xcolor}
\defaultfontfeatures{Ligatures=TeX}
%%\setmainfont{Lato}
%%\setmainfont{Charis SIL}
\setmainfont{IBM Plex Serif}
\usepackage{typearea}
\usepackage{lscape}
\usepackage[a4paper]{geometry}
\geometry{a4paper,total={170mm,257mm},left=10mm,right=10mm, top=15mm, bottom=20mm}
\usepackage[english,portuguese]{babel}
\usepackage{amsmath,amsfonts,amsthm,bm}
\usepackage{graphicx}
\usepackage{float,wrapfig}
\usepackage{colortbl}
\usepackage{tabularx}
\usepackage{pst-labo}
\usepackage{setspace}
\usepackage{xfrac}
\usepackage{tikz}
\usepackage{pgfplots}
\pgfplotsset{compat=1.3}
%% Diagraman latex
\usepackage{endiagram}
\usepackage{smartdiagram}
\usepackage[tikz]{bclogo}
\usetikzlibrary{fit,patterns,shadows.blur,shapes,decorations.pathreplacing,decorations.markings,arrows.meta,arrows,positioning,shadows,trees}
\usetikzlibrary{decorations.pathmorphing} %% to chemfig config bond
\usepackage{upgreek}
\usepackage[modules={all}]{chemmacros}
%%\chemsetup{modules={reactions,spectroscopy,thermodynamics,redox,isotopes}}
%%\chemsetup{modules={all}}
\NewChemState\EPot{ symbol=E , subscript-pos=right , superscript=o, pre= , unit=\volt }
%\usepackage[version=4,arrows=pgf-filled]{mhchem}
\usepackage{chemfig,elements,cancel,siunitx}
\NewChemPhase\lqdd{\(\ell\)}
\NewChemPhase\gr{grafite}
\NewChemPhase\reac{reação}
%\setchemfig{fixed length=false, atom sep=2.0em, arrow offset=6pt, scheme debug=false,angle increment=30}
\setchemfig{angle increment=30, atom sep=1.67em, double bond sep=0.67ex, bond style={line width=0.1em}, cram width=0.8ex, cram dash width=0.1em, cram dash sep=0.2em, arrow style={line width=0.067em},  arrow head=-{Triangle}, arrow label sep=1ex, cycle radius coeff=0.75, chemfig style={line width=0.1em}, }
\renewcommand{\CancelColor}{\color{red}}
\usepackage{circuitikz}
\usepackage{mol2chemfig}
\usepackage{subfig,caption}
\captionsetup{font=small, labelfont={bf,sf}}
\usepackage{wrapfig,qrcode}
\usepackage{array,longtable} % ajust colunm table
\newcolumntype{J}{>{\centering\arraybackslash}m{7.5cm}}
\newcolumntype{K}{>{\centering\arraybackslash}m{6.5cm}}
\newcolumntype{L}{>{\centering\arraybackslash}m{5cm}}
\newcolumntype{B}{>{\centering\arraybackslash}m{2.5cm}}
\newcolumntype{N}{>{\centering\arraybackslash}m{1.4cm}}
\usepackage[most]{tcolorbox}
\newcounter{mycounter}
%%% Colobor
%%% Example colorbox
\newtcolorbox{Box2}[2][]{
lower separated=false,
colback=white,
colframe=black,fonttitle=\bfseries,
colbacktitle=black,
coltitle=white,
enhanced, attach boxed title to top left={yshift=-0.1in,xshift=0.15in}, boxed title style={boxrule=0pt,colframe=white,}, title=#2,#1}
%%%%%%%% Cabecalho
\usepackage{framed,amsmath}
\newtcolorbox{mybox}[2][]{
enhanced,title=#2, fonttitle=\sffamily\small,
top=2pt,
bottom=1mm,
boxrule=0.4pt,
coltitle=black,
colback=white,
attach boxed title to top center={yshift=-\tcboxedtitleheight/2,
yshifttext=-\tcboxedtitleheight/2},
boxed title style={
colframe=white,
colback=white,
left=0.2pt,
right=0.2pt},
#1}
\usepackage{tabularray}
%%%%%%
\newtcolorbox{exercisebox}%
{enhanced,breakable,colback=white, colframe=green!15!white,colbacktitle=white!15!pink, coltitle=pink!50!black,left=0pt,right=0mm,top=3mm,bottom=3mm,pad at break=0pt,bottomrule at break=0pt,toprule at break=0pt,borderline={0mm}{0mm}{green!50!white,dashed}, attach boxed title to top center={yshift=-2mm},boxed title style={boxrule=0.4pt},title=Exercícios,}
\usepackage{eso-pic}
\usepackage{etoolbox}
\usepackage{enumitem}
\newcommand\circitem[1]{%
\tikz[baseline=(char.base)]{%https://tex.stackexchange.com/questions/204116/uniform-size-of-circles-around-enumitems
\node[circle,draw=gray, fill=gray!30,
minimum size=1.2em,inner sep=0] (char) {#1};}}
\newcommand\boxitem[1]{%
\tikz[baseline=(char.base)]{%https://tex.stackexchange.com/questions/204116/uniform-size-of-circles-around-enumitems
\node[fill=orange!30,
minimum size=1.2em,inner sep=0] (char) {#1};}}
%\usepackage{widetext}% needs packages "flushend" & "cuted" of "sttools" % bundle, which perhaps must separately be installed
\newcommand{\dd}[1]{\hspace{2pt}d#1}
\definecolor{color1}{RGB}{0,0,90} % Color of the article title and sections
\definecolor{color2}{RGB}{0,20,20} % Color of the boxes behind the abstract and
\definecolor{cinza}{HTML}{C0C0C0}
%%% Custom Exercios
\usepackage{bohr}
\usepackage{multicol}
\setlength{\columnsep}{1.5cm}
\setlength{\columnseprule}{0.2pt}
\usepackage[no-files]{xsim}
\usepackage{tasks}
\xsimsetup{
goal-print={\pgfmathprintnumber[fixed zerofill,precision=1]{#1}}
}
\newcommand*\circled[2]{\tikz[baseline=(char.base)]{
\node[shape=circle,fill,inner sep=2pt, text=white] (char) {#1};}}
%%%%%-Custom Xsim exercises %%%%%
\DeclareExerciseEnvironmentTemplate{custom}
{%\item[\GetExerciseProperty{counter}]
\Needspace*{0\baselineskip}
\noindent
\circled{\XSIMmixedcase{\GetExerciseProperty{counter}}}~~~%
\noindent
\IfInsideSolutionF{%
\GetExercisePropertyT{points}{ % notice the space
(%
\printgoal{\PropertyValue}
\IfExerciseGoalSingularTF{points}
{%\XSIMtranslate{point}
}
{% \XSIMtranslate{points}
}%
)%
}
}}
{\vspace{\baselineskip}}
%%%%%------- Custom  resposta -------%%%%%%%
\DeclareExerciseEnvironmentTemplate{space}
%{\textbf{\GetExerciseProperty{counter}} }
{\noindent\circled{\XSIMmixedcase{\GetExerciseProperty{counter}}}~~~}
% {\circled{\XSIMmixedcase{\GetExerciseProperty{counter}}}}~~~%
{\qquad}
\newcommand*\answer[1]{%
\XSIMexpandcode{%
\SetExerciseProperty{solution-body}
{\noexpand{\Alph{task}}}}%
#1%
}
%\sisetup{locale=DE}
\xsimsetup{
collect = true,
exercise/within = section,
exercise/template = custom,
exercise/the-counter =  \arabic{exercise},
solution/template= custom ,
%%solution-name = solution,  % used with headings=true
%solution/print=true,
%print-collection/print=both,
%print-solutions/collection=true
%goal-print= {\pgfmathprintnumber[fixed zerofill,precision=1]\num{#1}}
}
\RenewDocumentCommand\printpoints{}{%
\TotalExerciseTypeGoal{exercise}{points}{}{}%
}
\NewTasksEnvironment[label = (\emph{\alph*}), label-width = 12pt]{choice}[\choice]
\newenvironment{questions}{\itemize}{\enditemize}
\everymath{\displaystyle}
\DeclareExerciseHeadingTemplate{solution}{%
\section*{Gabarito}%
}
%\usepackage{filecontents}
\NewTblrTheme{fancy}{
\SetTblrStyle{firsthead}{font=\bfseries}
\SetTblrStyle{firstfoot}{fg=blue2}
\SetTblrStyle{middlefoot}{\itshape}
\SetTblrStyle{caption-tag}{red2}
}

\newcommand{\lh}{\underline{\hspace{1cm}}}
%%\onehalfspacing
\def\professor{Fábio Lima}
\def\aluno{ }
\def\numerochamada{}
\def\disciplina{Química}
%%\def\disciplina{UC III}
%%\def\disciplina{R.A.}
\def\turma{3 Ano }
%%\def\tipo{{\bfseries Avaliação Bimestral}}
\def\tipo{\bfseries Avaliação Mensal}
%\def\tipo{\bfseries Atividade Avaliativa}
\def\bimestre{1 Bimestre}
%\def\escola{E.E. 26 de Agosto}
\def\escola{E.E. José Mamede de Aquino}
%\def\escola{E.E. Joaquim Murtinho}
\def\dataprova{}
\author{fabio}
\date{\today}
\title{}
\hypersetup{
 pdfauthor={fabio},
 pdftitle={},
 pdfkeywords={},
 pdfsubject={},
 pdfcreator={Emacs 30.1 (Org mode 9.7.11)}, 
 pdflang={English}}
\begin{document}

\selectlanguage{portuguese}
\twocolumn[
%\input{../Modelos/CabeOficial}
\input{../Modelos/cabenovo}
%\input{../Modelos/mamede}
%\input{../Modelos/26agosto}
%% \input{../Modelos/geral}
%Cada questão vale {\textbf 2,0}

%%\section*{Regime de Progressão Parcial}
%\section*{Atividade}
%\section*{Trabalho}
%%\section*{\disciplina}

%{\bfseries Obrigatório a resolução das questões }

%\input{../Modelos/gabarito}

%%Total Prova: \printpoints
\smallbreak
\medbreak
\par\vspace{2ex}]%%%%\input{../Modelos/mamede}


\begin{bclogo}[logo=\bcattention, noborder=true, barre=none]{Habilidade MS.EM13CNT103 }
\begin{itemize}
\item Interpretação do conceito de velocidade das reações químicas e de como a concentração de reagentes e produtos mudam com a variação do tempo: cinética e as leis de velocidade das reações químicas; fatores que afetam a velocidade das reações químicas.
\item Proposição de explicações a partir de dados experimentais que demonstrem a relação da constante de velocidade de maneira qualitativa e quantitativa, por meio dos parâmetros físico-químicos: mecanismos de reação e catálise.
\end{itemize}
\end{bclogo}
\section{Cinética Química}
\label{sec:orgd0417d7}
\textbf{Cinética Química:} Basicamente a velocidade de uma reação pode ser calculada com base nas concentrações dos reagentes em função do tempo da reação: 

\begin{equation}
\nu = \displaystyle \frac{[A_{final}]-[A_{inicial}]}{t_{final} - t_{inicial}},\; \text{onde:}
\end{equation}
\(t\) se  refere ao  tempo e \textbf{{[}A]} é a concentração do reagente A
\subsection{Fatores que influenciam na velocidade de uma reação}
\label{sec:org8fb77b3}

\begin{enumerate}
\item \textbf{Afinidade química:} É a tendência que as substâncias químicas possuem para reagirem uma com as outras. Se a afinidade entre dois reagentes é grande, a reação ocorre com mais facilidade.
\item \textbf{Colisões eficazes:} A velocidade de uma reação é mais rápida quando as colisões são eficazes ou seja, alas ocorrem com força, ângulo e freqüência propícios para que as moléculas consigam se desagregar e formar o complexo ativado. Quando há a predominância de choque não eficazes, a velocidade da reação é lenta.
\item \textbf{Catalisador:} É uma substância que altera a velocidade de uma reação sem ser consumida durante o processo. Existem 2 tipos de catalisadores:   
\begin{itemize}
\item \textbf{Positivos} – Aumentam a velocidade de uma reação diminuindo a quantidade de energia de ativação necessária para que a reação ocorra.
\item \textbf{Negativos ou inibidores} – Retardam a velocidade de uma reação.
\end{itemize}
\end{enumerate}
\begin{center}
\begin{endiagram}[
tikz         = {xscale=1.7}, scale        = 1,
y-label-offset=2pt,
y-label-text = Energia,
x-label      = below,        x-label-text = progresso da reação,]
\ENcurve{5,8,2}
\node[above,xshift=4pt] at (N1-2) {A} ;
\ShowNiveaus[niveau=N1-1,length=1.5,shift=-.7]
\ShowNiveaus[niveau=N1-3,length=1.5,shift=.7]
\ENcurve[tikz={densely dotted,red}]{5,7,2}
\ShowNiveaus[tikz={densely dotted,red},niveau=N1-1,length=1.5,shift=-.7]
\ShowNiveaus[tikz={densely dotted,red},niveau=N1-3,length=1.5,shift=.7]
\node[above,yshift=-1cm,xshift=4pt,red] at (N1-2) {B} ;
%\ShowEa[tikz={blue,<->}]
\ShowGain[offset=-3.5,label=\(\Delta\)H]
\draw[above left] (N1-1) ++ (0.3,0) node { \ch{H2O2_{\aq}} } ;
\draw[above] (N1-3) ++ (1.2,0) node {\ch{H2O_{\lqdd} + 0.5 O2_{\gas}} } ;
\node[draw,text width=5.8cm] at (2.5, 12) {\small {\bfseries A} representa a reação sem catalisador. {\bfseries\color{red} B} representa a reação com catalisador. Observe que, com ou sem catalisador, o \(\Delta\)H da reação não se altera.};
\end{endiagram}
\end{center}
\begin{enumerate}\setcounter{enumi}{3}
\item \textbf{Catálise} é o nome dado ao aumento da velocidade provocado pelo catalisador. Existem dois tipos de catálise:
\begin{itemize}
\item \textbf{catálise homogênea:} O catalisador e os reagentes estão numa mesma fase física, formando um sistema homogêneo.
\end{itemize}
\begin{itemize}
\item \textbf{catálise heterogênea:} O catalisador e os reagentes estão em fases físicas diferentes, formando um sistema heterogêneo.
\end{itemize}
\item \textbf{Concentrações dos reagentes:} Quanto maior for a concentração dos reagentes, maior é a velocidade da reação
\item \textbf{Temperatura:} Quanto maior for a temperatura, maior será a velocidade de uma reação.
\item \textbf{Superfície de contato:} Quanto maior for a superfície de contato do sólido, maior será a velocidade da reação.
\item \textbf{Pressão:} Em reações em que há a participação dos gases, o aumento da pressão no sistema, acarreta um aumento da velocidade da reação.
\item \textbf{Estado físico dos reagentes:} De um modo geral, os gases reagem mais fácil e rapidamente do que os líquidos e estes, mais facilmente do que os sólidos.
\item \textbf{Presença de luz:} Algumas reações sofrem a influência da luz, afetando a velocidade da reação. Exemplo : A água oxigenada comercial deve ser vendida em frasco escuro e ao abrigo da luz pois a presença de luz acarreta sua decomposição em água e gás oxigênio.
\end{enumerate}
\subsection{Lei de velocidade de uma reação ou lei cinética.}
\label{sec:orgc8f7755}

A velocidade de uma reação é proporcional à concentração dos reagentes, elevadas a números que são determinados experimentalmente.

\begin{Box2}{Exemplo}
Para a reação :  \ch{aX + bY -> produtos} , a lei da velocidade tem a seguinte expressão:
\begin{equation}
\nu = k\cdot[X]^m\cdot[Y]^n \; ,\text{onde:}
\end{equation}
\(k\) = constante da velocidade que é característica da reação e da temperatura.
{[}X] e [Y] = concentração molar dos reagentes.
\(m\) e \(n\) = ordem de reação, onde \(m + n\) = ordem global da reação.

\end{Box2}
\subsection{Mecanismo das reações}
\label{sec:orgd5e0230}
Nem sempre as reações químicas ocorrem em apenas uma única etapa. Em geral, elas ocorrem em etapas. Assim, temos dois tipos de reações:

\begin{enumerate}[label=\bfseries\alph*)]
\item \textbf{Elementares} – São as que ocorrem em uma etapa única. Nesse caso, as ordens das reação (m e n) coincidem com os coeficientes do balanceamento da equação.
\item \textbf{Não elementares} – São aquelas que ocorrem em mais de uma etapa, sendo que uma delas é mais lenta. A equação da velocidade será, sempre, da etapa mais lenta. Nesse caso, as ordens da reação dependerão da etapa mais lenta.
\end{enumerate}

Analisando o exemplo, podemos observar que a síntese do óxido de ferro II (reação global) envolve três outras etapas (três outras reações químicas). Cada uma dessas etapas ocorre com uma velocidade diferente uma da outra. De acordo com \textbf{Guldberg e Waage, a etapa mais lenta é a que contribui diretamente para a determinação da velocidade da reação}.

Quando temos uma reação não elementar para realizar o cálculo da velocidade, a ordem de cada um dos participantes será determinada por cálculos individuais, já que, nesses casos, trata-se de um dado absolutamente experimental. Essa determinação das ordens dos reagentes é feita por meio de uma tabela com valores fornecida pelos exercícios, já que são dados experimentais. A seguir temos um exemplo de tabela para a reação de síntese da amônia (reação global):   

\begin{reaction*}
N2 + 3 H2 -> 2 NH3
\end{reaction*}
\begin{center}
\begin{tabular}{|c|c|c|c|}
\hline
Experimento & {[}\ch{N2}] & {[}\ch{H2}] & Velocidade (\si{\mole\per\liter\per\second})\\
\hline
I & 0,1 & 0,1 & 1\\
\hline
II & 0,2 & 0,1 & 4\\
\hline
III & 0,1 & 0,2 & 2\\
\hline
\end{tabular}
\end{center}


Na tabela podemos observar que há três experimentos (que representam etapas da reação), as concentrações dos reagentes e a velocidade de cada etapa.
Para determinar a ordem de cada um dos reagentes, já que é uma reação não elementar e não podemos utilizar os valores dos coeficientes da equação porque ela se processa em mais de uma etapa, vamos fazer um cálculo simples verificando o efeito da concentração na velocidade por meio das seguintes regras:
\begin{itemize}
\item Escolher dois experimentos na tabela;
\item Se for determinar a ordem de um reagente, a concentração do outro reagente nos dois experimentos deverá ser a mesma;
\item Comparar o efeito da concentração do reagente do qual se quer descobrir a ordem com a velocidade.
\end{itemize}
Seguindo as regras acima, vamos determinar as ordens de \ch{N2}e \ch{H2}:
\textbf{Para a ordem do \ch{N2}:}
\begin{itemize}
\item serão escolhidos os experimentos I e II;
\item nesses experimentos, a concentração de \ch{N2} mudou e a do H2 não.
\end{itemize}
Comparando a concentração com a velocidade, temos que do experimento I para o experimento II, a concentração de \ch{N2} dobrou, enquanto a velocidade quadruplicou. Comparando:
\begin{center}
{[} ] = V\\
2 = 4
\end{center}
Se colocarmos ambos com a mesma base, teremos:
\begin{center}
{[} ] = V\\
2 = \(2^2\)
\end{center}
Assim, temos que o que diferencia um do outro é o expoente 2 e, por isso, vamos considerar o reagente \ch{N2} de 2\textsuperscript{a} ordem.
\textbf{Para a ordem do \ch{H2}:}
\begin{itemize}
\item serão escolhidos os experimentos I e III;
\end{itemize}
-neles, a concentração de \ch{H2} mudou e a do \ch{N2} não.

Comparando a concentração com a velocidade, temos que do experimento I para o experimento III, a concentração de \ch{H2} dobrou e a velocidade também. Comparando:
\begin{center}
{[} ] = V\\
2 = 2
\end{center}
Assim, como eles já apresentam a mesma base, consideramos o reagente \ch{H2} de 1\textsuperscript{a} ordem. Com isso, a expressão da velocidade para esse exemplo seria:
\begin{equation*}
\nu = k\cdot[\ch{N2}]^2\cdot[\ch{H2}]^1
\end{equation*}
\end{document}
