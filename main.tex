% Created 2024-09-01 Sun 20:57
% Intended LaTeX compiler: lualatex
\documentclass[10pt]{scrartcl}


\KOMAoptions{
%headings=chapterprefix,
twocolumn=true,
%toc=indenttextentries,
%toc=flat,
twoside=true,
headinclude=true,
footinclude=true
%  captions=topbeside
}
%\usepackage[fontsize=12.3]{scrextend}
\usepackage{fontspec}
\usepackage[T1]{fontenc}
\usepackage{hyperref}
\usepackage[x11names,svgnames,table]{xcolor}
\defaultfontfeatures{Ligatures=TeX}
%%\setmainfont{Lato}
%%\setmainfont{Charis SIL}
\setmainfont{IBM Plex Serif}
\usepackage{typearea}
\usepackage{lscape}
\usepackage[a4paper]{geometry}
\geometry{a4paper,total={170mm,257mm},left=10mm,right=10mm, top=15mm, bottom=20mm}
\usepackage[english, portuguese, american]{babel}
\usepackage{amsmath,amsfonts,amsthm,bm}
\usepackage{graphicx}
\usepackage{float,wrapfig}
\usepackage{colortbl}
\usepackage{tabularx}
\usepackage{pst-labo}
\usepackage{setspace}
\usepackage{xfrac}
\usepackage{tikz}
\usepackage{pgfplots}
\pgfplotsset{compat=1.3}
%% Diagraman latex
\usepackage{endiagram}
\usepackage{smartdiagram}
\usepackage[tikz]{bclogo}
\usetikzlibrary{fit,patterns,shadows.blur,shapes,decorations.pathreplacing,decorations.markings,arrows.meta,arrows,positioning,shadows,trees}
\usetikzlibrary{decorations.pathmorphing} %% to chemfig config bond
\usepackage{upgreek}
\usepackage[modules={all}]{chemmacros}
%%\chemsetup{modules={reactions,spectroscopy,thermodynamics,redox,isotopes}}
%%\chemsetup{modules={all}}
\NewChemState\EPot{ symbol=E , subscript-pos=right , superscript=o, pre= , unit=\volt }
%\usepackage[version=4,arrows=pgf-filled]{mhchem}
\usepackage{chemfig,elements,cancel,siunitx}
\NewChemPhase\lqdd{\(\ell\)}
\NewChemPhase\gr{grafite}
\NewChemPhase\reac{reação}
\setchemfig{fixed length=false, atom sep=2.0em, arrow offset=6pt, scheme debug=false,angle increment=30}
\renewcommand{\CancelColor}{\color{red}}
\usepackage{circuitikz}
\usepackage{mol2chemfig}
\usepackage{subfig,caption}
\captionsetup{font=small, labelfont={bf,sf}}
\usepackage{wrapfig,qrcode}
\usepackage{array,longtable} % ajust colunm table
\newcolumntype{J}{>{\centering\arraybackslash}m{7.5cm}}
\newcolumntype{K}{>{\centering\arraybackslash}m{6.5cm}}
\newcolumntype{L}{>{\centering\arraybackslash}m{5cm}}
\newcolumntype{B}{>{\centering\arraybackslash}m{2.5cm}}
\newcolumntype{N}{>{\centering\arraybackslash}m{1.4cm}}
\usepackage[most]{tcolorbox}
\newcounter{mycounter}
%%% Colobor
%%% Example colorbox
\newtcolorbox{Box2}[2][]{
lower separated=false,
colback=white,
colframe=black,fonttitle=\bfseries,
colbacktitle=black,
coltitle=white,
enhanced, attach boxed title to top left={yshift=-0.1in,xshift=0.15in}, boxed title style={boxrule=0pt,colframe=white,}, title=#2,#1}
%%%%%%%% Cabecalho
\usepackage{framed,amsmath}
\newtcolorbox{mybox}[2][]{
enhanced,title=#2, fonttitle=\sffamily\small,
top=2pt,
bottom=1mm,
boxrule=0.4pt,
coltitle=black,
colback=white,
attach boxed title to top center={yshift=-\tcboxedtitleheight/2,
yshifttext=-\tcboxedtitleheight/2},
boxed title style={
colframe=white,
colback=white,
left=0.2pt,
right=0.2pt},
#1}
\usepackage{tabularray}
%%%%%%
\newtcolorbox{exercisebox}%
{enhanced,breakable,colback=white, colframe=green!15!white,colbacktitle=white!15!pink, coltitle=pink!50!black,left=0pt,right=0mm,top=3mm,bottom=3mm,pad at break=0pt,bottomrule at break=0pt,toprule at break=0pt,borderline={0mm}{0mm}{green!50!white,dashed}, attach boxed title to top center={yshift=-2mm},boxed title style={boxrule=0.4pt},title=Exercícios,}
\usepackage{eso-pic}
\usepackage{etoolbox}
\usepackage{enumitem}
\newcommand\circitem[1]{%
\tikz[baseline=(char.base)]{%https://tex.stackexchange.com/questions/204116/uniform-size-of-circles-around-enumitems
\node[circle,draw=gray, fill=gray!30,
minimum size=1.2em,inner sep=0] (char) {#1};}}
\newcommand\boxitem[1]{%
\tikz[baseline=(char.base)]{%https://tex.stackexchange.com/questions/204116/uniform-size-of-circles-around-enumitems
\node[fill=orange!30,
minimum size=1.2em,inner sep=0] (char) {#1};}}
%\usepackage{widetext}% needs packages "flushend" & "cuted" of "sttools" % bundle, which perhaps must separately be installed
\newcommand{\dd}[1]{\hspace{2pt}d#1}
\definecolor{color1}{RGB}{0,0,90} % Color of the article title and sections
\definecolor{color2}{RGB}{0,20,20} % Color of the boxes behind the abstract and
\definecolor{cinza}{HTML}{C0C0C0}
%%% Custom Exercios
\usepackage{bohr}
\usepackage{multicol}
\setlength{\columnsep}{1.5cm}
\setlength{\columnseprule}{0.2pt}
\usepackage[no-files]{xsim}
\usepackage{tasks}
\xsimsetup{
goal-print={\pgfmathprintnumber[fixed zerofill,precision=1]{#1}}
}
\newcommand*\circled[2]{\tikz[baseline=(char.base)]{
\node[shape=circle,fill,inner sep=2pt, text=white] (char) {#1};}}
%%%%%-Custom Xsim exercises %%%%%
\DeclareExerciseEnvironmentTemplate{custom}
{%\item[\GetExerciseProperty{counter}]
\Needspace*{0\baselineskip}
\noindent
\circled{\XSIMmixedcase{\GetExerciseProperty{counter}}}~~~%
\noindent
\IfInsideSolutionF{%
\GetExercisePropertyT{points}{ % notice the space
(%
\printgoal{\PropertyValue}
\IfExerciseGoalSingularTF{points}
{%\XSIMtranslate{point}
}
{% \XSIMtranslate{points}
}%
)%
}
}}
{\vspace{\baselineskip}}
%%%%%------- Custom  resposta -------%%%%%%%
\DeclareExerciseEnvironmentTemplate{space}
%{\textbf{\GetExerciseProperty{counter}} }
{\noindent\circled{\XSIMmixedcase{\GetExerciseProperty{counter}}}~~~}
% {\circled{\XSIMmixedcase{\GetExerciseProperty{counter}}}}~~~%
{\qquad}
\newcommand*\answer[1]{%
\XSIMexpandcode{%
\SetExerciseProperty{solution-body}
{\noexpand{\Alph{task}}}}%
#1%
}
%\sisetup{locale=DE}
\xsimsetup{
collect = true,
exercise/within = section,
exercise/template = custom,
exercise/the-counter =  \arabic{exercise},
solution/template= custom ,
%%solution-name = solution,  % used with headings=true
solution/print=false,
%print-collection/print=both,
%goal-print= {\pgfmathprintnumber[fixed zerofill,precision=1]\num{#1}}
}
\RenewDocumentCommand\printpoints{}{%
\TotalExerciseTypeGoal{exercise}{points}{}{}%
}
\NewTasksEnvironment[label = (\emph{\alph*}), label-width = 12pt]{choice}[\choice]
\newenvironment{questions}{\itemize}{\enditemize}
\everymath{\displaystyle}
\DeclareExerciseHeadingTemplate{solution}{%
\section*{Gabarito}%
}
%\usepackage{filecontents}
\newcommand{\lh}{\underline{\hspace{1cm}}}
%%\onehalfspacing
\def\professor{Fábio Lima}
\def\aluno{ ARIEL DA MATA FRANCO }
\def\numerochamada{05}
\def\disciplina{Química}
%%\def\disciplina{UC3}
%%\def\disciplina{R.A.}
\def\turma{3 Ano }
\def\tipo{{\bfseries Avaliação Bimestral}}
%\def\tipo{\bfseries Atividade}
%%\def\tipo{\bfseries Exame Final}
\def\bimestre{2 Bimestre}
%\def\escola{E.E. 26 de Agosto}
%\def\escola{E.E. José Mamede de Aquino}
\def\escola{E.E. Amélio de Carvalho Baís}
\def\dataprova{}
\DeclareExerciseCollection{Cadeias}
\DeclareExerciseCollection{Hibridizacao}
\DeclareExerciseCollection{Carbonos}
\author{fabio}
\date{\today}
\title{}
\hypersetup{
 pdfauthor={fabio},
 pdftitle={},
 pdfkeywords={},
 pdfsubject={},
 pdfcreator={Emacs 29.4 (Org mode 9.6.15)}, 
 pdflang={English}}
\begin{document}

\twocolumn[
\input{../Modelos/CabeOficial}
%%\input{../Modelos/cabenovo}
%%%\input{../Modelos/mamede}
%\input{../Modelos/26agosto}
%% \input{../Modelos/geral}
%Cada questão vale {\textbf 2,0}

%%\section*{Regime de Progressão Parcial}
%\section*{Atividade}
%%%\section*{Trabalho}
%%\section*{\disciplina}



%\input{../Modelos/gabarito}

%Total Prova: \printpoints
\smallbreak
\medbreak
\par\vspace{2ex}]%%%%\input{../Modelos/mamede}



\collectexercises{Cadeias}



\begin{exercise}[points=1.0]
Alcinos são hidrocarbonetos:
\begin{choice}
\choice alifáticos saturados.
\choice alicíclicos saturados.
\choice alifáticos insaturados com dupla ligação.
\choice alicíclicos insaturados com tripla ligação.
\choice alifáticos insaturados com tripla ligação.
\end{choice}
\end{exercise}
\begin{solution}
Letra E
\end{solution}


\begin{exercise}[points=1.0]
Considere as afirmações seguintes sobre hidrocarbonetos.

\begin{enumerate}[label=\Roman*)]
\item Hidrocarbonetos são compostos orgânicos constituídos somente de carbono e hidrogênio.
\item São chamados de alcenos somente os hidrocarbonetos insaturados de cadeia linear.
\item Cicloalcanos são hidrocarbonetos alifáticos saturados de fórmula geral  C$_{n}$H$_{2n}$ .
\item São hidrocarbonetos aromáticos: bromobenzeno,  $p$-nitrotolueno e naftaleno.
\end{enumerate}

São corretas as afirmações:
\begin{choice}
\choice I e III, apenas.
\choice I, III e IV, apenas.
\choice II e III, apenas.
\choice III e IV, apenas.
\choice I, II e IV, apenas.
\end{choice}
\end{exercise}
\begin{solution}
LETRA A
\end{solution}


\begin{exercise}[points=1.0]
A classificação quanto à cadeia carbônica, da molécula é
\begin{center}
\chemfig{CH_3-CH=CH-CH_2-NO_2}
\end{center}
\begin{choice}
\choice alifática, saturada, homogênea , normal.
\choice cíclica, insatura, homogênea, ramificada.
\choice alifática, insaturada, homogêna, normal.
\choice alicíclica, saturada, heterogênea, normal.
\choice aberta, insaturada, heterogênea, ramificada.
\end{choice}
\end{exercise}





\begin{exercise}[points=1.0]
O propanoato de metila, representado a seguir, apresenta cadeia carbônica:
\begin{center}
\chemfig{CH_3-CH_2-C([:90]=O)-O-CH_3}
\end{center}
\begin{choice}
\choice alifática, normal, saturada e heterogênea.
\choice alicíclica, normal, saturada e heterogênea.
\choice aberta, normal, insaturada e heterogênea.
\choice acíclica, normal, saturada e homogênea.
\choice alifática, ramificada, insaturada e homogênea.
\end{choice}
\end{exercise}


\begin{exercise}[points=1.0]
A caprolactama, matéria-prima para fabricação do nylon 6, apresenta a fórmula estrutural:
\begin{center}
\chemfig{*7(----N(-H)-(=O)--)}
\end{center}
A cadeia carbônica da caprolactama pode ser classificada, corretamente, como:
\begin{choice}
\choice cíclica, aromática e homogênea.
\choice acíclica, aromática e homogênea.
\choice cíclica, saturada e heterogênea.
\choice acíclica, alifática e heterogênea.
\choice cíclica, aromática e heterogênea.
\end{choice}
\end{exercise}



\begin{exercise}[points=1.0]
O linalol, substância isolada do óleo de alfazema, apresenta a seguinte fórmula estrutural:
\begin{center}
\chemfig{=[:90]-[:150](-[:150])(-[:30,,,1]OH)-[:210]-[:270]-[:330]=[:270](-[:210])-[:330]}
\end{center}
Essa cadeia carbônica é classificada como:
\begin{choice}
\choice acíclica, normal, insaturada e homogênea.
\choice acíclica, ramificada, insaturada e homogênea.
\choice alicíclica, ramificada, insaturada e homogênea.
\choice alicíclica, normal, saturada e heterogênea.
\choice acíclica, ramificada, saturada, e heterogênea.
\end{choice}
\end{exercise}






\begin{exercise}[points=1.0]
O pau-rosa, típico da região amazônica, é uma rica fonte natural do óleo essencial conhecido por linalol, o qual também pode ser isolado do óleo de alfazema. Esse óleo apresenta a seguinte fórmula estrutural:

\begin{center}
\chemfig{H_3C-C([:-90]-CH_3)=CH-CH_2-CH_2-C([:90]-OH)([:-90]-CH_3)-CH=CH_2}
\end{center}

Sua cadeia carbônica deve ser classificada como:
\begin{choice}
\choice alicíclica, ramificada, insaturada e homogênea.
\choice acíclica, ramificada, saturada e heterogênea.
\choice acíclica, ramificada, insaturada e homogênea.
\choice alicíclica, normal, saturada e heterogênea.
\choice alicíclica, ramificada, saturada e heterogênea.
\end{choice}
\end{exercise}
\begin{solution}
C
\end{solution}











\begin{exercise}[points=1]
A cadeia da molécula do ácido butírico é classificada
como:

\chemfig{H_3C-CH_2-CH_2-C([:90]=O)-OH}

\begin{choice}
\choice acíclica, normal, saturada e homogênea.
\choice aberta, normal, insaturada e heterogênea.
\choice alicíclica, normal, insaturada e homogênea.
\choice acíclica, ramificada, saturada e homogênea.
\choice cíclica, ramificada, insaturada e heterogênea.
\end{choice}
\end{exercise}
\begin{solution}
A
\end{solution}






\begin{exercise}[points=1]
O mirceno, responsável pelo "gosto azedo da cerveja", é representado pela estrutura:

\begin{center}
\chemfig{=[:90]-[:150](=[:90,,,1]CH_2)-[:210]-[:270]-[:330]=[:270](-[:330])%
-[:210]}
\end{center}

Considerando o composto indicado,assinale a alternativa correta quanto à classificação da cadeia

\begin{choice}
\choice acíclica, homogênea, saturada
\choice acíclica, heterogênea, insaturada
\choice cíclica, heterogênea, insaturada
\choice aberta, homogênea, saturada
\choice aberta, homogênea, insaturada
\end{choice}
\end{exercise}
\begin{solution}
E
\end{solution}







\begin{exercise}[points=1]
A cadeia carbônica abaixo é classificada como:

\chemfig{-[:330]=[:30]-[:330](-[:270,,,1]OH)-[:30](-[:90])-[:330]-[:30]}
\begin{choice}
\choice Aberta, ramificada, insaturada, heterogênea
\choice Alicíclica, ramificada, insaturada, heterogênea
\choice Acíclica, ramificada, insaturada, homogênea
\choice Alifática, linear, saturada, homogênea
\choice  Aberta, linear, saturada, heterogênea
\end{choice}
\end{exercise}
\begin{solution}
C
\end{solution}


\begin{exercise}[points=1]
Na reação:

\begin{center}


\resetchemfig 
\schemestart
\chemname{\chemfig{NH_4CNO}}{Cianato de Amônio}
\arrow{->[\Delta]}
\chemfig{O=C([7]-NH_2)([1]-NH_2)}
\schemestop
\end{center}

o produto foi preparado em laboratório, pela primeira vez, por:
\begin{choice}
\choice Bunsen
\choice Arrhenius
\choice Le Bel e van’t Hoff
\choice Wöhler
\choice Berzellus
\end{choice}
\end{exercise}
\begin{solution}
E
\end{solution}


\begin{exercise}[points=1]
A cafeína, um estimulante bastante comum no café, chá, guaraná etc., tem a seguinte fórmula estrutural

\chemfig{H_3C-[:42,,2]N-[:96]=_[:24]N-[:312]=_[:240](-[:168]\phantom{N})%
-[:300](=[:240]O)-N(-[:300,,,1]CH_3)-[:60](=O)-[:120]N(-[:180])%
-[:60,,,1]CH_3}

Podemos afirmar corretamente que a fórmula molecular da cafeína é:
\begin{choice}
\choice \ch{C5H9N4O2}
\choice \ch{C6H10N4O2}
\choice \ch{C6H9N4O2}
\choice \ch{C3H9N4O2}
\choice \ch{C8H10N4O2}
\end{choice}
\end{exercise}
\begin{solution}
E
\end{solution}




\begin{exercise}[points=1]
A cadeia carbônica acíclica, ramificada, homogênea e
insaturada é

\begin{choice}
\choice \chemfig{CH_3-CH_2-CH=CH_2}
\choice \chemfig{*3(=--)}
\choice \chemfig{H_3C-CH([:-90]-CH_3)-CH_2-C([1]=O)([:330]-OH)}
\choice \chemfig{H_3C-CH_2-C([:-90]-CH_3)=CH-CH([:90]-CH_3)-O-CH_3}
\choice \chemfig{H_3C-C([:90]-CH_3)=CH-CH_2-CH_2-C([:90]-CH_3)=CH-C([1]=O)([:330]-H)}
\end{choice}
\end{exercise}
\begin{solution}
E
\end{solution}


\begin{exercise}[points=1]
A substância de fórmula \chemfig{CH_3-O-CH_2-CH_3} tem cadeia carbônica

\begin{choice}
\choice acíclica, homogênea e normal.
\choice cíclica, heterogênea e ramificada.
\choice cíclica, homogênea e saturada.
\choice acíclica, insaturada e heterogênea.
\choice acíclica, saturada e heterogênea
\end{choice}
\end{exercise}
\begin{solution}
E
\end{solution}


\begin{exercise}[points=1]
A acrilonitrila,\chemfig{H_2C=CH-CN}, matéria-prima usada na obtenção de fibras têxteis, tem cadeia carbônica:

\begin{choice}
\choice acíclica e ramificada.
\choice cíclica e insaturada.
\choice cíclica e ramificada.
\choice aberta e homogênea.
\choice aberta e saturada.
\end{choice}
\end{exercise}
\begin{solution}
D
\end{solution}






\begin{exercise}[points=1]
Preocupações com a melhoria da qualidade de vida levaram a propor a substituição do uso do PVC pelo poliureftalato de etileno ou PET, menos poluentes na combustão. Esse polímero está relacionado com os compostos:

\begin{center}
\schemestart
\chemname{\chemfig{HO-[:60,,2](=[:120]O)-=_[:60]-=_[:300](-(=[:60]O)-[:300,,,1]OH)%
-[:240]=_[:180](-[:120])}}{I - Ácido terftálico} \qquad
\chemname{\chemfig{H_2C=CH_2}}{II -Etileno}
\schemestop
\end{center}

\begin{choice}
\choice alicíclica e acíclica.
\choice saturada e insaturada.
\choice heterocíclica e aberta.
\choice aromática e insaturada.
\choice acíclica e homogênea
\end{choice}
\end{exercise}
\begin{solution}

\end{solution}


\begin{exercise}[points=1]
O composto (5) (+) Sulcatol, cuja fórmula estrutural é mostrada abaixo, é um
feromônio sexual do besouro da madeira (\emph{Gnothotricus retusus})

\begin{center}
\chemname{\chemfig{H_3C-[:30,,2](-[:90,,,1]CH_3)=[:330]-[:30]-[:330]-[:30](%
-[:330,,,1]CH_3)(<:[:150]H)<[:30,,,1]OH}}{(5) (+) Sulcatol}
\end{center}

Com relação ao (5) (+) Sulcatol, pode-se afirmar que ele possui cadeia carbônica:
\begin{choice}
\choice alifática, homogênea, insaturada e ramificada.
\choice alicíclica, heterogênea, insaturada e ramificada.
\choice acíclica, homogênea, insaturada e normal.
\choice alifática, homogênea, saturada e ramificada.
\choice homocíclica, insaturada, heterogênea e ramificada.
\end{choice}
\end{exercise}
\begin{solution}
A
\end{solution}








\begin{exercise}[points=1]
A sibutramina é um fármaco controlado pela Agência Nacional de Vigilância Sanitária que tem por finalidade agir como moderador de apetite.
Sobre a sibutramina, é incorreto afirmar que:

\chemfig{-[:210](-[:150])-[:270]-[:330](-[:30]N(-[:90])-[:330])-[:270](-%
-[:270]-[:180]-[:90])-[:180]=_[:240]-[:180]=_[:120](-[:180]C{\ell})-[:60]=_(%
-[:300])}

\begin{choice}
\choice trata-se de uma substância aromática.
\choice sua fórmula molecular é \ch{C17H25NC$\ell$}
\choice  identifica-se um elemento da família dos halogênios em sua
estrutura.
\choice identifica-se a presença de ligações \(\pi\) (pi) em sua estrutura.
\choice O composto é um álcool
\end{choice}
\end{exercise}
\begin{solution}

\end{solution}





\begin{exercise}[points=1]
Considerando a metionina e a cisteína, assinale a afirmativa correta sobre suas estruturas.


\begin{center}
\chemname{\chemfig{H_3C-S-CH_2-CH_2-CH([:-90]-NH_2)-COOH}}{Metionina} \par

\chemname{\chemfig{HS-CH_2-CH([:-90]-NH_2)-COOH}}{Cisteína}
\end{center}
\begin{choice}
\choice Ambos os aminoácidos apresentam um átomo de carbono cuja hibridização é sp\textsuperscript{2} e cadeia carbônica homogênea.
\choice Ambos os aminoácidos apresentam um átomo de carbono cuja hibridização é sp\textsuperscript{2}, mas a metionina tem cadeia carbônica heterogênea e a cisteína, homogênea.
\choice Ambos os aminoácidos apresentam um átomo de carbono cuja hibridização é sp\textsuperscript{2} e cadeia carbônica heterogênea.
\choice Ambos os aminoácidos apresentam os átomos de carbono com hibridização sp e cadeia carbônica homogênea.
\choice Os compostos são aromáticos
\end{choice}
\end{exercise}
\begin{solution}

\end{solution}





\collectexercisesstop{Cadeias}

\collectexercises{Hibridizacao}




\begin{exercise}[points=1]
O chá da planta Bidens pilosa, conhecida vulgarmente pelo nome de picão, é usado para combater icterícia de recém-nascidos. Das folhas dessa planta, é extraída uma substância química, cujo nome oficial é 1 - fenilepta - 1, 3, 5 - triino e cuja estrutura é apresentada abaixo. Essa substância possui propriedades antimicrobianas e, quando irradiada com luz ultravioleta, apresenta atividade contra larvas de mosquitos e nematóides.
Sobre a estrutura dessa substância, pode-se afirmar que:

\begin{center}
\chemfig{*6(-=-([:0]-~-~-~-)=-=)}
\end{center}

\begin{choice}
\choice possui 12 átomos de carbono com hibridização sp\textsuperscript{2}.
\choice possui 12 ligações  carbono-carbono.
\choice não possui carbonos com hibridização sp\textsuperscript{3}.
\choice possui 3 átomos de carbono com hibridização sp.
\choice possui 9 ligações \(\pi\) carbono-carbono
\end{choice}
\end{exercise}
\begin{solution}

\end{solution}




\begin{exercise}[points=1]
As fenil-ureias substituídas pertencem ao primeiro grupo de herbicidas de alta eficiência
introduzido em 1956, do qual o [3–(3,4–diclorofenil)–1,1–dimetiluréia] (DCMU) faz parte.

\begin{center}
\chemname{\chemfig{H-[:30]N(-[:330](=[:270]O)-[:30]N(-[:90])-[:330])-[:90]=^[:30]%
-[:90](-[:30]C{\ell})=^[:150](-[:90]C{\ell})-[:210]=^[:270](-[:330])}}{DCMU}
\end{center}

Em relação à molécula do DCMU, é correto afirmar, exceto:

\begin{choice}
\choice Apresenta o fenômeno de ressonância.
\choice Apresenta carbonos trigonais e tetraédricos.
\choice Sua fórmula molecular é \ch{C8H10N2C$\ell$2O}.
\choice Possui três hidrogênios ligados a carbonos aromáticos.
\choice Apresenta ligação do tipo sp.
\end{choice}
\end{exercise}
\begin{solution}

\end{solution}








\begin{exercise}[points=1]
\begin{tcolorbox}[minipage,colback=white,arc=0pt,outer arc=0pt]
{\itshape ... O carbono é tretavalente}
 \par \hspace{2.5cm}
 \scriptsize A. Kekulé,1858
\end{tcolorbox}
A distribuição eletrônica do carbono, no estado fundamental, entretanto, mostra que ele é bivalente. Para que o carbono atenda ao postulado de Kekulé, ele sofre

\begin{choice}
\choice ressonância.
\choice isomeria.
\choice protonação.
\choice hibridização.
\choice efeito indutivo.
\end{choice}
\end{exercise}
\begin{solution}
D
\end{solution}






\begin{exercise}[points=1]
A morfina, uma droga utilizada em tratamento de câncer, tem a fórmula estrutural:

\begin{center}
%\setchemfig{fixed length=false}
\resetchemfig 
\chemfig{[:-30]**6(-(-OH)-?-*6(\chemabove{}{\quad\mathbf{3}}-(-[3]-[2,2]-[0,.5])*6(-(<:[:-150,1.155]O?)
-(<:OH)-\chemabove{}{\quad \mathbf{1}}=-)-(<:[1]H)-(-[2]N-\chemabove{C}{\mathbf{2}}H_3)--)---)}
\end{center}

Os carbonos assinalados possuem hibridização, respectivamente:

\begin{choice}
\choice \textbf{1}-sp\textsuperscript{2} \textbf{2}-sp\textsuperscript{3} \textbf{3}-sp\textsuperscript{2}.
\choice \textbf{1}-sp   \textbf{2}-sp\textsuperscript{3} \textbf{3}-sp\textsuperscript{3}.
\choice \textbf{1}-sp\textsuperscript{2} \textbf{2}-sp   \textbf{3}-sp\textsuperscript{2}.
\choice \textbf{1}-sp   \textbf{2}-sp\textsuperscript{3} \textbf{3}-sp\textsuperscript{3}.
\choice \textbf{1}-sp   \textbf{2}-sp   \textbf{3}-sp\textsuperscript{2}
\end{choice}
\end{exercise}
\begin{solution}
A
\end{solution}


\begin{exercise}[points=1]
"Segundo a WWF, três novos estudos científicos mostram que as mudanças biológicas nos sistemas hormonais e imunológicos dos ursos polares estão ligadas a poluentes tóxicos em seus corpos. Entre os produtos químicos mais perigosos estão os PCBs - bifenilas policloradas, substâncias industriais que foram banidas nos anos 80, mas que ainda são encontradas nas águas, no gelo e no solo do Ártico."
\emph{JB online, 2004}

A figura abaixo apresenta um exemplo de bifenila policlorada.

\begin{center}
\chemname{
\chemfig{C{\ell}-[:330]=_[:30](-[:90]C{\ell})-[:330](-[:30]C{\ell})=_[:270]-[:210]=_[:150]%
(-[:90])-[:210]-[:270]=_[:210]-[:150]=_[:90](-[:150]C{\ell})-[:30](=_[:330])%
-[:90]C{\ell}}}{2,3,2',3',4' pentaclorobifenila}
\end{center}

Assinale a opção que apresenta o número total de átomos de carbono com hibridação sp\textsuperscript{2} no composto.

\begin{choice}(2)
\choice 6
\choice 8
\choice 10
\choice 12
\choice 14
\end{choice}
\end{exercise}
\begin{solution}
D
\end{solution}


\begin{exercise}[points=1]
Tanto a borracha natural quanto a sintética são materiais poliméricos. O precursor da borracha natural é o priofosfato de geranila, sintetizado em rota bioquímica a partir do geraniol, que apresenta a estrutura

\begin{center}
\chemfig{-[:330](-[:270])=[:30]-[:330]-[:30]-[:330](-[:270])=[:30]-[:330]%
-[:30,,,1]OH}
\end{center}

O precursor da borracha sintética é o isopreno, que apresenta a estrutura:

\begin{center}
\chemfig{H_3C-[:30,,2]C(=[:90,,,1]CH_2)-[:330]\mcfbelow{C}{H}=[:30,,,1]CH_2%
}
\end{center}

Ambas as estruturas resultam no poliisopreno e são vulcanizadas com o objetivo de melhorar as propriedades mecânicas do polímero. A hibridação do carbono ligado ao oxigênio na estrutura do geraniol é do tipo:

\begin{choice}
\choice sp
\choice sp\textsuperscript{2}
\choice sp\textsuperscript{3}
\choice s
\choice p
\end{choice}
\end{exercise}
\begin{solution}
C
\end{solution}






\begin{exercise}[points=1]
"Gota” é uma doença caracterizada pelo excesso de ácido úrico no organismo. Normalmente, nos rins, o ácido úrico é filtrado e segue para a bexiga, de onde será excretado pela urina. Por uma falha nessa filtragem ou por um excesso de produção, os rins não conseguem expulsar parte do ácido úrico. Essa porção extra volta para a circulação, permanecendo no sangue. A molécula do ácido úrico, abaixo, é um composto que:

\begin{center}
\chemfig{O=C-[:306]\mcfbelow{N}{H}-[:18]C=^[:90]C(-[:162]\mcfabove{N}{H}%
-[:234]\phantom{C})-[:30]C(=[:90]O)-[:330,,,1]NH-[:270,,1]C(=[:330]O)%
-[:210]\mcfbelow{N}{H}(-[:150]\phantom{C})}
\end{center}
\begin{choice}
\choice possui o anel aromático em sua estrutura;
\choice apresenta quatro ligações \(\pi\) (pi) e treze ligações \(\sigma\) (sigma);
\choice é caracterizado por carbonos que apresentam hibridização sp\textsuperscript{2};
\choice apresenta a cadeia carbônica cíclica com dois radicais.
\choice apresenta ligações sp e sp\textsuperscript{2} apenas.
\end{choice}
\end{exercise}
\begin{solution}
C
\end{solution}



\begin{exercise}[points=1]
A pentoxiverina é utilizada como produto terapêutico no combate à tosse. Indique a alternativa cujos itens relacionam-se com a estrutura fornecida:

{\centering 
\chemfig{-[:18]-[:78]N(-[:138]-[:198])-[:18]-[:318]-[:18]O-[:318]-[:18]%
-[:318]O-[:18](=[:78]O)-[:318](-[:330]-[:258]-[:186]-[:114]-[:42])-[:54]%
=_[:114]-[:54]=_[:354]-[:294]=_[:234](-[:174])}
}
\begin{choice}
\choice 8 elétrons \(\pi\), 12 elétrons não-ligantes, 7 carbonos
sp\textsuperscript{2} e 12 carbonos sp3.
\choice 6 elétrons \(\pi\), 12 elétrons não-ligantes, 7 carbonos
sp\textsuperscript{2} e 12 carbonos sp\textsuperscript{3}.
\choice 6 elétrons \(\pi\), 14 elétrons não-ligantes, 6 carbonos
sp\textsuperscript{2} e 12 carbonos sp\textsuperscript{3}.
\choice 8 elétrons \(\pi\), 14 elétrons não-ligantes, 7 carbonos
sp\textsuperscript{2} e 13 carbonos sp\textsuperscript{3}.
\choice 8 elétrons \(\pi\), 12 elétrons não-ligantes, 6 carbonos
sp\textsuperscript{2} e 13 carbonos sp\textsuperscript{3}.
\end{choice}
\end{exercise}
\begin{solution}

\end{solution}






\begin{exercise}[points=1]
O composto \chemfig{H-C~C-H}  deve apresentar na
sua estrutura, para cada carbono:

\begin{choice}
\choice 2 ligações sigma e 2 ligações pi
\choice 2 ligações sigma e 3 ligações pi
\choice 3 ligações sigma e 2 ligações pi
\choice 5 ligações sigma
\choice somente ligações pi
\end{choice}
\end{exercise}
\begin{solution}

\end{solution}


\begin{exercise}[points=1]
O tingimento na cor azul de tecidos de algodão com o corante índigo, feito com o produto natural ou com o obtido sinteticamente, foi o responsável pelo sucesso do \textbf{jeans} em vários países
Observe a estrutura desse corante:

\begin{center}
\chemfig{O=[:108]-[:162]=_[:210]-[:150]=_[:90]-[:30]=_[:330](-[:270])%
-[:18]\mcfabove{N}{H}-[:306](-[:234])=-[:306]\mcfbelow{N}{H}-[:18]-[:90](%
-[:162](=[:108]O)-[:234])=_[:30]-[:330]=_[:270]-[:210](=_[:150])}
\end{center}

Nessa substância, encontramos um número de ligações \(\pi\) correspondente a:

\begin{choice}(2)
\choice 3
\choice 6
\choice 9
\choice 10
\choice 12
\end{choice}
\end{exercise}
\begin{solution}
C
\end{solution}


\begin{exercise}[points=1]
Observando a benzilmetilcetona, que apresenta a fórmula estrutural abaixo, pode-se afirmar que ela contém:

\begin{center}
\chemfig{H_3C-[:60,,2](=[:120]O)-O-[:300]=^[:240]-[:300]=^-[:60]=^[:120](%
-[:180])-[:60](=[:120]O)-[,,,1]OH}
\end{center}
\begin{choice}
\choice 6 carbonos sp\textsuperscript{2} e 2 carbonos sp\textsuperscript{3}.
\choice 8 carbonos sp\textsuperscript{2} e 1 carbono sp\textsuperscript{3}.
\choice 2 carbonos sp\textsuperscript{2} e 7 carbonos sp\textsuperscript{3}.
\choice 7 carbonos sp\textsuperscript{2} e 2 carbonos sp\textsuperscript{3}.
\choice 9 carbonos sp\textsuperscript{2}
\end{choice}
\end{exercise}
\begin{solution}
D
\end{solution}






\begin{exercise}[points=1]
Observe a fórmula estrutural da aspirina, mostrada abaixo:

\begin{center}
\chemfig{H_3C-[:60,,2](=[:120]O)-O-[:300]=^[:240]-[:300]=^-[:60]=^[:120](%
-[:180])-[:60](=[:120]O)-[,,,1]OH}
\end{center}

Pode-se afirmar que a aspirina contém:
\begin{choice}
\choice 2 carbonos sp\textsuperscript{2} e 1 carbono sp\textsuperscript{3}
\choice 2 carbonos sp\textsuperscript{2} e 7 carbonos sp\textsuperscript{3}
\choice 8 carbonos sp\textsuperscript{2} e 1 carbono sp\textsuperscript{3}
\choice 2 carbonos sp\textsuperscript{2}, 1 carbono sp\textsuperscript{3} e 6 carbonos sp
\choice 2 carbonos sp\textsuperscript{2}, 1 carbono sp e 6 carbonos sp\textsuperscript{3}
\end{choice}
\end{exercise}
\begin{solution}
C
\end{solution}






\collectexercisesstop{Hibridizacao}
\collectexercises{Carbonos}

\begin{exercise}[points=1.0]
A prednisona é um glicocorticóide sintético de potente ação antireumática, antiflamatória e antialérgica, cujo uso, como de qualquer outro derivado da cortisona, requer uma série de precauções em função dos efeitos colaterais que pode causar. Os pacientes submetidos a esse tratamento devem ser periodicamente
monitorados, e a relação entre o benefício e reações adversas deve ser um fator preponderante na sua indicação.
\begin{center}
\chemfig[cram width=2.5pt,cram dash width=1.pt]{>[:234]-[:120]-[:180](=[:120]O)>[:240]-[:300](-(<[:312]-[:24]-[:96](-(-[:300]-[,,,1]OH)=[:60]O)(<:[:74,,,1]OH)-[:168])-[:60])<:[:240]-[:180]-[:120]=_[:180]-[:120](=[:180]O)-[:60]=_-[:300](-)(-[:240])<[:60]}
\end{center}

Com base na fórmula estrutural apresentada acima, qual o número de átomos de carbono terciários que ocorrem em cada molécula da prednisona?

\begin{choice}(2)
\choice 3
\choice 4
\choice 5
\choice 6
\choice 7
\end{choice}
\end{exercise}
\begin{solution}
4
\end{solution}






\begin{exercise}[points=1.0]
Uma cadeia carbônica alifática, homogênea, saturada, apresenta um átomo de carbono
secundário, dois átomos de carbono quaternário e um átomo de carbono terciário. Esta cadeia apresenta:

\begin{choice}
\choice 7 átomos de C.
\choice 8 átomos de C.
\choice 9 átomos de C.
\choice 10 átomos de C.
\choice 11 átomos de C.
\end{choice}
\end{exercise}





\begin{exercise}[points=1.0]
No composto
\begin{center}
\chemfig{H_3C-CH_2-CH([:-90]-CH_3)-CH([:90]-CH_2-CH_3)-CH([:-90]-CH_3)-CH([:-90]-CH_2-CH_2-CH_3)-CH_3}
\end{center}
As quantidades totais de átomos de carbono primário, secundário e terciário são, respectivamente:

\begin{choice}
\choice 5, 2 e 3 
\choice 3, 5 e 2 
\choice 4, 3 e 5
\choice 6, 4 e 4
\choice 3, 4 e 5
\end{choice}
\end{exercise}
\begin{solution}
D
\end{solution}



\begin{exercise}[points=1.0]
O composto orgânico de fórmula plana abaixo possui:

\begin{center}
\chemfig{CH_3-CH([:-90]-CH_3)-C([:90]-CH_3)([:-90]-CH_3)-CH_2-CH_3}
\end{center}

\begin{choice}
\choice  5 carbonos primários, 3 secundários, 1 terciário e 2 quaternários
\choice 3 carbonos primários, 3 secundários, 1 terciário e 1 quaternário.
\choice 5 carbonos primários, 1 secundário, 1 terciário e 1 quaternário.
\choice 4 carbonos primários, 1 secundário, 2 terciários e 1 quaternário.
\choice 3 carbonos primários, 3 secundário, 2 terciários e 1 quaternário.
\end{choice}
\end{exercise}
\begin{solution}
C
\end{solution}





\begin{exercise}[points=1]
Isoflavonas são compostos encontrados em sementes de soja associados a proteínas. Seu consumo
frequente pode auxiliar as mulheres a minimizar os efeitos negativos da menopausa. A estrutura química de uma isoflavona está representada abaixo: Sua estrutura química relativamente plana, fundamental em suas propriedades, é uma conseqüência das características dos carbonos envolvidos.
Considerando-se a estrutura da isoflavona, quantos carbonos terciários podem ser evidenciados?

\chemfig{OH-[:150,,1]=_[:210]-[:150]=_[:90](-[:30]=_[:330]-[:270])-[:150]%
-[:210](=[:270]O)-[:150]=_[:210](-[:270,,,1]OH)-[:150]=_[:90](-[:150,,,2]HO%
)-[:30]=_[:330](-[:270])-[:30]O-[:330](=_[:270])}

\begin{choice}(2)
\choice 0
\choice 1
\choice 2
\choice 3
\choice 4
\end{choice}
\end{exercise}

\begin{solution}
D
\end{solution}



\begin{exercise}[points=1]
Muitos inseticidas utilizados na agricultura e no ambiente doméstico pertencem à classe de compostos denominados piretróides. Dentre os muitos piretróides disponíveis comercialmente, encontra-se a deltametrina, cujo isômero mais potente tem sua fórmula estrutural representada a seguir:

\chemfig{-[:230](-[:310])-[:210](<[:240]-[:300,,,,dlh](-[:240]Br)-Br)-[:90]%
(-[:330])<[:120](=[:60]O)-[:180]O>[:120](-[:60]-[:60,,,,trpl={0}{0}]N)%
-[:180]-[:240,,,,drh]-[:180](-[:120,,,,drh]-[:60]-[,,,,drh]-[:300])-[:240]O%
-[:180]-[:240,,,,drh]-[:180]-[:120,,,,drh]-[:60]-[,,,,drh](-[:300])}

\begin{choice}
\choice Existe um carbono quaternário.
\choice O composto apresenta dez ligações pi. 
\choice O composto possui três carbonos assimétricos.
\choice O composto possui sete carbonos quaternários.
\choice O composto possui quinze carbonos com hibridação sp\textsuperscript{2} e um carbono sp
\end{choice}
\end{exercise}
\begin{solution}

\end{solution}




\begin{exercise}[points=1]
A borracha natural é um líquido branco e leitoso, extraído da seringueira, conhecido como látex. O monômero que origina a borracha natural é o 2 -metil- 1, 3-butadieno. Sua fórmula estrutural está representada abaixo.


\begin{center}
\chemfig{H_2C=C([:-90]-CH_3)-CH=CH_2}
\end{center}

Sobre a estrutura do monômero, é correto afirmar que:
\begin{choice}
\choice é um hidrocarboneto insaturado de fórmula molecular \ch{C5H8}.
\choice é um hidrocarboneto de cadeia saturada e ramificada.
\choice tem fórmula molecular \ch{C4H5}.
\choice apresenta dois carbonos terciários, um carbono secundário e dois carbonos primários.
\choice apresenta dois carbonos carbonos quartenários.
\end{choice}
\end{exercise}
\begin{solution}

\end{solution}






\collectexercisesstop{Carbonos}

\printcollection{Carbonos}
\printcollection{Cadeias}
\printcollection{Hibridizacao}
\end{document}
