% Created 2025-02-27 Thu 04:47
% Intended LaTeX compiler: lualatex
\documentclass[10pt]{scrartcl}


\KOMAoptions{
%headings=chapterprefix,
twocolumn=true,
%toc=indenttextentries,
%toc=flat,
twoside=true,
headinclude=true,
footinclude=true
%  captions=topbeside
}
%\usepackage[fontsize=12.3]{scrextend}
\usepackage{fontspec}
\usepackage[T1]{fontenc}
\usepackage[english, portuguese, american]{babel}
\usepackage{hyperref}
\usepackage[x11names,svgnames,table]{xcolor}
\defaultfontfeatures{Ligatures=TeX}
%%\setmainfont{Lato}
%%\setmainfont{Charis SIL}
\setmainfont{IBM Plex Serif}
\usepackage{typearea}
\usepackage{lscape}
\usepackage[a4paper]{geometry}
\geometry{a4paper,total={170mm,257mm},left=10mm,right=10mm, top=15mm, bottom=20mm}
\usepackage[english,portuguese]{babel}
\usepackage{amsmath,amsfonts,amsthm,bm}
\usepackage{graphicx}
\usepackage{float,wrapfig}
\usepackage{colortbl}
\usepackage{tabularx}
\usepackage{pst-labo}
\usepackage{setspace}
\usepackage{xfrac}
\usepackage{tikz}
\usepackage{pgfplots}
\pgfplotsset{compat=1.3}
%% Diagraman latex
\usepackage{endiagram}
\usepackage{smartdiagram}
\usepackage[tikz]{bclogo}
\usetikzlibrary{fit,patterns,shadows.blur,shapes,decorations.pathreplacing,decorations.markings,arrows.meta,arrows,positioning,shadows,trees}
\usetikzlibrary{decorations.pathmorphing} %% to chemfig config bond
\usepackage{upgreek}
\usepackage[modules={all}]{chemmacros}
%%\chemsetup{modules={reactions,spectroscopy,thermodynamics,redox,isotopes}}
%%\chemsetup{modules={all}}
\NewChemState\EPot{ symbol=E , subscript-pos=right , superscript=o, pre= , unit=\volt }
%\usepackage[version=4,arrows=pgf-filled]{mhchem}
\usepackage{chemfig,elements,cancel,siunitx}
\NewChemPhase\lqdd{\(\ell\)}
\NewChemPhase\gr{grafite}
\NewChemPhase\reac{reação}
%\setchemfig{fixed length=false, atom sep=2.0em, arrow offset=6pt, scheme debug=false,angle increment=30}
\setchemfig{angle increment=30, atom sep=1.67em, double bond sep=0.67ex, bond style={line width=0.1em}, cram width=0.8ex, cram dash width=0.1em, cram dash sep=0.2em, arrow style={line width=0.067em},  arrow head=-{Triangle}, arrow label sep=1ex, cycle radius coeff=0.75, chemfig style={line width=0.1em}, }
\renewcommand{\CancelColor}{\color{red}}
\usepackage{circuitikz}
\usepackage{mol2chemfig}
\usepackage{subfig,caption}
\captionsetup{font=small, labelfont={bf,sf}}
\usepackage{wrapfig,qrcode}
\usepackage{array,longtable} % ajust colunm table
\newcolumntype{J}{>{\centering\arraybackslash}m{7.5cm}}
\newcolumntype{K}{>{\centering\arraybackslash}m{6.5cm}}
\newcolumntype{L}{>{\centering\arraybackslash}m{5cm}}
\newcolumntype{B}{>{\centering\arraybackslash}m{2.5cm}}
\newcolumntype{N}{>{\centering\arraybackslash}m{1.4cm}}
\usepackage[most]{tcolorbox}
\newcounter{mycounter}
%%% Colobor
%%% Example colorbox
\newtcolorbox{Box2}[2][]{
lower separated=false,
colback=white,
colframe=black,fonttitle=\bfseries,
colbacktitle=black,
coltitle=white,
enhanced, attach boxed title to top left={yshift=-0.1in,xshift=0.15in}, boxed title style={boxrule=0pt,colframe=white,}, title=#2,#1}
%%%%%%%% Cabecalho
\usepackage{framed,amsmath}
\newtcolorbox{mybox}[2][]{
enhanced,title=#2, fonttitle=\sffamily\small,
top=2pt,
bottom=1mm,
boxrule=0.4pt,
coltitle=black,
colback=white,
attach boxed title to top center={yshift=-\tcboxedtitleheight/2,
yshifttext=-\tcboxedtitleheight/2},
boxed title style={
colframe=white,
colback=white,
left=0.2pt,
right=0.2pt},
#1}
\usepackage{tabularray}
%%%%%%
\newtcolorbox{exercisebox}%
{enhanced,breakable,colback=white, colframe=green!15!white,colbacktitle=white!15!pink, coltitle=pink!50!black,left=0pt,right=0mm,top=3mm,bottom=3mm,pad at break=0pt,bottomrule at break=0pt,toprule at break=0pt,borderline={0mm}{0mm}{green!50!white,dashed}, attach boxed title to top center={yshift=-2mm},boxed title style={boxrule=0.4pt},title=Exercícios,}
\usepackage{eso-pic}
\usepackage{etoolbox}
\usepackage{enumitem}
\newcommand\circitem[1]{%
\tikz[baseline=(char.base)]{%https://tex.stackexchange.com/questions/204116/uniform-size-of-circles-around-enumitems
\node[circle,draw=gray, fill=gray!30,
minimum size=1.2em,inner sep=0] (char) {#1};}}
\newcommand\boxitem[1]{%
\tikz[baseline=(char.base)]{%https://tex.stackexchange.com/questions/204116/uniform-size-of-circles-around-enumitems
\node[fill=orange!30,
minimum size=1.2em,inner sep=0] (char) {#1};}}
%\usepackage{widetext}% needs packages "flushend" & "cuted" of "sttools" % bundle, which perhaps must separately be installed
\newcommand{\dd}[1]{\hspace{2pt}d#1}
\definecolor{color1}{RGB}{0,0,90} % Color of the article title and sections
\definecolor{color2}{RGB}{0,20,20} % Color of the boxes behind the abstract and
\definecolor{cinza}{HTML}{C0C0C0}
%%% Custom Exercios
\usepackage{bohr}
\usepackage{multicol}
\setlength{\columnsep}{1.5cm}
\setlength{\columnseprule}{0.2pt}
\usepackage[no-files]{xsim}
\usepackage{tasks}
\xsimsetup{
goal-print={\pgfmathprintnumber[fixed zerofill,precision=1]{#1}}
}
\newcommand*\circled[2]{\tikz[baseline=(char.base)]{
\node[shape=circle,fill,inner sep=2pt, text=white] (char) {#1};}}
%%%%%-Custom Xsim exercises %%%%%
\DeclareExerciseEnvironmentTemplate{custom}
{%\item[\GetExerciseProperty{counter}]
\Needspace*{0\baselineskip}
\noindent
\circled{\XSIMmixedcase{\GetExerciseProperty{counter}}}~~~%
\noindent
\IfInsideSolutionF{%
\GetExercisePropertyT{points}{ % notice the space
(%
\printgoal{\PropertyValue}
\IfExerciseGoalSingularTF{points}
{%\XSIMtranslate{point}
}
{% \XSIMtranslate{points}
}%
)%
}
}}
{\vspace{\baselineskip}}
%%%%%------- Custom  resposta -------%%%%%%%
\DeclareExerciseEnvironmentTemplate{space}
%{\textbf{\GetExerciseProperty{counter}} }
{\noindent\circled{\XSIMmixedcase{\GetExerciseProperty{counter}}}~~~}
% {\circled{\XSIMmixedcase{\GetExerciseProperty{counter}}}}~~~%
{\qquad}
\newcommand*\answer[1]{%
\XSIMexpandcode{%
\SetExerciseProperty{solution-body}
{\noexpand{\Alph{task}}}}%
#1%
}
%\sisetup{locale=DE}
\xsimsetup{
collect = true,
exercise/within = section,
exercise/template = custom,
exercise/the-counter =  \arabic{exercise},
solution/template= custom ,
%%solution-name = solution,  % used with headings=true
%solution/print=true,
%print-collection/print=both,
%print-solutions/collection=true
%goal-print= {\pgfmathprintnumber[fixed zerofill,precision=1]\num{#1}}
}
\RenewDocumentCommand\printpoints{}{%
\TotalExerciseTypeGoal{exercise}{points}{}{}%
}
\NewTasksEnvironment[label = (\emph{\alph*}), label-width = 12pt]{choice}[\choice]
\newenvironment{questions}{\itemize}{\enditemize}
\everymath{\displaystyle}
\DeclareExerciseHeadingTemplate{solution}{%
\section*{Gabarito}%
}
%\usepackage{filecontents}
\NewTblrTheme{fancy}{
\SetTblrStyle{firsthead}{font=\bfseries}
\SetTblrStyle{firstfoot}{fg=blue2}
\SetTblrStyle{middlefoot}{\itshape}
\SetTblrStyle{caption-tag}{red2}
}

\newcommand{\lh}{\underline{\hspace{1cm}}}
%%\onehalfspacing
\def\professor{Fábio Lima}
\def\aluno{ }
\def\numerochamada{}
\def\disciplina{Química}
%%\def\disciplina{UC III}
%%\def\disciplina{R.A.}
\def\turma{}
\def\tipo{{\bfseries Avaliação Bimestral}}
%%\def\tipo{\bfseries Avaliação Mensal}
\def\tipo{\bfseries Exame Final}
\def\bimestre{4 Bimestre}
\def\escola{E.E. 26 de Agosto}
%\def\escola{E.E. José Mamede de Aquino}
%\def\escola{E.E. Joaquim Murtinho}
\def\dataprova{}
\DeclareExerciseCollection{ListaRadioatividade}
\DeclareExerciseCollection{ListaCineticaRadioativa}
\DeclareExerciseCollection{ListaReacaoNuclear}
\date{\today}
\title{}
\hypersetup{
 pdfauthor={},
 pdftitle={},
 pdfkeywords={},
 pdfsubject={},
 pdfcreator={Emacs 29.4 (Org mode 9.6.15)}, 
 pdflang={English}}
\begin{document}

\selectlanguage{portuguese}
\twocolumn[
\input{../Modelos/CabeOficial}
%\input{../Modelos/cabenovo}
%\input{../Modelos/mamede}
%\input{../Modelos/26agosto}
%% \input{../Modelos/geral}
%Cada questão vale {\textbf 2,0}

%%\section*{Regime de Progressão Parcial}
%\section*{Atividade}
%\section*{Trabalho}
%%\section*{\disciplina}

%{\bfseries Obrigatório a resolução das questões }

%\input{../Modelos/gabarito}

%%Total Prova: \printpoints
\smallbreak
\medbreak
\par\vspace{2ex}]%%%%\input{../Modelos/mamede}


\collectexercises{ListaRadioatividade}



\begin{exercise}
Calcule o número de partículas \(\upalpha\)  e \(\upbeta\) que o urânio 238 (\isotope{238,U}), precisa emitir para se transformar em Rádio 226 (\isotope{226,Ra}).
\end{exercise}


\begin{exercise}
A partir de um átomo radioativo, chega-se ao elemento \isotope{212,Bi} por meio de quatro emissões \(\upalpha\) e três emissões \(\upbeta\) Identifique o elemento inicial e diga qual é o seu número de massa e o seu número atômico ?
\end{exercise}


\begin{exercise}
Após algumas desintegrações sucessivas, o \isotope{232,Th}, muito encontrado na orla marítima de Guarapari (ES), se transforma no \isotope{208,Pb}. Qual o número de partículas \(\upalpha\) e \(\upbeta\) emitidas nessa tranformação ?
\end{exercise}


\begin{exercise}
Um elemento X emite 8 partículas betas resultando no elemento Y. Y por sua vez emite 4 partículas alfas resultando em Z. Diga quem são, ao final do processo, isótopo ou isóbaro ?
\end{exercise}


\begin{exercise}
A partir de um átomo radioativo, chega-se ao elemento \isotope{220,Rn}, por meio de três emissões \(\upalpha\) e duas emissões \(\upbeta\). Identifique o elemento inicial e diga qual é o seu número de massa e o seu número atômico ?
\end{exercise}


\collectexercisesstop{ListaRadioatividade}


\collectexercises{ListaCineticaRadioativa}

\begin{exercise}
A desintegração de 1 g de molibdênio 99, até restarem 0,125 g de molibdênio, dura 180 horas. Qual a sua meia vida ?
\end{exercise}



\begin{exercise}
Após 40 dias um determinado radioisótopo, cuja meia-vida é de 10 dias, pesa 0,5 gramas. Qual a sua massa inicial?
\end{exercise}



\begin{exercise}
A meia-vida de um elemento radioativo é 15 minutos. Partindo-se de 320 mg desse elemento, após uma hora e meia sua massa fica reduzida a A mg. Qual o valor de A ?
\end{exercise}


\begin{exercise}
Quanto tempo levará para reste apenas 6,25\% de uma amostra de um material radioativo, cuja meia vida é de 5 minutos ?
\end{exercise}


\begin{exercise}
Sabendo que a meia-vida do estrôncio 90 é de aproximadamente 28 anos, determine a porcentagem do mesmo que ainda estará presente daqui a 112 anos
\end{exercise}



\begin{exercise}
Sabendo que o cobalto 60 perde metade de sua radioatividade a cada 5 anos, aproximadamente, qual a porcentagem residual de sua radioatividade após 15 anos ?
\end{exercise}


\begin{exercise}
Após 3 horas, o nível de atividade de uma amostra de um determinado isótopo radioativo, que decai em um isótopo estável, caiu para 20\% de seu valor inicial. Calcule a meia-vida de este isótopo
\vspace{2cm}
\end{exercise}


\begin{exercise}
O flúor-20 decai para néon-20 com meia-vida de 11 segundos. Em t = 0, uma amostra contém 120 gramas de átomos de flúor-20. Quantos gramas de átomos de flúor-20 permanecem na amostra em 

\begin{choice}
\choice  t = 5,0 segundos, 
\vspace{1.5cm}
\choice  t = 30 segundos 
\vspace{1.5cm}
\choice  t = 1,0 minutos?
\vspace{1.5cm}
\end{choice}
\end{exercise}


\collectexercisesstop{ListaCineticaRadioativa}






\collectexercises{ListaReacaoNuclear}

\begin{exercise}[points=3.5]
Escreva as equações

\begin{choice}(2)
\choice Emissão pósiton  \isotope{18,F} \vspace{2cm}
\choice  Emissão \(\upbeta\) \isotope{20,Ca} \vspace{2cm}
\choice Emissão alfa	\isotope{235,U} \vspace{2cm}
\choice  Emissão gama \vspace{2cm}
\end{choice}
\end{exercise}


\collectexercisesstop{ListaReacaoNuclear}

\section{Lei da Radioatividade}
\label{sec:orgf99036b}

\printcollection{ListaRadioatividade}


\section{Cinética de Desintegrações}
\label{sec:org362ba66}


\printcollection{ListaCineticaRadioativa}


\section{Reação Nuclear}
\label{sec:org4bbaf73}


\printcollection{ListaReacaoNuclear}
\end{document}
