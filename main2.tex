% Created 2025-02-23 dom 15:23
% Intended LaTeX compiler: lualatex
\documentclass[12pt]{scrartcl}


\KOMAoptions{
%headings=chapterprefix,
%%twocolumn=false,
%toc=indenttextentries,
%toc=flat,
twoside=true,
headinclude=true,
footinclude=true
%  captions=topbeside
}
%\usepackage[fontsize=12.3]{scrextend}
\usepackage{fontspec}
\usepackage[T1]{fontenc}
\usepackage{hyperref}
\usepackage[x11names,svgnames,table]{xcolor}
\defaultfontfeatures{Ligatures=TeX}
\setmainfont{Lato}
\usepackage{typearea}
\usepackage{lscape}
\usepackage[a4paper]{geometry}
\geometry{a4paper,total={170mm,257mm},left=30mm,right=30mm, top=15mm, bottom=20mm}
\usepackage[english, portuguese, american]{babel}
\usepackage{amsmath,amsfonts,amsthm,bm}
\usepackage{graphicx}
\usepackage{float,wrapfig}
\usepackage{colortbl}
\usepackage{tabularx}
\usepackage{pst-labo}
\usepackage{setspace}
\usepackage{xfrac}
\usepackage{tikz}
\usepackage{pgfplots}
\pgfplotsset{compat=1.3}
%% Diagraman latex
\usepackage{endiagram}
\usepackage{smartdiagram}
\usepackage[tikz]{bclogo}
\usetikzlibrary{fit,patterns,shadows.blur,shapes,decorations.pathreplacing,decorations.markings,arrows.meta,arrows,positioning,shadows,trees}
\usetikzlibrary{decorations.pathmorphing} %% to chemfig config bond
\usepackage{upgreek}
\usepackage{chemmacros}
%%\chemsetup{modules={reactions,spectroscopy,thermodynamics,redox,isotopes}}
\chemsetup{modules={all}}
%\usepackage[version=4,arrows=pgf-filled]{mhchem}
\usepackage{chemfig,elements,cancel,siunitx}
\NewChemPhase\lqdd{\(\ell\)}
\NewChemPhase\gr{grafite}
\NewChemPhase\reac{reação}
\setchemfig{
angle increment=30,
atom sep=1.67em,
double bond sep=0.67ex,
bond style={line width=0.1em},
cram width=0.8ex,
cram dash width=0.1em,
cram dash sep=0.2em,
arrow style={line width=0.067em},
arrow head=-{Triangle},
arrow label sep=1ex,
cycle radius coeff=0.75,
chemfig style={line width=0.1em},
}
%\setchemfig{fixed length=true, atom sep=2.0em, arrow offset=6pt, scheme debug=false,angle increment=30}
\renewcommand{\CancelColor}{\color{red}}
\usepackage{circuitikz}
\usepackage{mol2chemfig}
\usepackage{subfig,caption}
\captionsetup{font=small, labelfont={bf,sf}}
\usepackage{wrapfig,qrcode}
\usepackage{array,longtable} % ajust colunm table
\newcolumntype{J}{>{\centering\arraybackslash}m{7.5cm}}
\newcolumntype{K}{>{\centering\arraybackslash}m{6.5cm}}
\newcolumntype{L}{>{\centering\arraybackslash}m{5cm}}
\newcolumntype{B}{>{\centering\arraybackslash}m{2.5cm}}
\newcolumntype{N}{>{\centering\arraybackslash}m{1.4cm}}
\usepackage[most]{tcolorbox}
\newcounter{mycounter}
%%% Colobor
%%% Example colorbox
\newtcolorbox{Box2}[2][]{
lower separated=false,
colback=white,
colframe=black,fonttitle=\bfseries,
colbacktitle=black,
coltitle=white,
enhanced, attach boxed title to top left={yshift=-0.1in,xshift=0.15in}, boxed title style={boxrule=0pt,colframe=white,}, title=#2,#1}
%%%%%%%% Cabecalho
\usepackage{framed,amsmath}
\newtcolorbox{mybox}[2][]{
enhanced,title=#2, fonttitle=\sffamily\small,
top=2pt,
bottom=1mm,
boxrule=0.4pt,
coltitle=black,
colback=white,
attach boxed title to top center={yshift=-\tcboxedtitleheight/2,
yshifttext=-\tcboxedtitleheight/2},
boxed title style={
colframe=white,
colback=white,
left=0.2pt,
right=0.2pt},
#1}
\usepackage{tabularray}
%%%%%%
\newtcolorbox{exercisebox}%
{enhanced,breakable,colback=white, colframe=green!15!white,colbacktitle=white!15!pink, coltitle=pink!50!black,left=0pt,right=0mm,top=3mm,bottom=3mm,pad at break=0pt,bottomrule at break=0pt,toprule at break=0pt,borderline={0mm}{0mm}{green!50!white,dashed}, attach boxed title to top center={yshift=-2mm},boxed title style={boxrule=0.4pt},title=Exercícios,}
\usepackage{eso-pic}
\usepackage{etoolbox}
\usepackage{enumitem}
\newcommand\circitem[1]{%
\tikz[baseline=(char.base)]{%https://tex.stackexchange.com/questions/204116/uniform-size-of-circles-around-enumitems
\node[circle,draw=gray, fill=gray!30,
minimum size=1.2em,inner sep=0] (char) {#1};}}
\newcommand\boxitem[1]{%
\tikz[baseline=(char.base)]{%https://tex.stackexchange.com/questions/204116/uniform-size-of-circles-around-enumitems
\node[fill=orange!30,
minimum size=1.2em,inner sep=0] (char) {#1};}}
%\usepackage{widetext}% needs packages "flushend" & "cuted" of "sttools" % bundle, which perhaps must separately be installed
\newcommand{\dd}[1]{\hspace{2pt}d#1}
\definecolor{color1}{RGB}{0,0,90} % Color of the article title and sections
\definecolor{color2}{RGB}{0,20,20} % Color of the boxes behind the abstract and
\definecolor{cinza}{HTML}{C0C0C0}
%%% Custom Exercios
\usepackage{bohr}
\usepackage{multicol}
\setlength{\columnsep}{1.5cm}
\setlength{\columnseprule}{0.2pt}
\usepackage[no-files]{xsim}
\usepackage{tasks}
\xsimsetup{
goal-print={\pgfmathprintnumber[fixed zerofill,precision=1]{#1}}
}
\newcommand*\circled[2]{\tikz[baseline=(char.base)]{
\node[shape=circle,fill,inner sep=2pt, text=white] (char) {#1};}}
%%%%%-Custom Xsim exercises %%%%%
\DeclareExerciseEnvironmentTemplate{custom}
{%\item[\GetExerciseProperty{counter}]
\Needspace*{0\baselineskip}
\noindent
\circled{\XSIMmixedcase{\GetExerciseProperty{counter}}}~~~%
\noindent
\IfInsideSolutionF{%
\GetExercisePropertyT{points}{ % notice the space
(%
\printgoal{\PropertyValue}
\IfExerciseGoalSingularTF{points}
{%\XSIMtranslate{point}
}
{% \XSIMtranslate{points}
}%
)%
}
}}
{\vspace{\baselineskip}}

%%%%%------- Custom  resposta -------%%%%%%%
\DeclareExerciseEnvironmentTemplate{space}
%{\textbf{\GetExerciseProperty{counter}} }
{\noindent\circled{\XSIMmixedcase{\GetExerciseProperty{counter}}}~~~}
% {\circled{\XSIMmixedcase{\GetExerciseProperty{counter}}}}~~~%
{\qquad}
\newcommand*\answer[1]{%
\XSIMexpandcode{%
\SetExerciseProperty{solution-body}
{\noexpand{\Alph{task}}}}%
#1%
}
%\sisetup{locale=DE}
\xsimsetup{
collect = true,
exercise/within = section, %%% reset number xsim in
exercise/template = custom,
exercise/the-counter =  \arabic{exercise},
solution/template=custom,
%solution/print=true,
%print-collection/print=both,
%goal-print= {\pgfmathprintnumber[fixed zerofill,precision=1]\num{#1}}
}
\RenewDocumentCommand\printpoints{}{%
\TotalExerciseTypeGoal{exercise}{points}{}{}%
}
\NewTasksEnvironment[label = (\emph{\alph*}), label-width = 12pt]{choice}[\choice]
\newenvironment{questions}{\itemize}{\enditemize}
\DeclareExerciseHeadingTemplate{solution}{%
\section*{Gabarito}%
}
\everymath{\displaystyle}
%\usepackage{filecontents}
\usepackage{lineno}
\newcommand{\lh}{\underline{\hspace{1cm}}}
%%\onehalfspacing
\def\professor{Fábio Lima}
\def\aluno{}
\def\numerochamada{}
\def\disciplina{Química}
%%\def\disciplina{UCIII}
%%\def\disciplina{R.A.}
\def\turma{3 Ano }
%%\def\tipo{{\bfseries Avaliação Bimestral}}
\def\tipo{\bfseries RECUPERAÇÃO }
%%\def\tipo{\bfseries Exame Final}
\def\bimestre{2 Bimestre}
\def\escola{E.E. 26 de Agosto}
%%\def\escola{E.E. José Mamede de Aquino}
%\def\escola{E.E. Amelio Carvalho de Bais}
%5\def\escola{}
\def\dataprova{}
\DeclareExerciseCollection{ListaBalanceamentoIII}
\setcounter{secnumdepth}{0}
\author{fabio}
\date{\today}
\title{}
\hypersetup{
 pdfauthor={fabio},
 pdftitle={},
 pdfkeywords={},
 pdfsubject={},
 pdfcreator={Emacs 29.4 (Org mode 9.6.15)}, 
 pdflang={English}}
\begin{document}

\pagebreak 
%%\input{../Modelos/geral}
%%\input{../Modelos/cabenovo}
%%\input{../Modelos/26agosto}
\input{../Modelos/CabeOficial}
%\section*{Atividade de Química}
%\begin{itemize}
%\item Realizar a atividade de recuperação em dupla
%\item Atividade ira auxiliar na nota da Avaliação
%\end{itemize}
\smallbreak
\medbreak
%\vspace{-9cm}




\collectexercises{ListaBalanceamentoIII}

Balanceie as reações a seguir
\vspace{0.5cm}
\begin{exercise}
 \ch{\lh H3PO4 \; + \lh KOH -> \lh K3PO4 \; + \lh H2O} 
\end{exercise}

\begin{exercise}
 \ch{\lh K \; + \lh B2O3 -> \lh K2O \; + \lh  B} 
\end{exercise}

\begin{exercise}
 \ch{\lh HC$\ell$ + \lh NaOH ->  \lh NaC$\ell$ + \lh H2O}
\end{exercise}

\begin{exercise}
 \ch{\lh Na \; + \lh NaNO3 -> \lh Na2O \; + \lh N2}
\end{exercise}

\begin{exercise}
 \ch{\lh C \; +  \lh S8 -> \lh CS2} 
\end{exercise}

\begin{exercise}
 \ch{\lh Na \; + \lh O2 -> \lh Na2O} 
\end{exercise}

\begin{exercise}
 \ch{\lh N2 \; + \lh  O2 -> \lh N2O5}
\end{exercise}

\begin{exercise}
 \ch{\lh H3PO4 \; + \lh Mg(OH)2 -> \lh Mg3(PO4 )2\; + \lh H2O}
\end{exercise}


\begin{exercise}
 \ch{\lh NaOH \; + \lh H2CO3 -> \lh Na2CO3 + \lh H2O}
\end{exercise}


\begin{exercise}
 \ch{\lh KOH \; + \lh HBr ->  \lh KBr \; + \lh  H2O}
\end{exercise}



\begin{exercise}
 \ch{\lh Na \; + \lh O2 -> \lh Na2O} 
\end{exercise}



\begin{exercise}
 \ch{\lh A$\ell$(OH)3 \; + \lh  H2CO3 -> \lh A$\ell$2(CO3)3 + \lh H2O}  
\end{exercise}



\begin{exercise}
 \ch{\lh A$\ell$ \; + \lh S8 -> \lh A$\ell$2S3}
\end{exercise}



\begin{exercise}
 \ch{\lh Cs \; + \lh N2 -> \lh Cs3N} 
\end{exercise}

\begin{exercise}
 \ch{ \lh Mg \; + \lh C$\ell$2 -> \lh MgC$\ell$2} 
\end{exercise}

\begin{exercise}
 \ch{\lh Rb \; + \lh RbNO3 -> \lh  Rb2O \; + \lh N2} 
\end{exercise}


\begin{exercise}
 \ch{\lh C6H6\; + \lh O2 ->  \lh CO2\; + \lh H2O} 
\end{exercise}



\begin{exercise}
 \ch{\lh N2\; + \lh H2 -> \lh NH3}  
\end{exercise}



\begin{exercise}
 \ch{\lh C10H22 \ + \lh  O2 -> \lh CO2\ + \lh H2O}
\end{exercise}

\begin{exercise}
 \ch{\lh A$\ell$(OH)3\; + \lh  HBr -> \lh A$\ell$Br3 \; + \lh H2O}
\end{exercise}


\begin{exercise}
 \ch{\lh CH3CH2CH2CH3\; + \lh O2 -> \lh CO2 \; + \lh H2O} 
\end{exercise}

\begin{exercise}
\ch{\lh C3H8 \; + \lh O2 -> \lh CO2 \; + \lh H2O}
\end{exercise}

\begin{exercise}
\ch{\lh Li + \lh  A$\ell$C$\ell$3 -> \lh  LiC$\ell$ + \lh A$\ell$}
\end{exercise}


\begin{exercise}
\ch{\lh C2H6 \; + \lh O2 -> \lh  CO2 \; + \lh H2O}
\end{exercise}

\begin{exercise}
\ch{\lh  NH4OH \; + \lh H3PO4 -> \lh (NH4)3PO4 \; + \lh  H2O}
\end{exercise}

\begin{exercise}
\ch{\lh Rb\; + \lh P ->  \lh Rb3P}
\end{exercise}


\begin{exercise}
\ch{\lh CH4 \; + \lh O2 -> \lh  CO2 \; + \lh  H2O}
\end{exercise}

\begin{exercise}
\ch{\lh  A$\ell$(OH)3 \; + \lh H2SO4 -> \lh A$\ell$2(SO4)3\; + \lh H2O}
\end{exercise}

\begin{exercise}
\ch{\lh Na\; + \lh C$\ell$2 -> \lh NaC$\ell$}
\end{exercise}

\begin{exercise}
\ch{\lh Rb \; + \lh S8 -> \lh Rb2S}
\end{exercise}


\begin{exercise}
\ch{\lh H3PO4\; + \lh Ca(OH)2 -> \lh Ca3(PO4)2\; + \lh  H2O}
\end{exercise}


\begin{exercise}
\ch{\lh NH3  \;  + \lh HC$\ell$ -> \lh  NH4C$\ell$}
\end{exercise}

\begin{exercise}
\ch{\lh  Li\; + \lh  H2O ->  \lh  LiOH \; + \lh H2}
\end{exercise}


\begin{exercise}
\ch{\lh Ca3(PO4)2 \; + \lh  SiO2 \; + \lh  C ->  \lh CaSiO3 \; + \lh  CO \; + \lh  P}
\end{exercise}

\begin{exercise}
\ch{\lh NH3 \; + \lh  O2 -> \lh  N2 \; + \lh H2O}
\end{exercise}

\begin{exercise}
\ch{\lh  FeS2 \; + \lh  O2 ->  \lh  Fe2O3 \; + \lh SO2}
\end{exercise}

\begin{exercise}
\ch{ \lh  C \; + \lh  SO2 ->  \lh  CS2 + \lh + \lh CO}
\end{exercise}

\begin{exercise}
\ch{\lh S8 \; + \lh Br2 -> \lh SBr2}
\end{exercise}



\begin{exercise}
\ch{\lh  S8 \; + \lh NO2 -> \lh  SO2 \;  + \lh N2}
\end{exercise}



\begin{exercise}
\ch{\lh S8 \; + \lh NO3 -> \lh  SO2 \; + \lh NO}
\end{exercise}

\begin{exercise}
\ch{\lh  C3H8 \; + \lh  O2 -> \lh  CO2 \; + \lh  H2O}
\end{exercise}

\begin{exercise}
\ch{\lh C7H14 \; + \lh O2 ->  \lh CO  \; + \lh H2}
\end{exercise}

\begin{exercise}
\ch{\lh C6H6 \; + \lh HNO3 -> \lh C6H5NO2\; + \lh H2O}
\end{exercise}


\begin{exercise}
\ch{\lh C3H4 \; + \lh I2 -> \lh C3H4I2}
\end{exercise}


\begin{exercise}
\ch{\lh CO2 \; + \lh  C$\ell$2 -> \lh CC$\ell$4 \; + \lh  O2 }
\end{exercise}

\begin{exercise}
\ch{\lh  S7 \; + \lh P2O5\; + \lh  O2 -> \lh SO3 \; + \lh P4}
\end{exercise}

\begin{exercise}
\ch{\lh N2 \; + \lh C2H6 -> \lh N2H4 \; + \lh C2H2}
\end{exercise}

\begin{exercise}
\ch{\lh C5H10 \; + \lh O2 -> \lh CH2O}
\end{exercise}


\begin{exercise}
\ch{\lh C6H12O6 \; + \lh F2 -> \lh C6H6F6 \; + \lh H2O \; + \lh O2}
\end{exercise}



\begin{exercise}
\ch{\lh  NaOH \; + \lh  H2SO4 -> \lh  Na2SO4\; + \lh  H2O}
\end{exercise}


\begin{exercise}
\ch{\lh C6O6Cr\; + \lh C$\ell$2 ->  \lh CrC$\ell$3 \; + \lh CO}
\end{exercise}


\begin{exercise}
\ch{\lh P4 \; + \lh HC$\ell$ \; + \lh  O2 -> \lh  PC$\ell$3\; + \lh H2O}
\end{exercise}


\begin{exercise}
\ch{\lh H3PO4\; + \lh  C  -> \lh P4\; + \lh  CO\; + \lh H2O}
\end{exercise}

\begin{exercise}
\ch{\lh Na + \lh C2C$\ell$6 ->\lh  NaC$\ell$ + \lh C2C$\ell$2}
\end{exercise}

\begin{exercise}
\ch{\lh NOC$\ell$ \; + \lh WC6O6 -> \lh WN2O2C$\ell$2 \; +  \lh CO}
\end{exercise}

\begin{exercise}
\ch{\lh NH3 \; + \lh CO -> \lh CH4 \; + \lh  N2\; + \lh O2}
\end{exercise}

\begin{exercise}
\ch{\lh PC$\ell$3 \; +  \lh H2O -> \lh   H3PO3 \; + \lh HC$\ell$}
\end{exercise}



\collectexercisesstop{ListaBalanceamentoIII}

\printcollection{ListaBalanceamentoIII}
\end{document}
