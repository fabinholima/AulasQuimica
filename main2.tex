% Created 2025-04-07 seg 13:09
% Intended LaTeX compiler: lualatex
\documentclass[12pt]{scrartcl}


\KOMAoptions{
%headings=chapterprefix,
%%twocolumn=false,
%toc=indenttextentries,
%toc=flat,
twoside=true,
headinclude=true,
footinclude=true
%  captions=topbeside
}
%\usepackage[fontsize=12.3]{scrextend}
\usepackage{fontspec}
\usepackage[T1]{fontenc}
\usepackage{hyperref}
\usepackage[x11names,svgnames,table]{xcolor}
\defaultfontfeatures{Ligatures=TeX}
\setmainfont{Lato}
\usepackage{typearea}
\usepackage{lscape}
\usepackage[a4paper]{geometry}
\geometry{a4paper,total={170mm,257mm},left=30mm,right=30mm, top=15mm, bottom=20mm}
\usepackage[english, portuguese]{babel}
\usepackage{amsmath,amsfonts,amsthm,bm}
\usepackage{graphicx}
\usepackage{float,wrapfig}
\usepackage{colortbl}
\usepackage{tabularx}
\usepackage{pst-labo}
\usepackage{setspace}
\usepackage{xfrac}
\usepackage{tikz}
\usepackage{pgfplots}
\pgfplotsset{compat=1.3}
%% Diagraman latex
\usepackage{endiagram}
\usepackage{smartdiagram}
\usepackage[tikz]{bclogo}
\usetikzlibrary{fit,patterns,shadows.blur,shapes,decorations.pathreplacing,decorations.markings,arrows.meta,arrows,positioning,shadows,trees}
\usetikzlibrary{decorations.pathmorphing} %% to chemfig config bond
\usepackage{upgreek}
\usepackage{chemmacros}
%%\chemsetup{modules={reactions,spectroscopy,thermodynamics,redox,isotopes}}
\chemsetup{modules={all}}
%\usepackage[version=4,arrows=pgf-filled]{mhchem}
\usepackage{chemfig,elements,cancel,siunitx}
\NewChemPhase\lqdd{\(\ell\)}
\NewChemPhase\gr{grafite}
\NewChemPhase\reac{reação}
\setchemfig{
angle increment=30,
atom sep=1.67em,
double bond sep=0.67ex,
bond style={line width=0.1em},
cram width=0.8ex,
cram dash width=0.1em,
cram dash sep=0.2em,
arrow style={line width=0.067em},
arrow head=-{Triangle},
arrow label sep=1ex,
cycle radius coeff=0.75,
chemfig style={line width=0.1em},
}
%\setchemfig{fixed length=true, atom sep=2.0em, arrow offset=6pt, scheme debug=false,angle increment=30}
\renewcommand{\CancelColor}{\color{red}}
\usepackage{circuitikz}
\usepackage{mol2chemfig}
\usepackage{subfig,caption}
\captionsetup{font=small, labelfont={bf,sf}}
\usepackage{wrapfig,qrcode}
\usepackage{array,longtable} % ajust colunm table
\newcolumntype{J}{>{\centering\arraybackslash}m{7.5cm}}
\newcolumntype{K}{>{\centering\arraybackslash}m{6.5cm}}
\newcolumntype{L}{>{\centering\arraybackslash}m{5cm}}
\newcolumntype{B}{>{\centering\arraybackslash}m{2.5cm}}
\newcolumntype{N}{>{\centering\arraybackslash}m{1.4cm}}
\usepackage[most]{tcolorbox}
\newcounter{mycounter}
%%% Colobor
%%% Example colorbox
\newtcolorbox{Box2}[2][]{
lower separated=false,
colback=white,
colframe=black,fonttitle=\bfseries,
colbacktitle=black,
coltitle=white,
enhanced, attach boxed title to top left={yshift=-0.1in,xshift=0.15in}, boxed title style={boxrule=0pt,colframe=white,}, title=#2,#1}
%%%%%%%% Cabecalho
\usepackage{framed,amsmath}
\newtcolorbox{mybox}[2][]{
enhanced,title=#2, fonttitle=\sffamily\small,
top=2pt,
bottom=1mm,
boxrule=0.4pt,
coltitle=black,
colback=white,
attach boxed title to top center={yshift=-\tcboxedtitleheight/2,
yshifttext=-\tcboxedtitleheight/2},
boxed title style={
colframe=white,
colback=white,
left=0.2pt,
right=0.2pt},
#1}
\usepackage{tabularray}
%%%%%%
\newtcolorbox{exercisebox}%
{enhanced,breakable,colback=white, colframe=green!15!white,colbacktitle=white!15!pink, coltitle=pink!50!black,left=0pt,right=0mm,top=3mm,bottom=3mm,pad at break=0pt,bottomrule at break=0pt,toprule at break=0pt,borderline={0mm}{0mm}{green!50!white,dashed}, attach boxed title to top center={yshift=-2mm},boxed title style={boxrule=0.4pt},title=Exercícios,}
\usepackage{eso-pic}
\usepackage{etoolbox}
\usepackage{enumitem}
\newcommand\circitem[1]{%
\tikz[baseline=(char.base)]{%https://tex.stackexchange.com/questions/204116/uniform-size-of-circles-around-enumitems
\node[circle,draw=gray, fill=gray!30,
minimum size=1.2em,inner sep=0] (char) {#1};}}
\newcommand\boxitem[1]{%
\tikz[baseline=(char.base)]{%https://tex.stackexchange.com/questions/204116/uniform-size-of-circles-around-enumitems
\node[fill=orange!30,
minimum size=1.2em,inner sep=0] (char) {#1};}}
%\usepackage{widetext}% needs packages "flushend" & "cuted" of "sttools" % bundle, which perhaps must separately be installed
\newcommand{\dd}[1]{\hspace{2pt}d#1}
\definecolor{color1}{RGB}{0,0,90} % Color of the article title and sections
\definecolor{color2}{RGB}{0,20,20} % Color of the boxes behind the abstract and
\definecolor{cinza}{HTML}{C0C0C0}
%%% Custom Exercios
\usepackage{bohr}
\usepackage{multicol}
\setlength{\columnsep}{1.5cm}
\setlength{\columnseprule}{0.2pt}
\usepackage[no-files]{xsim}
\usepackage{tasks}
\xsimsetup{
goal-print={\pgfmathprintnumber[fixed zerofill,precision=1]{#1}}
}
\newcommand*\circled[2]{\tikz[baseline=(char.base)]{
\node[shape=circle,fill,inner sep=2pt, text=white] (char) {#1};}}
%%%%%-Custom Xsim exercises %%%%%
\DeclareExerciseEnvironmentTemplate{custom}
{%\item[\GetExerciseProperty{counter}]
\Needspace*{0\baselineskip}
\noindent
\circled{\XSIMmixedcase{\GetExerciseProperty{counter}}}~~~%
\noindent
\IfInsideSolutionF{%
\GetExercisePropertyT{points}{ % notice the space
(%
\printgoal{\PropertyValue}
\IfExerciseGoalSingularTF{points}
{%\XSIMtranslate{point}
}
{% \XSIMtranslate{points}
}%
)%
}
}}
{\vspace{\baselineskip}}

%%%%%------- Custom  resposta -------%%%%%%%
\DeclareExerciseEnvironmentTemplate{space}
%{\textbf{\GetExerciseProperty{counter}} }
{\noindent\circled{\XSIMmixedcase{\GetExerciseProperty{counter}}}~~~}
% {\circled{\XSIMmixedcase{\GetExerciseProperty{counter}}}}~~~%
{\qquad}
\newcommand*\answer[1]{%
\XSIMexpandcode{%
\SetExerciseProperty{solution-body}
{\noexpand{\Alph{task}}}}%
#1%
}
%\sisetup{locale=DE}
\xsimsetup{
collect = true,
exercise/within = section, %%% reset number xsim in
exercise/template = custom,
exercise/the-counter =  \arabic{exercise},
solution/template=custom,
%solution/print=true,
%print-collection/print=both,
%goal-print= {\pgfmathprintnumber[fixed zerofill,precision=1]\num{#1}}
}
\RenewDocumentCommand\printpoints{}{%
\TotalExerciseTypeGoal{exercise}{points}{}{}%
}
\NewTasksEnvironment[label = (\emph{\alph*}), label-width = 12pt]{choice}[\choice]
\newenvironment{questions}{\itemize}{\enditemize}
\DeclareExerciseHeadingTemplate{solution}{%
\section*{Gabarito}%
}
\everymath{\displaystyle}
%\usepackage{filecontents}
\usepackage{lineno}
\newcommand{\lh}{\underline{\hspace{1cm}}}
%%\onehalfspacing
\def\professor{Fábio Lima}
\def\aluno{}
\def\numerochamada{}
\def\disciplina{Química}
%%\def\disciplina{UCIII}
%%\def\disciplina{R.A.}
\def\turma{2 Ano }
%%\def\tipo{{\bfseries Avaliação Bimestral}}
\def\tipo{\bfseries Atividade Avaliativa }
%%\def\tipo{\bfseries Exame Final}
\def\bimestre{1 Bimestre}
%%\def\escola{E.E. 26 de Agosto}
\def\escola{E.E. José Mamede de Aquino}
%\def\escola{E.E. Amelio Carvalho de Bais}
%5\def\escola{}
\def\dataprova{}
\DeclareExerciseCollection{ListaLeiHess}
\setcounter{secnumdepth}{0}
\author{fabio}
\date{\today}
\title{}
\hypersetup{
 pdfauthor={fabio},
 pdftitle={},
 pdfkeywords={},
 pdfsubject={},
 pdfcreator={Emacs 30.1 (Org mode 9.7.11)}, 
 pdflang={English}}
\begin{document}

\pagebreak 
%%\input{../Modelos/geral}
%\input{../Modelos/cabenovo}
%%\input{../Modelos/26agosto}
\input{../Modelos/CabeOficial}
%\section*{Atividade de Química}
%\begin{itemize}
%\item Realizar a atividade de recuperação em dupla
%\item Atividade ira auxiliar na nota da Avaliação
%\end{itemize}
\smallbreak
\medbreak
%\vspace{-9cm}





\collectexercises{ListaLeiHess}

\begin{exercise}
Calcule a entalpia de reação para a formação de cloreto de alumínio anidro, usando os dados abaixo:
% 2 CO2_{\gas} + H2O_{\lqdd}  & $\quad \enthalpy[unit=\kilo\joule]{-1300}$ \\
\begin{reactions*}
2 A$\ell$_{\sld}	+	6 HC$\ell$_{\aq} -> 2 A$\ell$C$\ell$3_{\aq}	+	3 H2_{\gas}	& $\quad \enthalpy[unit=\kilo\joule]{-1049}$\\	
HC$\ell$_{\gas} -> HC$\ell$_{\aq} & $\quad \enthalpy[unit=\kilo\joule]{-74}$\\
H2_{\gas}	+ C$\ell$2_{\gas}	->  2 HC$\ell$_{\gas} & $\quad \enthalpy[unit=\kilo\joule]{-185}$\\	
A$\ell$C$\ell$3_{\sld}	->  A$\ell$C$\ell$3_{\aq} & $\quad \enthalpy[unit=\kilo\joule]{-323}$
\end{reactions*}

Calcule o $\Delta$H da reação
\begin{reaction*}
2  A$\ell$_{\sld} + 3 C$\ell$2_{g} -> 2 A$\ell$C$\ell$3_{\sld} 
\end{reaction*}

\blank[blank-style={\phantom{#1}},width=12\linewidth]{}
\end{exercise}



\begin{exercise}
Use a Lei de Hess para calcular o $\Delta$H da reação
\begin{reaction*}
C4H8_{\gas} + 6 O2_{\gas} -> 4 CO2_{\gas} + 4 H2O_{\gas}
\end{reaction*}

A seguir as reações:


\begin{enumerate}[label=\Roman{*}.]
\item \ch{2 H2_{\gas} + O2_{\gas} -> 2 H2O_{\gas}}  $\qquad \enthalpy[unit=\kilo\joule]{-571}$
\item \ch{C4H8_{\gas} + H2_{\gas} -> C4H10_{\gas}} $\qquad \enthalpy[unit=\kilo\joule]{-126}$
\item \ch{2 C4H10_{\gas} + 13 O2_{\gas} -> 8 CO2_{\gas} + 10 H2O_{\gas}} $\qquad \enthalpy[unit=\kilo\joule]{-5754}$
\end{enumerate}


\blank[blank-style={\phantom{#1}},width=12\linewidth]{}
\end{exercise}
\begin{solution}
\end{solution}


\begin{exercise}
A seguir as entalpias de reações:

\begin{enumerate}[label=\Roman{*}.]
\item \ch{H2_{\gas} + F2_{\gas} -> 2 HF_{\gas}} $\qquad \enthalpy[unit=\kilo\joule]{-537}$
\item \ch{C_{\sld} + 2 F2_{\gas} -> CF4_{\gas}} $\qquad \enthalpy[unit=\kilo\joule]{-680}$
\item \ch{C_{\sld} + 2 H2_{\gas} -> C2H4_{\gas}} $\qquad \enthalpy[unit=\kilo\joule]{-52}$
\end{enumerate}


Calcule o $\Delta$H para a reação \textbf{NÃO BALANCEADA} abaixo.
\begin{reaction*}
C2H4_{\gas} +  F2_{\gas} -> CF4_{\gas} +  HF_{\gas}
\end{reaction*}

\blank[blank-style={\phantom{#1}},width=12\linewidth]{}
\end{exercise}
\begin{solution}
∆H = –1949
\end{solution}



\begin{exercise}
O diagrama a seguir 

\begin{center}
\tikzset{every picture/.style={line width=1.0pt}} %set default line width to 0.75pt        

\begin{tikzpicture}[x=0.75pt,y=0.75pt,yscale=-0.8,xscale=0.8]
%uncomment if require: \path (0,376); %set diagram left start at 0, and has height of 376

%Straight Lines [id:da4742405153664734] 
\draw    (108,307) -- (108,7) ;
\draw [shift={(108,5)}, rotate = 90] [color={rgb, 255:red, 0; green, 0; blue, 0 }  ][line width=0.75]    (10.93,-3.29) .. controls (6.95,-1.4) and (3.31,-0.3) .. (0,0) .. controls (3.31,0.3) and (6.95,1.4) .. (10.93,3.29)   ;
%Straight Lines [id:da4987255852646296] 
\draw    (108,307) -- (575,309.99) ;
\draw [shift={(577,310)}, rotate = 180.37] [color={rgb, 255:red, 0; green, 0; blue, 0 }  ][line width=0.75]    (10.93,-3.29) .. controls (6.95,-1.4) and (3.31,-0.3) .. (0,0) .. controls (3.31,0.3) and (6.95,1.4) .. (10.93,3.29)   ;
%Straight Lines [id:da8739968665747467] 
\draw    (138,76) -- (313,76) ;
%Straight Lines [id:da5716536683186184] 
\draw    (302,179) -- (479,181) ;
%Straight Lines [id:da12291859962824414] 
\draw    (142,270) -- (496,272) ;
%Straight Lines [id:da7467539186542762] 
\draw    (310,82) -- (308.04,170) ;
\draw [shift={(308,172)}, rotate = 271.27] [color={rgb, 255:red, 0; green, 0; blue, 0 }  ][line width=0.75]    (10.93,-3.29) .. controls (6.95,-1.4) and (3.31,-0.3) .. (0,0) .. controls (3.31,0.3) and (6.95,1.4) .. (10.93,3.29)   ;
%Straight Lines [id:da7999265194147146] 
\draw    (200,85) -- (200,260) ;
\draw [shift={(200,262)}, rotate = 270] [color={rgb, 255:red, 0; green, 0; blue, 0 }  ][line width=0.75]    (10.93,-3.29) .. controls (6.95,-1.4) and (3.31,-0.3) .. (0,0) .. controls (3.31,0.3) and (6.95,1.4) .. (10.93,3.29)   ;
%Straight Lines [id:da3547253027331094] 
\draw    (371,185) -- (370.03,260) ;
\draw [shift={(370,262)}, rotate = 270.74] [color={rgb, 255:red, 0; green, 0; blue, 0 }  ][line width=0.75]    (10.93,-3.29) .. controls (6.95,-1.4) and (3.31,-0.3) .. (0,0) .. controls (3.31,0.3) and (6.95,1.4) .. (10.93,3.29)   ;

% Text Node
\draw (75.6,108.52) node [anchor=north west][inner sep=0.75pt]  [rotate=-269.87,xscale=0.55,yscale=0.55] [align=left] {\Large Entalpia (KJ)};
% Text Node
\draw (145,44) node [anchor=north west][inner sep=0.75pt]  [xscale=0.55,yscale=0.55] [align=left] {\Large \ch{CH4_{\gas} + 2 O2_{\gas}}};
% Text Node
\draw (365,139) node [anchor=north west][inner sep=0.75pt]  [xscale=0.55,yscale=0.55] [align=left] {\Large \ch{CO_{\gas} + 2 H2O_{\lqdd} + 1/2 O2_{\gas}}};
% Text Node
\draw (414,245) node [anchor=north west][inner sep=0.75pt]  [xscale=0.55,yscale=0.55] [align=left] {\Large\ch{CO2_{\gas} + 2 H2O_{\lqdd}}};
% Text Node
\draw (215,108) node [anchor=north west][inner sep=0.75pt]  [xscale=0.55,yscale=0.55] [align=left] {\Large $\enthalpy[unit=\kilo\joule]{-607}$};
% Text Node
\draw (271,219) node [anchor=north west][inner sep=0.75pt]  [xscale=0.55,yscale=0.55] [align=left] {\Large $\enthalpy[unit=\kilo\joule]{-283}$};
% Text Node
\draw (145,162) node [anchor=north west][inner sep=0.75pt]  [xscale=0.55,yscale=0.55] [align=left] {\Large $\Delta$H$_r$ = ?};
% Text Node
\draw (393,318) node [anchor=north west][inner sep=0.75pt]  [xscale=0.55,yscale=0.55] [align=left] {\Large coordenada de reação};


\end{tikzpicture}
\end{center}

Analisando o diagrama qual o valor do $\Delta$H$_r$ para a reação \ch{CO2_{\gas} + 2 H2O_{\lqdd} -> CH4_{\gas} + 2 O2_{\gas}}.

\blank[blank-style={\phantom{#1}},width=12\linewidth]{}
\end{exercise}
\begin{solution}
\DeltaH=890 kJ
\end{solution}


\collectexercisesstop{ListaLeiHess}
\printcollection{ListaLeiHess}
\end{document}
