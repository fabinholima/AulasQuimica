% Created 2024-09-17 Tue 03:49
% Intended LaTeX compiler: lualatex
\documentclass[10pt]{scrartcl}


\KOMAoptions{
%headings=chapterprefix,
%%twocolumn=false,
%toc=indenttextentries,
%toc=flat,
twoside=true,
headinclude=true,
footinclude=true
%  captions=topbeside
}
%\usepackage[fontsize=12.3]{scrextend}
\usepackage{fontspec}
\usepackage[T1]{fontenc}
\usepackage{hyperref}
\usepackage[x11names,svgnames,table]{xcolor}
\defaultfontfeatures{Ligatures=TeX}
\setmainfont{Lato}
\usepackage{typearea}
\usepackage{lscape}
\usepackage[a4paper]{geometry}
\geometry{a4paper,total={170mm,257mm},left=30mm,right=30mm, top=15mm, bottom=20mm}
\usepackage[english, portuguese, american]{babel}
\usepackage{amsmath,amsfonts,amsthm,bm}
\usepackage{graphicx}
\usepackage{float,wrapfig}
\usepackage{colortbl}
\usepackage{tabularx}
\usepackage{pst-labo}
\usepackage{setspace}
\usepackage{xfrac}
\usepackage{tikz}
\usepackage{pgfplots}
\pgfplotsset{compat=1.3}
%% Diagraman latex
\usepackage{endiagram}
\usepackage{smartdiagram}
\usepackage[tikz]{bclogo}
\usetikzlibrary{fit,patterns,shadows.blur,shapes,decorations.pathreplacing,decorations.markings,arrows.meta,arrows,positioning,shadows,trees}
\usetikzlibrary{decorations.pathmorphing} %% to chemfig config bond
\usepackage{upgreek}
\usepackage{chemmacros}
%%\chemsetup{modules={reactions,spectroscopy,thermodynamics,redox,isotopes}}
\chemsetup{modules={all}}
%\usepackage[version=4,arrows=pgf-filled]{mhchem}
\usepackage{chemfig,elements,cancel,siunitx}
\NewChemPhase\lqdd{\(\ell\)}
\NewChemPhase\gr{grafite}
\NewChemPhase\reac{reação}
\setchemfig{fixed length=true, atom sep=2.0em, arrow offset=6pt, scheme debug=false,angle increment=30}
\renewcommand{\CancelColor}{\color{red}}
\usepackage{circuitikz}
\usepackage{mol2chemfig}
\usepackage{subfig,caption}
\captionsetup{font=small, labelfont={bf,sf}}
\usepackage{wrapfig,qrcode}
\usepackage{array,longtable} % ajust colunm table
\newcolumntype{J}{>{\centering\arraybackslash}m{7.5cm}}
\newcolumntype{K}{>{\centering\arraybackslash}m{6.5cm}}
\newcolumntype{L}{>{\centering\arraybackslash}m{5cm}}
\newcolumntype{B}{>{\centering\arraybackslash}m{2.5cm}}
\newcolumntype{N}{>{\centering\arraybackslash}m{1.4cm}}
\usepackage[most]{tcolorbox}
\newcounter{mycounter}
%%% Colobor
%%% Example colorbox
\newtcolorbox{Box2}[2][]{
lower separated=false,
colback=white,
colframe=black,fonttitle=\bfseries,
colbacktitle=black,
coltitle=white,
enhanced, attach boxed title to top left={yshift=-0.1in,xshift=0.15in}, boxed title style={boxrule=0pt,colframe=white,}, title=#2,#1}
%%%%%%%% Cabecalho
\usepackage{framed,amsmath}
\newtcolorbox{mybox}[2][]{
enhanced,title=#2, fonttitle=\sffamily\small,
top=2pt,
bottom=1mm,
boxrule=0.4pt,
coltitle=black,
colback=white,
attach boxed title to top center={yshift=-\tcboxedtitleheight/2,
yshifttext=-\tcboxedtitleheight/2},
boxed title style={
colframe=white,
colback=white,
left=0.2pt,
right=0.2pt},
#1}
\usepackage{tabularray}
%%%%%%
\newtcolorbox{exercisebox}%
{enhanced,breakable,colback=white, colframe=green!15!white,colbacktitle=white!15!pink, coltitle=pink!50!black,left=0pt,right=0mm,top=3mm,bottom=3mm,pad at break=0pt,bottomrule at break=0pt,toprule at break=0pt,borderline={0mm}{0mm}{green!50!white,dashed}, attach boxed title to top center={yshift=-2mm},boxed title style={boxrule=0.4pt},title=Exercícios,}
\usepackage{eso-pic}
\usepackage{etoolbox}
\usepackage{enumitem}
\newcommand\circitem[1]{%
\tikz[baseline=(char.base)]{%https://tex.stackexchange.com/questions/204116/uniform-size-of-circles-around-enumitems
\node[circle,draw=gray, fill=gray!30,
minimum size=1.2em,inner sep=0] (char) {#1};}}
\newcommand\boxitem[1]{%
\tikz[baseline=(char.base)]{%https://tex.stackexchange.com/questions/204116/uniform-size-of-circles-around-enumitems
\node[fill=orange!30,
minimum size=1.2em,inner sep=0] (char) {#1};}}
%\usepackage{widetext}% needs packages "flushend" & "cuted" of "sttools" % bundle, which perhaps must separately be installed
\newcommand{\dd}[1]{\hspace{2pt}d#1}
\definecolor{color1}{RGB}{0,0,90} % Color of the article title and sections
\definecolor{color2}{RGB}{0,20,20} % Color of the boxes behind the abstract and
\definecolor{cinza}{HTML}{C0C0C0}
%%% Custom Exercios
\usepackage{bohr}
\usepackage{multicol}
\setlength{\columnsep}{1.5cm}
\setlength{\columnseprule}{0.2pt}
\usepackage[no-files]{xsim}
\usepackage{tasks}
\xsimsetup{
goal-print={\pgfmathprintnumber[fixed zerofill,precision=1]{#1}}
}
\newcommand*\circled[2]{\tikz[baseline=(char.base)]{
\node[shape=circle,fill,inner sep=2pt, text=white] (char) {#1};}}
%%%%%-Custom Xsim exercises %%%%%
\DeclareExerciseEnvironmentTemplate{custom}
{%\item[\GetExerciseProperty{counter}]
\Needspace*{0\baselineskip}
\noindent
\circled{\XSIMmixedcase{\GetExerciseProperty{counter}}}~~~%
\noindent
\IfInsideSolutionF{%
\GetExercisePropertyT{points}{ % notice the space
(%
\printgoal{\PropertyValue}
\IfExerciseGoalSingularTF{points}
{%\XSIMtranslate{point}
}
{% \XSIMtranslate{points}
}%
)%
}
}}
{\vspace{\baselineskip}}

%%%%%------- Custom  resposta -------%%%%%%%
\DeclareExerciseEnvironmentTemplate{space}
%{\textbf{\GetExerciseProperty{counter}} }
{\noindent\circled{\XSIMmixedcase{\GetExerciseProperty{counter}}}~~~}
% {\circled{\XSIMmixedcase{\GetExerciseProperty{counter}}}}~~~%
{\qquad}
\newcommand*\answer[1]{%
\XSIMexpandcode{%
\SetExerciseProperty{solution-body}
{\noexpand{\Alph{task}}}}%
#1%
}
%\sisetup{locale=DE}
\xsimsetup{
collect = true,
exercise/within = section, %%% reset number xsim in
exercise/template = custom,
exercise/the-counter =  \arabic{exercise},
solution/template=custom,
%solution/print=true,
%print-collection/print=both,
%goal-print= {\pgfmathprintnumber[fixed zerofill,precision=1]\num{#1}}
}
\RenewDocumentCommand\printpoints{}{%
\TotalExerciseTypeGoal{exercise}{points}{}{}%
}
\NewTasksEnvironment[label = (\emph{\alph*}), label-width = 12pt]{choice}[\choice]
\newenvironment{questions}{\itemize}{\enditemize}
\DeclareExerciseHeadingTemplate{solution}{%
\section*{Gabarito}%
}
\everymath{\displaystyle}
%\usepackage{filecontents}
\usepackage{lineno}
\newcommand{\lh}{\underline{\hspace{1cm}}}
%%\onehalfspacing
\def\professor{Fábio Lima}
\def\aluno{}
\def\numerochamada{}
\def\disciplina{Química}
%%\def\disciplina{UCIII}
%%\def\disciplina{R.A.}
\def\turma{3 Ano }
%%\def\tipo{{\bfseries Avaliação Bimestral}}
\def\tipo{\bfseries RECUPERAÇÃO }
%%\def\tipo{\bfseries Exame Final}
\def\bimestre{2 Bimestre}
\def\escola{E.E. 26 de Agosto}
%%\def\escola{E.E. José Mamede de Aquino}
%\def\escola{E.E. Amelio Carvalho de Bais}
%5\def\escola{}
\def\dataprova{}
\DeclareExerciseCollection{HidroListaI}
\setcounter{secnumdepth}{0}
\date{\today}
\title{}
\hypersetup{
 pdfauthor={},
 pdftitle={},
 pdfkeywords={},
 pdfsubject={},
 pdfcreator={Emacs 29.4 (Org mode 9.6.15)}, 
 pdflang={English}}
\begin{document}

\pagebreak 
%%\input{../Modelos/geral}
%%\input{../Modelos/cabenovo}
%%\input{../Modelos/26agosto}
\input{../Modelos/CabeOficial}
%\section*{Atividade de Química}
\begin{itemize}
\item Realizar a atividade de recuperação em dupla
\item Atividade ira auxiliar na nota da Avaliação
\end{itemize}
\smallbreak
\medbreak
%\vspace{-9cm}





\collectexercises{HidroListaI}



\begin{exercise}
Determine o números de carbonos carbonos primários, secundários, terciários e quartenários, existentes em cada uma das estruturas a seguir e escreva suas fórmulas moleculares.

\begin{choice}(2)



\choice \chemfig{\mcfatomno{1}=^[:30]\mcfatomno{2}-[:90]\mcfatomno{3}(%
=^[:150]\mcfatomno{4}-[:210]\mcfatomno{5}=^[:270]\mcfatomno{6}%
-[:330]\phantom{1})-[:30]\mcfatomno{10}=_[:330]\mcfatomno{9}%
-[:270]\mcfatomno{8}=_[:210]\mcfatomno{7}(-[:150]\phantom{2})}


\choice \chemfig{\mcfatomno{1}-[:30]\mcfatomno{2}=[:330]\mcfatomno{3}(%
-[:270]\mcfatomno{8})-[:30]\mcfatomno{4}(-[:90]\mcfatomno{9})%
-[:330]\mcfatomno{5}-[:30]\mcfatomno{6}~[:330]\mcfatomno{7}}

\choice \chemfig{\mcfatomno{4}-[:210]\mcfatomno{2}(-[:105]\mcfatomno{5})%
-[:240]\mcfatomno{1}-[:120]\mcfatomno{3}(-\phantom{2})}


\choice \chemfig{\mcfatomno{11}-[:147]\mcfatomno{10}-[:207]\mcfatomno{9}(%
-[:270]\mcfatomno{12})-[:147]\mcfatomno{8}(<:[:87]\mcfatomno{14})%
-[:207]\mcfatomno{7}(<:[:267]\mcfatomno{15}(-[:327]\mcfatomno{17})%
-[:207]\mcfatomno{16})-[:147]\mcfatomno{6}(<[:87]\mcfatomno{13})%
-[:207]\mcfatomno{2}-[:216]\mcfatomno{1}-[:144]\mcfatomno{5}%
-[:72]\mcfatomno{4}-\mcfatomno{3}(-[:288]\phantom{2})}


\choice \chemfig{\mcfatomno{1}=^[:30]\mcfatomno{2}-[:90]\mcfatomno{3}%
=^[:150]\mcfatomno{4}-[:210]\mcfatomno{5}=^[:270]\mcfatomno{6}(%
-[:330]\phantom{1})}



\choice \chemfig{\mcfatomno{1}-[:30]\mcfatomno{2}-[:330]\mcfatomno{3}(%
<[:270]\mcfatomno{9})-[:30]\mcfatomno{4}(<[:90]\mcfatomno{10})%
-[:330]\mcfatomno{5}~[:30]\mcfatomno{6}-[:330]\mcfatomno{7}%
=_[:30]\mcfatomno{8}-[:330]\mcfatomno{11}(-[:30]\mcfatomno{13})%
-[:270]\mcfatomno{14}-[:210]\mcfatomno{16}-[:150]\mcfatomno{12}(%
-[:90]\phantom{7})-[:210]\mcfatomno{15}} 
\end{choice}
\end{exercise}


\begin{exercise}
Classifique as cadeias carbônicas dos compostos indicados e sua fórmula molecular.

\begin{choice}(2)



\choice \chemname{\chemfig{CH_2([:90]-C{\ell})-CH_2-S-CH_2-CH_2([:90]-C{\ell})}}{gás mostarda usado em guerras}

\choice \chemfig{-[:30](-[:90])-[:330](-[:270])-[:30]=[:330]}

%\choice \chemname{\chemfig{CH_3-CH_2-CH_2-C([:90]=O)-O-CH_2-CH_3}}{Essência de abacaxi}

\choice \chemname{\chemfig{=^[:30]-[:90](=^[:150]-[:210]=^[:270]-[:330])-[:30]=_[:330]-[:270]%
(=_[:210]-[:150])-[:330]=^[:30]-[:90]=^[:150](-[:210])}}{Fumaça do cigarro}


\choice \chemname{\chemfig{-[:60]-[:180](-[:300])}}{Anestésico}


\choice \chemname{\chemfig{NH_2-[:210,,1]-[:270]=_[:210]-[:150]=_[:90]-[:30](=_[:330])}}{Corantes}

\choice \chemname{\chemfig{O=[:90]-[:30]=^[:90]-[:150](=[:90]O)-[:210]=^[:270](-[:330])}}{Veneno besouro}
\end{choice}
\end{exercise}






\begin{exercise}
Determine a formula molecular dos compostos

\begin{choice}(2)

\choice \chemfig{-[:60](=[:120]O)-O-[:300]=^[:240]-[:300]=^-[:60]=^[:120](-[:180])%
-[:60](-[,,,1]OH)=[:120]O}

\choice \chemfig{-[:42]N-[:96]=_[:24]N-[:312]=_[:240](-[:168]\phantom{N})-[:300](%
=[:240]O)-N(-[:300])-[:60](=O)-[:120]N(-[:180])-[:60]}

\choice \chemfig{-[:90]N(-[:30](-[:330,,,1]NH_2)=[:90,,,1]NH)-[:150]-[:210](%
=[:270]O)-[:150,,,2]HO}

%\choice \chemfig{O=[:270](-[:210,,,2]HO)-[:330](-[:270,,,1]NH_2)-[:30]-[:330]-[:30]-[:330]N=[:30](-[:90,,,1]NH_2)-[:330,,,1]NH_2}


%\choice \chemfig{-[:282](=[:222]O)-[:342]\mcfbelow{N}{H}-[:42]-[:342]-[:42]=_[:96]%
%-[:24]\mcfabove{N}{H}-[:312]=^[:240](-[:168])-[:300]=^(-[:300]O-)-[:60]%
%=^[:120](-[:180])}

\choice \chemfig{O=[:300](-[:240,,,2]HO)-(-[:300,,,1]NH_2)-[:60]-[,,,1]SH}
\end{choice}
\end{exercise}


\begin{exercise}
Dê o nome IUPAC para os compostos abaixo:

\begin{choice}(2)
\choice \chemfig{-[:210](-[:270])-[:150]-[:210]-[:150]-[:90]-[:30](-[:330]-[:270])%
-[:90]-[:150]}

\choice \chemfig{-[:234]-[:288]-[:216](-[:270])-[:144]-[:72](-)-[:126]}

\choice \chemfig{-[:30](=[:90])-[:330]=[:30]-[:330]-[:30]=[:330]}

\choice \chemfig{-[:30](-[:90](-[:30]-[:330]-[:30])-[:150])-[:330](-[:270])-[:30]%
-[:330]~[:30]-[:30]}

\choice \chemfig{CH_3-[:150,,1](-[:90,,,1]CH_3)-[:210](-[:270,,,1]CH_3)-[:150](%
-[:90,,,1]CH_3)=[:210](-[:270,,,1]CH_3)-[:150]~[:210]-[:210]-[:150]%
-[:210,,,2]H_3C}
\end{choice}
\end{exercise}





\collectexercisesstop{HidroListaI}

\printcollection{HidroListaI}
\end{document}
