% Created 2024-05-31 Fri 15:56
% Intended LaTeX compiler: lualatex
\documentclass[10pt]{scrartcl}


\KOMAoptions{
%headings=chapterprefix,
%%twocolumn=false,
%toc=indenttextentries,
%toc=flat,
twoside=true,
headinclude=true,
footinclude=true
%  captions=topbeside
}
%\usepackage[fontsize=12.3]{scrextend}
\usepackage{fontspec}
\usepackage[T1]{fontenc}
\usepackage{hyperref}
\usepackage[x11names,svgnames,table]{xcolor}
\defaultfontfeatures{Ligatures=TeX}
\setmainfont{Lato}
\usepackage{typearea}
\usepackage{lscape}
\usepackage[a4paper]{geometry}
\geometry{a4paper,total={170mm,257mm},left=30mm,right=30mm, top=15mm, bottom=20mm}
\usepackage[english, portuguese, american]{babel}
\usepackage{amsmath,amsfonts,amsthm,bm}
\usepackage{graphicx}
\usepackage{float,wrapfig}
\usepackage{colortbl}
\usepackage{tabularx}
\usepackage{pst-labo}
\usepackage{setspace}
\usepackage{xfrac}
\usepackage{tikz}
\usepackage{pgfplots}
\pgfplotsset{compat=1.3}
%% Diagraman latex
\usepackage{endiagram}
\usepackage{smartdiagram}
\usepackage[tikz]{bclogo}
\usetikzlibrary{fit,patterns,shadows.blur,shapes,decorations.pathreplacing,decorations.markings,arrows.meta,arrows,positioning,shadows,trees}
\usetikzlibrary{decorations.pathmorphing} %% to chemfig config bond
\usepackage{upgreek}
\usepackage{chemmacros}
%%\chemsetup{modules={reactions,spectroscopy,thermodynamics,redox,isotopes}}
\chemsetup{modules={all}}
%\usepackage[version=4,arrows=pgf-filled]{mhchem}
\usepackage{chemfig,elements,cancel,siunitx}
\NewChemPhase\lqdd{\(\ell\)}
\NewChemPhase\gr{grafite}
\NewChemPhase\reac{reação}
\setchemfig{fixed length=true, atom sep=2.0em, arrow offset=6pt, scheme debug=false,angle increment=30}
\renewcommand{\CancelColor}{\color{red}}
\usepackage{circuitikz}
\usepackage{mol2chemfig}
\usepackage{subfig,caption}
\captionsetup{font=small, labelfont={bf,sf}}
\usepackage{wrapfig,qrcode}
\usepackage{array,longtable} % ajust colunm table
\newcolumntype{J}{>{\centering\arraybackslash}m{7.5cm}}
\newcolumntype{K}{>{\centering\arraybackslash}m{6.5cm}}
\newcolumntype{L}{>{\centering\arraybackslash}m{5cm}}
\newcolumntype{B}{>{\centering\arraybackslash}m{2.5cm}}
\newcolumntype{N}{>{\centering\arraybackslash}m{1.4cm}}
\usepackage[most]{tcolorbox}
\newcounter{mycounter}
%%% Colobor
%%% Example colorbox
\newtcolorbox{Box2}[2][]{
lower separated=false,
colback=white,
colframe=black,fonttitle=\bfseries,
colbacktitle=black,
coltitle=white,
enhanced, attach boxed title to top left={yshift=-0.1in,xshift=0.15in}, boxed title style={boxrule=0pt,colframe=white,}, title=#2,#1}
%%%%%%%% Cabecalho
\usepackage{framed,amsmath}
\newtcolorbox{mybox}[2][]{
enhanced,title=#2, fonttitle=\sffamily\small,
top=2pt,
bottom=1mm,
boxrule=0.4pt,
coltitle=black,
colback=white,
attach boxed title to top center={yshift=-\tcboxedtitleheight/2,
yshifttext=-\tcboxedtitleheight/2},
boxed title style={
colframe=white,
colback=white,
left=0.2pt,
right=0.2pt},
#1}
\usepackage{tabularray}
%%%%%%
\newtcolorbox{exercisebox}%
{enhanced,breakable,colback=white, colframe=green!15!white,colbacktitle=white!15!pink, coltitle=pink!50!black,left=0pt,right=0mm,top=3mm,bottom=3mm,pad at break=0pt,bottomrule at break=0pt,toprule at break=0pt,borderline={0mm}{0mm}{green!50!white,dashed}, attach boxed title to top center={yshift=-2mm},boxed title style={boxrule=0.4pt},title=Exercícios,}
\usepackage{eso-pic}
\usepackage{etoolbox}
\usepackage{enumitem}
\newcommand\circitem[1]{%
\tikz[baseline=(char.base)]{%https://tex.stackexchange.com/questions/204116/uniform-size-of-circles-around-enumitems
\node[circle,draw=gray, fill=gray!30,
minimum size=1.2em,inner sep=0] (char) {#1};}}
\newcommand\boxitem[1]{%
\tikz[baseline=(char.base)]{%https://tex.stackexchange.com/questions/204116/uniform-size-of-circles-around-enumitems
\node[fill=orange!30,
minimum size=1.2em,inner sep=0] (char) {#1};}}
%\usepackage{widetext}% needs packages "flushend" & "cuted" of "sttools" % bundle, which perhaps must separately be installed
\newcommand{\dd}[1]{\hspace{2pt}d#1}
\definecolor{color1}{RGB}{0,0,90} % Color of the article title and sections
\definecolor{color2}{RGB}{0,20,20} % Color of the boxes behind the abstract and
\definecolor{cinza}{HTML}{C0C0C0}
%%% Custom Exercios
\usepackage{bohr}
\usepackage{multicol}
\setlength{\columnsep}{1.5cm}
\setlength{\columnseprule}{0.2pt}
\usepackage[no-files]{xsim}
\usepackage{tasks}
\xsimsetup{
goal-print={\pgfmathprintnumber[fixed zerofill,precision=1]{#1}}
}
\newcommand*\circled[2]{\tikz[baseline=(char.base)]{
\node[shape=circle,fill,inner sep=2pt, text=white] (char) {#1};}}
%%%%%-Custom Xsim exercises %%%%%
\DeclareExerciseEnvironmentTemplate{custom}
{%\item[\GetExerciseProperty{counter}]
\Needspace*{0\baselineskip}
\noindent
\circled{\XSIMmixedcase{\GetExerciseProperty{counter}}}~~~%
\noindent
\IfInsideSolutionF{%
\GetExercisePropertyT{points}{ % notice the space
(%
\printgoal{\PropertyValue}
\IfExerciseGoalSingularTF{points}
{%\XSIMtranslate{point}
}
{% \XSIMtranslate{points}
}%
)%
}
}}
{\vspace{\baselineskip}}

%%%%%------- Custom  resposta -------%%%%%%%
\DeclareExerciseEnvironmentTemplate{space}
%{\textbf{\GetExerciseProperty{counter}} }
{\noindent\circled{\XSIMmixedcase{\GetExerciseProperty{counter}}}~~~}
% {\circled{\XSIMmixedcase{\GetExerciseProperty{counter}}}}~~~%
{\qquad}
\newcommand*\answer[1]{%
\XSIMexpandcode{%
\SetExerciseProperty{solution-body}
{\noexpand{\Alph{task}}}}%
#1%
}
%\sisetup{locale=DE}
\xsimsetup{
collect = true,
exercise/within = section, %%% reset number xsim in
exercise/template = custom,
exercise/the-counter =  \arabic{exercise},
solution/template=custom,
%solution/print=true,
%print-collection/print=both,
%goal-print= {\pgfmathprintnumber[fixed zerofill,precision=1]\num{#1}}
}
\RenewDocumentCommand\printpoints{}{%
\TotalExerciseTypeGoal{exercise}{points}{}{}%
}
\NewTasksEnvironment[label = (\emph{\alph*}), label-width = 12pt]{choice}[\choice]
\newenvironment{questions}{\itemize}{\enditemize}
\DeclareExerciseHeadingTemplate{solution}{%
\section*{Gabarito}%
}
\everymath{\displaystyle}
%\usepackage{filecontents}
\usepackage{lineno}
\newcommand{\lh}{\underline{\hspace{1cm}}}
%%\onehalfspacing
\def\professor{Fábio Lima}
\def\aluno{}
\def\numerochamada{}
\def\disciplina{Química}
%%\def\disciplina{UCIII}
%%\def\disciplina{R.A.}
\def\turma{3 Ano }
%%\def\tipo{{\bfseries Avaliação Bimestral}}
\def\tipo{\bfseries RECUPERAÇÃO }
%%\def\tipo{\bfseries Exame Final}
\def\bimestre{2 Bimestre}
\def\escola{E.E. 26 de Agosto}
%%\def\escola{E.E. José Mamede de Aquino}
%\def\escola{E.E. Amelio Carvalho de Bais}
%5\def\escola{}
\def\dataprova{}
\DeclareExerciseCollection{BalanceamentoRedoxI}
\DeclareExerciseCollection{BalanceamentoRedoxII}
\DeclareExerciseCollection{BalanceamentoRedoxIII}
\setcounter{secnumdepth}{0}
\date{\today}
\title{}
\hypersetup{
 pdfauthor={},
 pdftitle={},
 pdfkeywords={},
 pdfsubject={},
 pdfcreator={Emacs 29.3 (Org mode 9.6.15)}, 
 pdflang={English}}
\begin{document}

\pagebreak 
%%\input{../Modelos/geral}
%%\input{../Modelos/cabenovo}
%%\input{../Modelos/26agosto}
\input{../Modelos/CabeOficial}
%\section*{Tópicos RPP de Química}
\smallbreak
\medbreak
\vspace{-9cm}


\collectexercises{BalanceamentoRedoxI}

\begin{exercise}
\ch{As2O3 + HNO3 + H2O ->  H3AsO4 + NO} \vspace{1cm} 
\end{exercise}


\begin{exercise}
\ch{KI + KNO2 + H2SO4 ->  I2 + NO + K2SO4 + H2O}\vspace{1cm}
\end{exercise}



\begin{exercise}
\ch{KI + H2SO4 -> K2SO4 + I2 + H2S + H2O}  \vspace{1cm}
\end{exercise}


\begin{exercise}
\ch{C + H2SO4 -> CO2 + SO2 + H2O}  \vspace{1cm}
\end{exercise}


\begin{exercise}
\ch{Cu + H2SO4 -> CuSO4 + SO2 + H2O}  \vspace{1cm}
\end{exercise}


\begin{exercise}
\ch{KSCN + H2O + I2 -> KHSO4 + HI + ICN}  \vspace{1cm}
\end{exercise}


\begin{exercise}
\ch{A$\ell$Br3 + KMnO4 + H2SO4 -> A$\ell$2(SO4)3 + K2SO4 + MnSO4 + Br2 + H2O}  \vspace{1cm}
\end{exercise}

\begin{exercise}
\ch{FeSO4 + KMnO4 + H2SO4 -> Fe2(SO4)3 + K2SO4 + MnSO4 + H2O}  \vspace{1cm}
\end{exercise}


\begin{exercise}
\ch{Na2C2O4 + KMnO4 + H2SO4 -> K2SO4 + Na2SO4 + MnSO4 + CO2 + H2O} \vspace{1cm}
\end{exercise}



\begin{exercise}
\ch{KMnO4 + Na2SO3 + H2SO4 ->  MnSO4 + Na2SO4 + K2SO4 + H2O}
\end{exercise}



\collectexercisesstop{BalanceamentoRedoxI}




\collectexercises{BalanceamentoRedoxII}

\begin{exercise}
 \ch{Mn^{2+} + BiO^{3-} ->  MnO4^- +  Bi^{3+}} 
\end{exercise}
\begin{solution}
\ch{14 H^+ + 2 Mn^{2+} + 5 BiO3^- -> 2 MnO4^- + 5 Bi^{3+} + 7 H2O}
\end{solution}
\vspace{1cm}


\begin{exercise}
 \ch{MnO4^- + S2O3^2- -> S4O6^2- + Mn^2+}
\end{exercise}
\vspace{1cm}



\begin{exercise}
 \ch{C$\ell$O3^- + C$\ell$^- -> C$\ell$2 + C$\ell$O2} 
\end{exercise}
\vspace{1cm}


\begin{exercise}
 \ch{P + Cu^2+ -> Cu + H2PO4^-}
\end{exercise}
\vspace{1cm}



\begin{exercise}
 \ch{PH3 + I2 ->  H3PO2^- + I^-} 
\end{exercise}
\vspace{1cm}






\begin{exercise}
 \ch{NO2 ->  NO3^- + NO} 
\end{exercise}
\bigskip



\begin{exercise}
 \ch{H2O2 + Cr2O7^2- -> O2 + Cr^3+}
\end{exercise}
\begin{solution}
 \ch{8 H^+ + 3 H2O2 + Cr2O7^2- -> 3 O2 + 2 Cr^3+ + 7 H2O }
\end{solution}
\bigskip




\begin{exercise}
 \ch{PbO2 + I2 -> Pb^2+ + IO3^-}
\end{exercise}
\begin{solution}
 \ch{8H^+ + 5 PbO2 + I2 ->  5 Pb^2+ + 2 IO3^- + 4 H2O}
\end{solution}
\bigskip




\begin{exercise}
 \ch{ReO4^- + IO^- -> IO3^- + Re}
\end{exercise}
\begin{solution}
 \ch{ReO4^- + IO^- -> IO3^- + Re} 
\end{solution}
\bigskip
\bigskip




\begin{exercise}
 \ch{As -> H2AsO4- + AsH3}
\end{exercise}
\begin{solution}
 \ch{12 H2O + 8 As -> 3 H2AsO4- + 5 AsH3 + 3 H^+}
\end{solution}
\bigskip
\bigskip




\collectexercisesstop{BalanceamentoRedoxII}




\collectexercises{BalanceamentoRedoxIII}


\begin{exercise}
 \ch{MnO4^- +  C2O4^{2-} ->  MnO2 + CO2}
\end{exercise}
\bigskip


\begin{exercise}
 \ch{Cu(NH3)4^{2+} + S2O4^{2-} -> SO3^{2-} + Cu + NH3}
\end{exercise}
\begin{solution}
\end{solution}
\bigskip


\begin{exercise}
 \ch{Zn + NO3^- ->  Zn(OH)4^2- + NH3}
\end{exercise}
\begin{solution}
\end{solution}
\bigskip

\begin{exercise}
 \ch{A$\ell$  + OH^- -> A$\ell$O2^- + H2} 
\end{exercise}
\bigskip

\begin{exercise}
 \ch{Zn ->  Zn(OH)4^{2-} + H2} 
\end{exercise}




\begin{exercise}
\end{exercise}
\begin{solution}
\end{solution}
\bigskip
\bigskip





\begin{exercise}
\end{exercise}
\begin{solution}
\end{solution}
\bigskip
\bigskip


\begin{exercise}
\end{exercise}
\begin{solution}
\end{solution}
\bigskip
\bigskip



\collectexercisesstop{BalanceamentoRedoxIII}



\section{Realize o balanceamento das reações redox}

\printcollection{BalanceamentoRedoxI}

\newpage

\section{Equilibre cada reação redox em solução ácida.}


\printcollection{BalanceamentoRedoxII}

\newpage


\section{Equilibre cada reação redox em solução básica.}


\printcollection{BalanceamentoRedoxIII}
\end{document}
