% Created 2024-11-12 Tue 15:17
% Intended LaTeX compiler: lualatex
\documentclass[12pt]{scrartcl}


\KOMAoptions{
%headings=chapterprefix,
%%twocolumn=false,
%toc=indenttextentries,
%toc=flat,
twoside=true,
headinclude=true,
footinclude=true
%  captions=topbeside
}
%\usepackage[fontsize=12.3]{scrextend}
\usepackage{fontspec}
\usepackage[T1]{fontenc}
\usepackage{hyperref}
\usepackage[x11names,svgnames,table]{xcolor}
\defaultfontfeatures{Ligatures=TeX}
\setmainfont{Lato}
\usepackage{typearea}
\usepackage{lscape}
\usepackage[a4paper]{geometry}
\geometry{a4paper,total={170mm,257mm},left=30mm,right=30mm, top=15mm, bottom=20mm}
\usepackage[english, portuguese, american]{babel}
\usepackage{amsmath,amsfonts,amsthm,bm}
\usepackage{graphicx}
\usepackage{float,wrapfig}
\usepackage{colortbl}
\usepackage{tabularx}
\usepackage{pst-labo}
\usepackage{setspace}
\usepackage{xfrac}
\usepackage{tikz}
\usepackage{pgfplots}
\pgfplotsset{compat=1.3}
%% Diagraman latex
\usepackage{endiagram}
\usepackage{smartdiagram}
\usepackage[tikz]{bclogo}
\usetikzlibrary{fit,patterns,shadows.blur,shapes,decorations.pathreplacing,decorations.markings,arrows.meta,arrows,positioning,shadows,trees}
\usetikzlibrary{decorations.pathmorphing} %% to chemfig config bond
\usepackage{upgreek}
\usepackage{chemmacros}
%%\chemsetup{modules={reactions,spectroscopy,thermodynamics,redox,isotopes}}
\chemsetup{modules={all}}
%\usepackage[version=4,arrows=pgf-filled]{mhchem}
\usepackage{chemfig,elements,cancel,siunitx}
\NewChemPhase\lqdd{\(\ell\)}
\NewChemPhase\gr{grafite}
\NewChemPhase\reac{reação}
\setchemfig{
angle increment=30,
atom sep=1.67em,
double bond sep=0.67ex,
bond style={line width=0.1em},
cram width=0.8ex,
cram dash width=0.1em,
cram dash sep=0.2em,
arrow style={line width=0.067em},
arrow head=-{Triangle},
arrow label sep=1ex,
cycle radius coeff=0.75,
chemfig style={line width=0.1em},
}
%\setchemfig{fixed length=true, atom sep=2.0em, arrow offset=6pt, scheme debug=false,angle increment=30}
\renewcommand{\CancelColor}{\color{red}}
\usepackage{circuitikz}
\usepackage{mol2chemfig}
\usepackage{subfig,caption}
\captionsetup{font=small, labelfont={bf,sf}}
\usepackage{wrapfig,qrcode}
\usepackage{array,longtable} % ajust colunm table
\newcolumntype{J}{>{\centering\arraybackslash}m{7.5cm}}
\newcolumntype{K}{>{\centering\arraybackslash}m{6.5cm}}
\newcolumntype{L}{>{\centering\arraybackslash}m{5cm}}
\newcolumntype{B}{>{\centering\arraybackslash}m{2.5cm}}
\newcolumntype{N}{>{\centering\arraybackslash}m{1.4cm}}
\usepackage[most]{tcolorbox}
\newcounter{mycounter}
%%% Colobor
%%% Example colorbox
\newtcolorbox{Box2}[2][]{
lower separated=false,
colback=white,
colframe=black,fonttitle=\bfseries,
colbacktitle=black,
coltitle=white,
enhanced, attach boxed title to top left={yshift=-0.1in,xshift=0.15in}, boxed title style={boxrule=0pt,colframe=white,}, title=#2,#1}
%%%%%%%% Cabecalho
\usepackage{framed,amsmath}
\newtcolorbox{mybox}[2][]{
enhanced,title=#2, fonttitle=\sffamily\small,
top=2pt,
bottom=1mm,
boxrule=0.4pt,
coltitle=black,
colback=white,
attach boxed title to top center={yshift=-\tcboxedtitleheight/2,
yshifttext=-\tcboxedtitleheight/2},
boxed title style={
colframe=white,
colback=white,
left=0.2pt,
right=0.2pt},
#1}
\usepackage{tabularray}
%%%%%%
\newtcolorbox{exercisebox}%
{enhanced,breakable,colback=white, colframe=green!15!white,colbacktitle=white!15!pink, coltitle=pink!50!black,left=0pt,right=0mm,top=3mm,bottom=3mm,pad at break=0pt,bottomrule at break=0pt,toprule at break=0pt,borderline={0mm}{0mm}{green!50!white,dashed}, attach boxed title to top center={yshift=-2mm},boxed title style={boxrule=0.4pt},title=Exercícios,}
\usepackage{eso-pic}
\usepackage{etoolbox}
\usepackage{enumitem}
\newcommand\circitem[1]{%
\tikz[baseline=(char.base)]{%https://tex.stackexchange.com/questions/204116/uniform-size-of-circles-around-enumitems
\node[circle,draw=gray, fill=gray!30,
minimum size=1.2em,inner sep=0] (char) {#1};}}
\newcommand\boxitem[1]{%
\tikz[baseline=(char.base)]{%https://tex.stackexchange.com/questions/204116/uniform-size-of-circles-around-enumitems
\node[fill=orange!30,
minimum size=1.2em,inner sep=0] (char) {#1};}}
%\usepackage{widetext}% needs packages "flushend" & "cuted" of "sttools" % bundle, which perhaps must separately be installed
\newcommand{\dd}[1]{\hspace{2pt}d#1}
\definecolor{color1}{RGB}{0,0,90} % Color of the article title and sections
\definecolor{color2}{RGB}{0,20,20} % Color of the boxes behind the abstract and
\definecolor{cinza}{HTML}{C0C0C0}
%%% Custom Exercios
\usepackage{bohr}
\usepackage{multicol}
\setlength{\columnsep}{1.5cm}
\setlength{\columnseprule}{0.2pt}
\usepackage[no-files]{xsim}
\usepackage{tasks}
\xsimsetup{
goal-print={\pgfmathprintnumber[fixed zerofill,precision=1]{#1}}
}
\newcommand*\circled[2]{\tikz[baseline=(char.base)]{
\node[shape=circle,fill,inner sep=2pt, text=white] (char) {#1};}}
%%%%%-Custom Xsim exercises %%%%%
\DeclareExerciseEnvironmentTemplate{custom}
{%\item[\GetExerciseProperty{counter}]
\Needspace*{0\baselineskip}
\noindent
\circled{\XSIMmixedcase{\GetExerciseProperty{counter}}}~~~%
\noindent
\IfInsideSolutionF{%
\GetExercisePropertyT{points}{ % notice the space
(%
\printgoal{\PropertyValue}
\IfExerciseGoalSingularTF{points}
{%\XSIMtranslate{point}
}
{% \XSIMtranslate{points}
}%
)%
}
}}
{\vspace{\baselineskip}}

%%%%%------- Custom  resposta -------%%%%%%%
\DeclareExerciseEnvironmentTemplate{space}
%{\textbf{\GetExerciseProperty{counter}} }
{\noindent\circled{\XSIMmixedcase{\GetExerciseProperty{counter}}}~~~}
% {\circled{\XSIMmixedcase{\GetExerciseProperty{counter}}}}~~~%
{\qquad}
\newcommand*\answer[1]{%
\XSIMexpandcode{%
\SetExerciseProperty{solution-body}
{\noexpand{\Alph{task}}}}%
#1%
}
%\sisetup{locale=DE}
\xsimsetup{
collect = true,
exercise/within = section, %%% reset number xsim in
exercise/template = custom,
exercise/the-counter =  \arabic{exercise},
solution/template=custom,
%solution/print=true,
%print-collection/print=both,
%goal-print= {\pgfmathprintnumber[fixed zerofill,precision=1]\num{#1}}
}
\RenewDocumentCommand\printpoints{}{%
\TotalExerciseTypeGoal{exercise}{points}{}{}%
}
\NewTasksEnvironment[label = (\emph{\alph*}), label-width = 12pt]{choice}[\choice]
\newenvironment{questions}{\itemize}{\enditemize}
\DeclareExerciseHeadingTemplate{solution}{%
\section*{Gabarito}%
}
\everymath{\displaystyle}
%\usepackage{filecontents}
\usepackage{lineno}
\newcommand{\lh}{\underline{\hspace{1cm}}}
%%\onehalfspacing
\def\professor{Fábio Lima}
\def\aluno{}
\def\numerochamada{}
\def\disciplina{Química}
%%\def\disciplina{UCIII}
%%\def\disciplina{R.A.}
\def\turma{3 Ano }
%%\def\tipo{{\bfseries Avaliação Bimestral}}
\def\tipo{\bfseries RECUPERAÇÃO }
%%\def\tipo{\bfseries Exame Final}
\def\bimestre{2 Bimestre}
\def\escola{E.E. 26 de Agosto}
%%\def\escola{E.E. José Mamede de Aquino}
%\def\escola{E.E. Amelio Carvalho de Bais}
%5\def\escola{}
\def\dataprova{}
\DeclareExerciseCollection{VolumetriaI}
\setcounter{secnumdepth}{0}
\author{fabio}
\date{\today}
\title{}
\hypersetup{
 pdfauthor={fabio},
 pdftitle={},
 pdfkeywords={},
 pdfsubject={},
 pdfcreator={Emacs 29.4 (Org mode 9.6.15)}, 
 pdflang={English}}
\begin{document}

\pagebreak 
%%\input{../Modelos/geral}
%%\input{../Modelos/cabenovo}
%%\input{../Modelos/26agosto}
\input{../Modelos/CabeOficial}
%\section*{Atividade de Química}
%\begin{itemize}
%\item Realizar a atividade de recuperação em dupla
%\item Atividade ira auxiliar na nota da Avaliação
%\end{itemize}
\smallbreak
\medbreak
%\vspace{-9cm}




\collectexercises{VolumetriaI}


\begin{exercise}
Uma amostra de 50,0 mL de um vinho branco requer 21,48 mL de uma solução 0,03776 \unit{\mol\per\litre} de NaOH para alcançar o ponto final com fenolftaleína. Determine a acidez do vinho em termos de \% m/v de ácido tartárico (\ch{H2C4H4O6} – MM = 150,09 g/mol).

\schemestart
\chemfig{O=[:270](-[:330,,,1]OH)-[:210](-[:270,,,1]OH)-[:150](-[:90,,,1]OH)%
-[:210](=[:270]O)-[:150,,,2]HO} \+ NaOH \arrow \chemfig{O=[:270](-[:330,,,1]ONa)-[:210](-[:270,,,1]OH)-[:150](-[:90,,,1]OH)-[:210](=[:270]O)-[:150,,,2]NaO} + 2 \chemfig{H_2O} 
\schemestop 
\end{exercise}

\begin{solution}
Resposta: 012\% m/v de H2C2H4O6
\end{solution}

\begin{exercise}
Uma alíquota de 25 mL de vinagre foi diluída para 250 mL em um balão volumétrico. Em seguida, várias alíquotas de 50 mL dessa solução diluída foram titulados com um volume médio de 34,88 mL de NaOH 0,096 \unit{\mol\per\litre} . Determine a acidez do vinho em termos de \% m/v de ácido acético (MM = 60 g/mol).

\begin{reaction*}
CH3COOH + NaOH -> CH3COONa + H2O
\end{reaction*}
\end{exercise}
\begin{solution}
Resposta: 4,02 \% m/v de HC2H4O2
\end{solution}



\begin{exercise}
O ácido benzoico extraído de 106,3 g de molho de tomate foram titulados com 14,76 mL de solução 0,05250 mol/L de NaOH. Determine a porcentagem e termos de benzoato de sódio (144,10 g/mol).

\schemestart
\chemfig{O=[:330](-[:30,,,1]OH)-[:270]=_[:330]-[:270]=_[:210]-[:150]=_[:90](-[:30])} + NaOH \arrow \chemfig{O=[:330](-[:30,,,1]ONa)-[:270]=_[:330]-[:270]=_[:210]-[:150]=_[:90](-[:30])} + \chemfig{H_2O}
\schemestop
\end{exercise}
\begin{solution}
Resposta: 0,11\%.
\end{solution}


\begin{exercise}
Barrilha, que é carbonato de sódio impuro \ch{Na2CO3}, é um insumo básico da indústria química. Uma amostra de barrilha de 10 g foi totalmente dissolvida em 800 mL de ácido clorídrico 0,2 \unit{\mol\per\litre} de \ch{HC$\ell$}. O excesso de ácido clorídrico foi neutralizado por 250 mL de NaOH 0,1 \unit{\mol\per\litre}. Qual é o teor de carbonato de sódio, em porcentagem de massa na amostra da barrilha? Dados: (\ch{Na2CO3} MM = 122 \unit{\gram\per\mol})

\begin{reactions}
Na2CO3 + 2 HC$\ell$ -> 2 NaC$\ell$ + H2O + CO2 \\
HC$\ell$  + NaOH -> NaC$\ell$ + H2O
\end{reactions}
\end{exercise}




\begin{exercise}
Na determinação de 25 mL de solução de concentração de uma solução de \ch{HNO3} foram gastos 25 mL de solução de KOH 0,1 mol/L.
\begin{reaction*}
HNO3 + KOH -> KNO3 + H2O
\end{reaction*}
\end{exercise}


\begin{exercise}
A pureza do conservante de alimentos ácido benzóico (\ch{C7H6O2}) foi determinada pela titulação deste com uma solução padronizada de NaOH 0,1 \unit{\mol\per\litre}, gastando 35 mL. Para tanto, foi preparada uma solução do ácido pela dissolução de 0,50 g em 50 mL de água deionizada. Todos os 50 mL foram utilizados na titulação com NaOH. Determine a pureza do conservante.

\schemestart
\chemfig{O=[:330](-[:30,,,1]OH)-[:270]=_[:330]-[:270]=_[:210]-[:150]=_[:90](-[:30])} + NaOH \arrow \chemfig{O=[:330](-[:30,,,1]ONa)-[:270]=_[:330]-[:270]=_[:210]-[:150]=_[:90](-[:30])} + \chemfig{H_2O}
\schemestop
\end{exercise}


\begin{exercise}
Um estudante verifica que 20 mL de hidróxido de potássio (KOH) 0,3 \unit{\mol\per\litre} são necessários para neutralizar uma amostra de 30 mL de ácido clorídrico (HC\(\ell\)). Determine a molaridade do HC\(\ell\).
\begin{reaction*}
KOH + HC$\ell$ -> KC$\ell$ + H2O 
\end{reaction*}
\end{exercise}



\begin{exercise}
Para realizar a titulação de 20 mL de hidróxido de sódio (NaOH) de molaridade desconhecida, foram utilizados 50 mL de ácido sulfúrico (\ch{H2SO4}) 0,2 \unit{\mol\per\litre}. Qual a molaridade do hidróxido de sódio?

\begin{reaction*}
H2SO4 + 2 NaOH -> Na2SO4 + 2 H2O
\end{reaction*}
\end{exercise}



\collectexercisesstop{VolumetriaI}

\printcollection{VolumetriaI}
\end{document}
