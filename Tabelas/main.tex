\documentclass[12pt]{scrartcl}
\KOMAoptions{
	%headings=chapterprefix,
	%twocolumn=true,
	%toc=indenttextentries,
	%toc=flat,
	twoside=true,
	headinclude=true,
	footinclude=true
	%  captions=topbeside
}

\usepackage{fontspec}
\usepackage[T1]{fontenc}
\usepackage{hyperref}
\usepackage[x11names,svgnames,table]{xcolor}
\defaultfontfeatures{Ligatures=TeX}
%\setmainfont{Lato}
%\setmainfont{Charis SIL}
%\setmainfont{Alegreya}

\usepackage{typearea}
\usepackage{lscape}
\usepackage[a4paper]{geometry}
\geometry{a4paper,total={170mm,257mm},left=10mm,right=10mm, top=15mm, bottom=20mm}
\usepackage[portuguese, american]{babel}
\usepackage{amsmath,amsfonts,amsthm,bm}

\usepackage{xcolor}
\usepackage{tabularray}
\usepackage{chemmacros}
\usepackage{chemfig}
 
 
 \begin{document}
 
 \NewTblrTheme{fancy}{
 	\SetTblrStyle{caption-tag}{font=\bfseries}
 }
 
 
\selectlanguage{portuguese}
 %\UseTblrTemplate{caption-tag}{default}{\bfseries}

%%% ==================================
%  Aqui entra o input da tabela
%
%%% 


\begin{longtblr}[theme=fancy,
	caption = {Classificação das Funções Oxigenadas},
	%note{a} = {It is the first footnote.},
	]{
       colspec = {c c c }, colsep = 17mm, hlines = {2pt, white},
		%row{odd} = {azure8}, row{even} = {gray8},
		row{1} = {2em,Coral1,fg=white,font=\bfseries\sffamily},
		row{2} = {2em,azure3,fg=white,font=\bfseries\sffamily},
		%row{12} = {bg=gray8, font=\bfseries},
		rowsep=.5cm,		
	}
  \hline
  \SetCell[c=3]{h,5cm} Funções Oxigenadas & & \\ 
  Nome  & Grupo & Exemplo   \\
  Álcool & \chemfig{R-OH} & \chemname[]{\chemfig{CH_3-CH_2-OH}}{Etanol} \\
  Enol & \chemfig{R-[:30](-[:90]OH)=[:330]R} & \\
  Fenol & \chemfig{*6(-=-=(-OH)-=)}
  
\end{longtblr}



 \end{document}
