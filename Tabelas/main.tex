\documentclass[12pt]{scrartcl}
\KOMAoptions{
	%headings=chapterprefix,
	%twocolumn=true,
	%toc=indenttextentries,
	%toc=flat,
	twoside=true,
	headinclude=true,
	footinclude=true
	%  captions=topbeside
}

\usepackage{fontspec}
\usepackage[T1]{fontenc}
\usepackage{hyperref}
\usepackage[x11names,svgnames,table]{xcolor}
\defaultfontfeatures{Ligatures=TeX}
%\setmainfont{Lato}
%\setmainfont{Charis SIL}
%\setmainfont{Alegreya}

\usepackage{typearea}
\usepackage{lscape}
\usepackage[a4paper]{geometry}
\geometry{a4paper,total={170mm,257mm},left=10mm,right=10mm, top=15mm, bottom=20mm}
\usepackage[portuguese, american]{babel}
\usepackage{amsmath,amsfonts,amsthm,bm}

\usepackage{xcolor}
\usepackage{tabularray}
\usepackage{chemmacros}
\usepackage{chemfig}
  \setchemfig{
   	angle increment=30,
     atom sep=1.67em,
     double bond sep=0.67ex,
     bond style={line width=0.1em},
     cram width=0.8ex,
     cram dash width=0.1em,
     cram dash sep=0.2em,
     arrow style={line width=0.067em},
     arrow head=-{Triangle},
     arrow label sep=1ex,
     cycle radius coeff=0.75,
     chemfig style={line width=0.1em},
 }
 
 \begin{document}
 
 \NewTblrTheme{fancy}{
 	\SetTblrStyle{caption-tag}{font=\bfseries}
 }
 
 
\selectlanguage{portuguese}
 %\UseTblrTemplate{caption-tag}{default}{\bfseries}

%%% ==================================
%  Aqui entra o input da tabela
%
%%% 
%\KOMAoptions{pagesize,paper=landscape,DIV=20}

\begin{longtblr}[theme=fancy,
	caption = {Classificação das Funções Oxigenadas},
	%note{a} = {It is the first footnote.},
	]{
       colspec = {c c c }, colsep = 12mm, hlines = {2pt, white},
		%row{odd} = {azure8}, row{even} = {gray8},
		row{1,12} = {1em,Coral1,fg=white,font=\bfseries\sffamily}, 		rowsep=.1cm,		
		row{2} = {1em,azure3,fg=white,font=\bfseries\sffamily},
		%row{12} = {bg=gray8, font=\bfseries},
		rowsep=.3cm,		
	}
  \hline
  \SetCell[c=3]{h,5cm} Funções Oxigenadas & & \\ 
  Nome  & Grupo & Exemplo   \\
  Álcool & \chemfig{R-OH} & \chemname[]{\chemfig{CH_3-CH_2-OH}}{Etanol} \\ \hline
  Enol & \chemfig{R-[:30](-[:90]OH)=[:330]R} & \chemfig{-[:30]-[:330]=[:30](-[:330])-[:90,,,1]OH} \\ \hline
  Fenol & \chemfig{*6(-=-=(-OH)-=)} & \chemfig{*6(-=-=(-OH)-=)} \\ \hline 
  Cetona & \chemfig{R-[:30](=[:90]O)-[:330]R} & \chemfig{-[:30](=[:90]O)-[:330]} \qquad \chemfig{O=[:90]-[:36]-[:108]-[:180]-[:252](-[:324])} \\ \hline
  Epóxido & \chemfig{R-[:30](-[:135]R)-[:60]O-[:300](-[:180])(-[:45]R)-[:330]R} &  \chemfig{O-[:141]-[:213]-[:285]-[:357](-[:69]\phantom{O})}\\ \hline 
  Éter & \chemfig{R-O-R} & \chemfig{-[:30]-[:330]O-[:30]} \\ \hline 
  Ácido Carboxílico & \chemfig{R-[:30](-[:330,,,1]OH)=[:90]O} & \chemfig{OH-[:150,,1](=[:90]O)-[:210]-[:150]-[:210]} \\ \hline 
  Éster & \chemfig{R-[:30](=[:90]O)-[:330]O-[:30]R} & \chemfig{-[:210]O-[:150](=[:90]O)-[:210]-[:150]-[:210]}\\ \hline
  Sal Orgânico &  \chemfig{R-[:30](=[:90]O)-[:330]\charge{45:1.5pt=$\scriptstyle-$}{O}-\charge{45:1.5pt=$\scriptstyle+$}{M}} & \chemfig{H_3C-[:30](=[:90]O)-[:330]\charge{45:1.5pt=$\scriptstyle-$}{O}-\charge{45:1.5pt=$\scriptstyle+$}{Na}} \\ \hline
    \SetCell[c=3]{h,5cm} Funções Nitrogenadas & & \\  \hline 
    Amina && \\ \hline 
    Amida && \\ \hline 
    Imina && \\  \hline 
    Isonitrila && \\ \hline 
    Nitrila && \\ \hline 
\end{longtblr}



 \end{document}
