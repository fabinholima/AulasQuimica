\documentclass[12pt]{scrartcl}
\KOMAoptions{
	%headings=chapterprefix,
	%twocolumn=true,
	%toc=indenttextentries,
	%toc=flat,
	twoside=true,
	headinclude=true,
	footinclude=true
	%  captions=topbeside
}

\usepackage{fontspec}
\usepackage[T1]{fontenc}
\usepackage{hyperref}
\usepackage[x11names,svgnames,table]{xcolor}
\defaultfontfeatures{Ligatures=TeX}
%\setmainfont{Lato}
\setmainfont{Charis SIL}
%\setmainfont{Alegreya}

\usepackage{typearea}
\usepackage{lscape}
\usepackage[a4paper]{geometry}
\geometry{a4paper,total={170mm,257mm},left=10mm,right=10mm, top=15mm, bottom=20mm}
\usepackage[portuguese, american]{babel}
\usepackage{amsmath,amsfonts,amsthm,bm}

\usepackage{xcolor}
\usepackage{tabularray}
\usepackage{chemmacros}
\usepackage{chemfig}
 
 
\begin{document}








	\NewTblrTheme{fancy}{
		\SetTblrStyle{caption-tag}{font=\bfseries}
	}
	
\selectlanguage{portuguese}
 %\UseTblrTemplate{caption-tag}{default}{\bfseries}



\begin{center}
\begin{talltblr}[theme=fancy,
caption = {Entalpias padrão de formação.},
%note{a} = {It is the first footnote.},
]{
colspec = {c c c c }, colsep = 7mm, hlines = {2pt, white},
%row{odd} = {azure8}, row{even} = {gray8},
row{1} = {2em,azure3,fg=white,font=\bfseries\sffamily},
%row{12} = {bg=gray8, font=\bfseries},
}
Composto  & Fórmula Química & Estado  & Entalpia {\(\Delta\)H$^\circ_f$  \unit{\kilo\joule\per\mol}} \\
Ácido sulfúrico                     & \ch{H2SO4}                     & líquido & -814,0   \\ \hline
\SetCell[r=2,c=1]{c}Água           					    &\SetCell[r=2,c=1]{c} \ch{H2O}    			  & gasoso  & -241,8   \\
                                    &                           & líquido & -285,8   \\ \hline 
Amoníaco                            & \ch{NH3}                       & gasoso  & -46,1    \\ \hline
Benzeno                             & \ch{C6H6}                      & gasoso  & +82,9    \\
Butano                              & \ch{CH3CH2CH2CH3}              & gasoso  & -126,2   \\
Carbonato de sódio                  & \ch{Na2CO3}                    & gasoso  & -1 131,0 \\ \hline 
\SetCell[r=4]{c}Cloreto de sódio   & \SetCell[r=4]{c} NaC$\ell$     & aquoso  & -407,0   \\ 
                                    &                           & gasoso  & -181,4   \\
                                    &                           & líquido & -385,9   \\
                                    &                           & sólido  & -411,1   \\ \hline
Dióxido de Nitrogênio                    & \ch{NO­2}                      & gasoso  & +33,0    \\ \hline 
Dióxido de carbono                  & \ch{CO­2}                      & gasoso  & -393,5   \\ \hline
Dióxido de enxofre                  & \ch{SO­2}                      & gasoso  & -297,0   \\ \hline
Dodecano                            & \ch{C12H26}                    & gasoso  & -291,0   \\ \hline
Etano                               & \ch{CH3CH3}                    & gasoso  & -84,7    \\ \hline
\SetCell[r=2]{c} Etanol             & \SetCell[r=2]{c} \ch{CH3CH2OH} & gasoso  & -235,3   \\
                                    &                           & líquido & -277,7   \\ \hline
Etileno (eteno)                     & \ch{CH2=CH2}                    & gasoso  & +52,3    \\ \hline
Etino (acetileno)                   & \chemfig{HC~CH}                      & gasoso  & +226,7   \\ \hline
Hidróxido de amónio                 & \ch{NH4OH}                     & aquoso  & -80,8    \\ \hline
\SetCell[r=2]{c}Hidróxido de sódio  & \SetCell[r=2]{c} NaOH     & aquoso  & -469,6   \\
                                    &                           & sólido  & -426,7   \\ \hline 
Metano                              & CH4                       & gasoso  & -74,9    \\ \hline 
\SetCell[r=2]{c} Metanol            & \SetCell[r=2]{c} \ch{CH3OH}    & gasoso  & -200,7   \\
                                    &                           & líquido & -238,7   \\ \hline
Monóxido de nitrogênio                   & NO                        & gasoso  & +90,0    \\ \hline
Monóxido de carbono                 & CO                        & gasoso  & -110,5   \\ \hline
\SetCell[r=2]{c} Nitrato de sódio    & \SetCell[r=2]{c} \ch{NaNO3}    & aquoso  & -446,2   \\
                                    &                           & sólido  & -424,8   \\ \hline
\SetCell[r=2]{c} Octano             & \SetCell[r=2]{c} \ch{C8H18}    & gasoso  & -208,5   \\
                                    &                           & líquido & -250,0   \\ \hline
Peróxido de hidrogénio              & \ch{H2O2}                      & gasoso  & -136,3   \\ \hline
Propano                             & \ch{CH3CH2CH3}                 & gasoso  & -103,9   \\ \hline
(Propeno (propileno)                & \ch{C3H6}                      & gasoso  & +20,4    \\ \hline
Sílica                              & \ch{SiO2}                      & sólido  & -911,0  \\ \hline
Cloreto de prata 			  & \ch{AgC$\ell$} & sólido & -127  \\ \hline 
\end{talltblr}
\end{center}



\end{document}