\documentclass[12pt]{scrartcl}
\KOMAoptions{
	%headings=chapterprefix,
	%twocolumn=true,
	%toc=indenttextentries,
	%toc=flat,
	twoside=true,
	headinclude=true,
	footinclude=true
	%  captions=topbeside
}

\usepackage{fontspec}
\usepackage[T1]{fontenc}
\usepackage{hyperref}
\usepackage[x11names,svgnames,table]{xcolor}
\defaultfontfeatures{Ligatures=TeX}
%\setmainfont{Lato}
\setmainfont{Charis SIL}
%\setmainfont{Alegreya}

\usepackage{typearea}
\usepackage{lscape}
\usepackage[a4paper]{geometry}
\geometry{a4paper,total={170mm,257mm},left=10mm,right=10mm, top=15mm, bottom=20mm}
\usepackage[portuguese, american]{babel}
\usepackage{amsmath,amsfonts,amsthm,bm}

\usepackage{xcolor}
\usepackage{tabularray}
\usepackage{chemmacros}
\usepackage{chemfig}
\setchemfig{fixed length=false, atom sep=2.0em, arrow offset=6pt, scheme debug=false,angle increment=30}


\begin{document}
	
	
	
	
	
	
	
	
	
	
	\NewTblrTheme{fancy}{
		\SetTblrStyle{caption-tag}{font=\bfseries}
		\SetTblrInner[tblr,longtblr]{rowsep=2.5pt}
		\DefTblrTemplate{firsthead, middlehead,lasthead}{default}{} % <---
		\DefTblrTemplate{contfoot-text}{normal}{\scriptsize\textit{Continued on the next page}}
		\SetTblrTemplate{contfoot-text}{normal}
	}
	
	\selectlanguage{portuguese}
	%\UseTblrTemplate{caption-tag}{default}{\bfseries}
	

{
	\centering
%	\refstepcounter{table}
	
%\SetTblrInner{rowsep=3.5pt}

%\begin{longtable}{NNBBB} 
\begin{longtblr}[theme=fancy,
	caption = {Grupos substituintes orgânicos formados por carbono e hidrogênio},
	%note{a} = {It is the first footnote.},
	]{
		colspec = {c c c c }, colsep = 7mm, hlines = {2pt, white},
		%row{odd} = {azure8}, row{even} = {gray8},
		row{1} = {2em,azure3,fg=white,font=\bfseries\sffamily},
		%row{12} = {bg=gray8, font=\bfseries},
		rowsep=.5cm,		
	}
 \hline
	Grupos Alquila  & Radical & Estrutura   \\
	1 carbono & metil & \chemfig{-CH_3}
	 \\ \hline
	2 carbonos & etil & \chemfig{-CH_2-CH_3} \\
	\hline 
	3 carbonos & propril &  \chemfig{-CH_2-CH_2-CH_3} \\ \hline 
	 & isopropil &  \chemfig{-CH([:-90]-CH_3)-CH_3}  \\
	  \\ \hline
	\SetCell[r=4]{m,2.5cm} 4 carbonos & butil &  \chemfig{-CH_2|{(CH_2)_2}CH_3} \\ \hline
	 & isobutil &  \chemfig{-CH_2-CH([:-90]-CH_3)-CH_3} \\ \hline 
	& \emph{s}-butil (\emph{sec}-butil) &  \chemfig{-CH([:-90]-CH_3)-CH_2CH_3} \\ \hline
	& \emph{t}-butil (\emph{terc}-butil) & \chemfig{-C([:90]-CH_3)([:-90]-CH_3)-CH_3}
	 \\ \hline 
	5 carbonos & pentil &  \chemfig{-CH_2|{(CH_2)_3}CH_3} \\ \hline
	  & isopentil &  \chemfig{-CH_2-CH_2-CH([:-90]-CH_3)-CH_3} \\  \hline
	& neopentil & \chemfig{-CH_2-C([:-90]-CH_3)([:90]-CH_3)-CH_3}  \\ \hline
	& \emph{t}-pentil (\emph{terc}-pentil) & \chemfig{-C([:90]-CH_3)([:-90]-CH_3)-CH_2-CH_3}\\  \hline 
	\SetCell[r=4,]{m,2cm}Outros grupos &  vinil ou etenil & \chemfig{-CH=CH_2}  \\  \hline
	& isopropenil & \chemfig{-C([:-90]-CH_3)=CH_2}\\ \hline
	& propenil & \chemfig{-CH=CH-CH_3} \\ \hline
	& ali ou propen-2-il & \chemfig{-CH_2-CH=CH_2}\\ \hline 
	\SetCell[r=4,]{m,2cm} Aromáticos & fenil & \chemfig{-(*6(-=-=-=))}  \\ \hline
	& naft-1-il & \chemfig{*6(-=(*6(-=-=(-)--))-=-=)}\\ \hline
	& benzil & \chemfig{-CH_2-(*6(-=-=-=))} \\ \hline
	& naft-2-il & \chemfig{*6(-=(*6(-=-(-)=--))-=-=)}\\
	\hline 
\end{longtblr}
}

\end{document}