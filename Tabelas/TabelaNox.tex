\documentclass[10pt]{scrartcl}
\KOMAoptions{
	%headings=chapterprefix,
	%twocolumn=true,
	%toc=indenttextentries,
	%toc=flat,
	twoside=true,
	headinclude=true,
	footinclude=true
	%  captions=topbeside
}

\usepackage{fontspec}
\usepackage[T1]{fontenc}
\usepackage{hyperref}
\usepackage[x11names,svgnames,table]{xcolor}
\defaultfontfeatures{Ligatures=TeX}
%%\setmainfont{Lato}
%\setmainfont{Charis SIL}
\setmainfont{Alegreya}

\usepackage{typearea}
\usepackage{lscape}
\usepackage[a4paper]{geometry}
\geometry{a4paper,total={170mm,257mm},left=10mm,right=10mm, top=15mm, bottom=20mm}
\usepackage[portuguese, american]{babel}
\usepackage{amsmath,amsfonts,amsthm,bm}

\usepackage{xcolor}
\usepackage{tabularray}
\usepackage{chemmacros}
\usepackage{chemfig}
 
 
\begin{document}
	\NewTblrTheme{fancy}{
		\SetTblrStyle{caption-tag}{font=\bfseries}
	}
	
\selectlanguage{portuguese}
 %\UseTblrTemplate{caption-tag}{default}{\bfseries}
\begin{talltblr}[theme=fancy,
caption = {Tabela de Nox de alguns elementos},
%note{a} = {It is the first footnote.},
]{
colspec = {cccl}, colsep = 5mm, hlines = {2pt, white},
%row{odd} = {azure8}, row{even} = {gray8},
row{1} = {2em,azure3,fg=white,font=\bfseries\sffamily},
}
Elementos & Nox & Ocorrência & Exemplo \\
Substâncias simples & zero & Substâncias simples & \ch{O2}, C, Fe , Mg\\ \hline 
{Metais alcalinos \\ Li, Na, K, Rb, Cs, Fr} & +1 & Substâncias compostas & NaF, \ch{Na2O} \\ \hline 
{Metais alcalinos terrosos \\ Be, Mg, Ca, Ba, Sr, Ra} & +2 & Substâncias compostas & CaO, \ch{BeC$\ell$2} \\ \hline
{Calcogênios \\ S, Se, Te} & -2 & Substâncias compostas & \ch{H2S}, \ch{CS2}, BaS \\ \hline
{Halogênios \\ F, \ch{C$\ell$}, Br, I} & -1 & Substâncias compostas & HF, NaBr, \ch{CaI2}\\ \hline 
{Prata \\ Ag} & +1 & Substâncias compostas  & AgF, AgS, \ch{AgNO3} \\ \hline 
{Zinco \\  Zn} &  +2  & Substâncias compostas & \ch{ZnC$\ell$2}, ZnS, ZnO \\ \hline
{Alumínio \\ \ch{A$\ell$}} & +3 &  Substâncias compostas & \ch{A$\ell$C$\ell$3}, \ch{A$\ell$2O3}, \ch{A$\ell$F3} \\ \hline 
\SetCell[r=2]{c} {Hidrogênio \\ H} &  +1  & Ligado aos não metais &  \ch{CH4} , \ch{H2SO4} \\ \hline
 & -1& {Ligado aos metais  \\ alcalinos e alcalinos terrosos} & NaH, \ch{A$\ell$H3}, \ch{CaH2} \\ \hline 
\SetCell[r=6]{c}{Oxigênio \\  O} &  -2 &  Maioria dos compostos & \ch{H2SO4} , \ch{KMnO4}, \ch{HNO3} \\ \hline
 & -2  & Óxidos  compostos binários &  \ch{Na2O}, \ch{H2O}, CaO \\ \hline
& -1/2  & Superóxidos, compostos binários & \ch{Na2O4} \\ \hline
& -1 & peróxidos compostos binários &  \ch{H2O2}\\ \hline
& +1  & Fluoretos & \ch{O2F2} \\ \hline
& +2  &Fluoretos & \ch{OF2} \\ \hline
\end{talltblr}

\end{document}